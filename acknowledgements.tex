
\begin{acknowledge}

I would like to thank my advisors, Manoel Couder and Michael Wiescher,
for helping me as I explored and discovered my path in science and in
life. The major achievements in my graduate career would not have been
possible without their guidance each day and their understanding as I
struggled through the low points in my journey. I would also like to
thank the remainder of my committee, Drs.\ Daniel Bardayan, Mark Caprio,
and John LoSecco, for guiding me through the candidacy and defense
process, and being understanding with my completion-from-afar and the
scheduling madness that entailed.

The entirety of this project would be impossible without the excellent
support I had throughout the Nuclear Science Lab and the Department of
Physics. In particular, Daniel Robertson and Edward Stech were
invaluable for learning standard operating procedures throughout the
lab, especially with the accelerators, targets, and detector systems,
and how to be a graduate student within the NSL. Thank you to Jerry
Schur for helping me with every network problem that I happened to
forget the solution to at precisely the wrong time. Thank you to the
amazing work done both within and without the machine shop from Dave
Futa, Jerry Lingle, Bradley Mulder, and Matt Sanford. There have been
more times than I could count where this project would not have moved
forward without you. Thank you to the excellent support from Susan
Baxmeyer and Shari Herman, who never failed to cheer me up when I saw
them in the office and who helped me navigate through the parts of
graduate school life that I had no idea of what to do otherwise. Thank
you to Janet Weikel for helping me in countless ways and always being a
happy face as I entered and left the NSL.

Thank you to the greater St.\ George group, past and
present\textemdash{} Manoel Couder, Jerry Hinnefeld, Zach Meisel, Gwen
Gilardy, Patricia Huestis, Edward Lamere, Luis Morales, Shane Moylan,
and Chris Seymour\textemdash{}for being the support for when tuning went
poorly and the source of jubilation when things went well. To those
future graduate students and postdocs within the group, I thank you for
taking this project on your own shoulders, and I hope this is a decent
starting point for your own work.

Thank you to Will Bauder, Stephanie Lyons Blyth, Matt Bowers, Hyu Soon
Jung, Wenting Liu, Alex Long, Karen Ostdiek, Karl Smith, Kiana
Setoodhar, and Ethan Uberseder, my former group mates, co-graduate
students, office mates, and guiding older scientists, for helping me as
I struggled to figure out who I was as a scientist and for helping me
when the work of physics became too much. Thank you to the other
graduate students and postdocs within the NSL and the Department of
Physics who, one way or another, not only helped make my time in
graduate school bearable but also enjoyable. Your names are far too
numerous to list here.

Thank you to Christina Marentette, my high school physics teacher and
one of the largest influences on my life trajectory. I would not be
where I am now without your guidance and example. Thank you to my crew
members on the Mars Desert Research Station rotation
89\textemdash{}Brian Shiro, Carla Haroz, Darrel Robertson, Luis Saraiva,
and Kiri Wagstaff\textemdash{}for being such good examples of what I
could be when I ``grew up''. A special thank you to Kiri, for
reintroducing me to python and setting me on my course to my life beyond
physics.

Thank you to the people I met through curling, both local and remote,
that gave me an outlet that turned into what will be a life-long
passion. Thank you especially to the founding South Bend Regional
Curling Club members\textemdash{}skip Dean Palmer, second Jared
Coughlin, lead Blair Vandenburg, and alternate Ralph
Lantz\textemdash{}who were an amazing happenstance at the end of my time
in South Bend. Competing in a national championship is something that I
will never forget.

Thank you to the friends I've made in that strange interim between
leaving graduate school and finishing graduate school: the fellows and
mentors at the Insight Data Science program, my co-workers and teammates
at Gartner, and the additional people I've met during my time in NYC.
Having an additional cohort of people interested in my progress helped
keep that progress from stalling, and being understanding of the time
and effort required without creating an undue burden was more help than
they'll ever know.

To those close friends that I've made in graduate school, thank you for
everything. You've each impacted my life in so many different ways that
listing everything out would triple the length of this dissertation. You
are better than I could have imagined as the people going through this
process with me.

Thank you to my friends from Michigan State: Nichole Hoerner, Gregory
Klein, and Ashleigh Winkelmann. Our continued friendship over the years
has made me realize how lucky I am to have stumbled into your lives.

Thank you to Pokie and Mike Olsen for allowing a young graduate student
into your home and for reawakening my love of board games. I can barely
remember a time before I had to struggle to pay for a 6-cost
development, and I didn't fully appreciate how amazing it was until you
left. Thank you to Michael Planer for being one of my first friends and
teaching me the wonders of having a whole bunch of extra T-shirts
around. Thank you to Will Bauder (and his wife, Laura) for introducing
me to so many things and showing me first-hand that you can become happy
after leaving graduate school. Thank you to Charlie Mueller for being an
amazing friend and a hilarious companion, everywhere between Red River
Gorge to that stretch of Cleveland Road. Thank you to Chris Wotta for
always being up for talking at length about programming and for bonding
over our mutual Michigan-ness. Thank you again to Jared Coughlin for
thinking that I'd be interested in being on his curling team and for
being a climbing buddy. Thank you to Joseph Hagmann for being that
guiding light that directly and indirectly helped me and my wife through
some of the best and worst times of our time in South Bend. We could not
have done this without you. Thank you to Kate Rueff for helping me
survive graduate school in more ways than one. You\textemdash{}and of
course, your dog Maxwell\textemdash{}have done so much for me, and I
don't think I could ever come close to repaying you. Thank you MacKenzie
Warren for being the sounding board of my thoughts and feelings from our
time together as roommates through the present day. You've shown me that
people can survive graduate school and come out better from it.

Thank you to Clark and Jacee Casarella and Alicia and Edward Lamere for
being our best friends. There's honestly no way to split the four of you
up in these acknowledgements, nor would I want to. I could not imagine
my time in graduate school without any of you. There are too many
memories, joyous and bitter-sweet and wonderful, and too many emotions
involved with writing this. Though the distance may physically separate
us as we are, scattered across states and time zones, there's nothing
that could keep us apart. Just simply, thank you for everything.

My family has been a huge party of who am I today. I am lucky to have
been surrounded by amazing people growing up: my siblings Caitlin,
Colleen, and Patrick; my cousins Brandon, Christopher, Lindsey,
Meredith, Billy, Missy, Sage, Katie, Kristen, and Kelly; my aunts and
uncles Michelle, Angelo, Rene\'{e}, Debbie, Buddy, Kathie, Chris, Trish,
and Teddy,; and my grandparents Francis, Thaddeus, Theresa, and William.
I am also extremely fortunate to have had my family grown during my time
in grad school: my cousin Grace; my sister-in-law Hannah; and my
brother-in-law Peter with his wife Jane and their kids Coulter and
Eloise; and my entire new extended-extended families. I'd especially
like to thank my parents-in-law Katie and Bruno for welcoming me into
their home and their lives early in my graduate career. Your additional
support helped in both the good times and bad, and having a second set
of loving and caring parents helped in more ways than I can say. I want
to thank my parents, Brigette and Mike, for raising me to question the
world around me and never be comfortable with where I was. You supported
me while I tinkered with Legos in the basement, through my academic and
athletic pursuits, and were always available to help me through my years
in graduate school. More than anything, I could not have imagined
attempting this had it not been for your nurturing and support all of
the years of my life.

Thank you to Link for being both a hassle and a joy at the end of my
graduate career. You reawakened my love of Chicken McNuggets, kept me
(too) warm while I lay on the couch or in bed, annoyed me as you scarfed
down discarded chicken wings, worried me as you went to and from the
vet, comforted me when I was sad, and relished in my joy with me. You
are the goodest boy.

Finally, I would like to thank my wife, Laura Amelia. There is a deep
feeling of happiness that I never felt before that you have brought into
my life, that you continue to bring into my life every single day. From
the first day I met you to me writing these very words, there is not a
moment spent with you that I would want to change. Everything that we
have done together\textemdash{}the trips to the Grand Teton National
Park; the move to NYC; getting a dog; making the long-haul drives
criss-crossing the country; surprising me with a trip to Cedar Pointe;
consoling and driving me to Detroit at the lowest point of my time in
grad school; helping me when I was down; smiling when I was up; caring
for me and loving me through everything\textemdash{}absolutely
everything that has happened in our lives together is a memory that I
would not want to lose, because those memories are with you. While this
segment of our lives together is now over, our journey together through
this wondrous world is only just beginning. I love you with all of my
being and everything that I have to offer to you. I love you so much,
Laura. Thank you.

\end{acknowledge}
