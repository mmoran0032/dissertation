\begin{abstract}
The St.\ George recoil separator is designed to measure $(\alpha,\gamma)$
cross sections of astrophysical interest in inverse kinematics. The design of
the separator allows for a relatively large energy ($\pm7.5$\,\%~$\Delta E/E$)
and angular ($\pm40$~mrad) acceptance that must be verified across a wide
range of electric and magnetic rigidities before primary experimental work can
begin. Additionally, the beam rejection properties of the separator system must
be determined to ensure that the direct incident beam is adequately rejected
such that the produced recoils can be confidently measured. The commissioning
work to experimentally verify these properties, and the procedures used, will
be discussed.

Utilization of the separator for measuring cross sections of astrophysical
interest outside of the design parameters is an additional benefit of the
commissioning work and expands the potential domain of study for the separator.
The first such experiment to measure two strong resonances in the \alpa{}
cross section has been completed. This reaction study is additionally a test of
the separators energy and angular acceptances \emph{in situ} as a precursor to
studying $(\alpha,\gamma)$ reactions. The results of this measurement in
relation to the properties of St.\ George and the astrophysical reaction rate
will be discussed. An analysis package and pipeline were developed to support
the study of the reaction in question and any future experiments using St.\
George or similar experiments.
\end{abstract}
