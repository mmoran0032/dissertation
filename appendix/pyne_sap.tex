\chapter{ANALYSIS PACKAGE}

The analysis and plots contained within this dissertation were completed
using Python and a small set of standard scientific python packages:


\begin{description}
    \item[Matplotlib] a 2D plotting package that supports multiple
        backends and output formats
    \item[NumPy] a standard numeric package that adds array (vector and
        matrix) computing
    \item[Pandas]
    \item[PyMC3]
    \item[SciPy] a scientific utilities package built on top of NumPy
        that includes general-purpose routines such as curve fitting, root
        finding, signal processing and more
\end{description}}

These three packages are standard components of the Python Scientific
Stack, and their usage and internals are well-documented and trusted by
many scientists in multiple fields. The usage of Python in the
scientific community has increased steadily over the years, with
multiple special purpose packages built on top of the foundation of
these three packages, particularly NumPy.

For the actual analysis, a two-part analysis framework was developed in
tandem with the analysis work: Python for Nuclear Experiments (PyNE) and
the St. George Analysis Package (SAP). These packages are designed to be
extensible by other research groups, guided by the requirements of the
St. George group, while providing an easy-to-use and understand
object-oriented interface to performing nuclear astrophysics research.
Development work was chronicled on the packages' GitHub
page\footnote{\url{https://github.com/mmoran0032/pyne}}, and may be
installed from there.

These packages would not be possible without two other important
packages: the ROOT\footnote{\url{http://root.cern.ch/}}
Data Analysis Framework\cite{ROOT} and
\texttt{evt2root}\footnote{\url{https://github.com/ksmith0/evt2root}}.
The
decision to not use these packages (and thus, C++) for the analysis will
be discussed following the description of the packages themselves.


\section{Python for Nuclear Experiments}


\section{St. George Analysis Package}


\section{Justification}

The \texttt{pyne} (Python for Nuclear Experiments) data framework is
based on two main analysis codes and frameworks: the \texttt{ROOT} data
analysis framework, and Dr. Karl Smith's \texttt{evt2root} conversion
utility. The pyne environment is designed to be similar to these
packages, but since its development was alongside the analysis done for
this experiment, additional functionality not needed for this analysis
was not included. In particular, the buffer file processing only handles
\texttt{.evt} and \texttt{.Chn} files, and the crate focus was on just a
single ADC module. Additionally, most of the analysis work with these
basic structures should be done using a separate package (such as the
sap (St. George Analysis Package)), and keeping the two responsibilities
of data conversion and structure and analysis of said data results in
cleaner code for the user.

Development of \texttt{pyne} started in March 2016, and the current
version can be found on the package's GitHub page. Within that same
repository is the \texttt{sap} package, but that code is not reproduced
here. The code listed is from version 0.6.0 (updated on May 12, 2017).
The packages outside of the Python standard library required for use are
\texttt{numpy}, a standard numeric package for scientific Python,
\texttt{scipy}, a scientific analysis package built on top of
\texttt{numpy}, and \texttt{matplotlib}, a 2D plotting interface. All
three of these packages are standards for scientific computing with
Python. Currently, only Python 3.X is supported. As official support for
Python 2.7 is set to terminate in 2020, and most scientific package
managers have already switched to Python 3.X, this decision is
warranted.
