\chapter{INTRODUCTION}

The elements making up the universe were formed during a variety of processes,
beginning with Big Bang Nucleosynthesis (BBN) that formed the lightest
elements. Those elements common to life on Earth were primarily formed through
burning processes inside of stars, grouped together under the title of
Stellar Nucleosynthesis. Depending on the conditions within the stellar
environment, which are characterized by macroscopic qualities about the star
(temperature, pressure, mass, etc.) and the elemental composition of the
stellar interior where the burning process takes place, the reactions
accessible to the nuclei within the star differ. The creation and destruction
of different elements and isotopes may be inhibited or enhanced by these
differing conditions, and the study of these processes at the nuclear level
has spawned the field of nuclear astrophysics in order to understand the inner
workings of these stars.

The study of these reactions has increasingly taken place within nuclear
accelerator laboratories, which can attempt to replicate the conditions inside
of stars in order to measure the qualities of the reactions as they take place
in the cosmos. Individual reactions may be isolated by the choice of the
accelerated beam particle and its properties, and the choice of target material
and its properties. As the underlying cross section governing the observed
properties of the reaction may change rapidly within a narrow band of energy,
immense amounts of work and effort has been expended in order to detect these
changes with enough precision in order to be confident that the cross section
has been accurately described. Whole classes of reactions and detection methods
have been devised to study those reactions previous out-of-reach to experiment,
either due to the energies in question or the complexities of detecting the
desired particles.

In order to properly measure a single cross section, a [COMPLETE?] knowledge of
the relevant physical processes and experimental methods is paramount.

% some more lead-in...?


\section{Astrophysically-important Reaction Channels}

% discuss stable hydrogen and helium burning phases, CNO cycle

% discuss s-process and ap-process

The specific and directed study of those nuclear reactions that have an effect
on the properties or life cycle of celestial bodies is grouped under the
umbrella term \emph{nuclear astrophysics}. These reactions may take place
during the standard lifecycle of a star, called \emph{stable burning}
processes; during the death throws of a star, called \emph{explosive burning}
processes; or during the grander timeline of the universe as a whole, among
other coarse groupings of reactions. A more-complete exploration of these
aspects of nuclear astrophysics was first laid out in [B2FH].

Within the realm of stable burning processes, changes in temperature commonly
result in reaction channels or sequences becoming either energetically
favorable or unfavorable.

The two primary reactions that will
be discussed, \alpa{} and \react{\nuc{14}{N}}{\alpha}{\gamma}{\mnuc{18}{F}},
are important reactions in a number of different reaction chains that take
place throughout the lifecycle of stars, and the impact of these reactions will
be explored.

\subsection{CNO Cycle}

The Carbon-Nitrogen-Oxygen (CNO) cycle is one pathway for stars to fuse
hydrogen into helium as a source of thermonuclear power. The pathway cycles the
catalytic nuclei to produce \nuc{4}{He} through the reaction sequences
\[
    \mreact{}{}{}{}
\]
The CNO cycles dominate the energy production of heavy stars with properties
[MASS and TEMPERATURE].

\subsection{$s$-Process Nucleosynthesis}

- end state of CNO is more 14N than anything else
- Breakout reaction feeds into s-process by generating seed nuclei
- s-process progresses in AGB stars over long timescales
- produces high mass nuclei near the valley of stability

\subsection{Other Channels}

- where does 27Al(p,a) fit in?


\section{Recoil Mass Separation}

When measuring a cross section, the experimenter must make a decision as to
whether to detect $\gamma$-ray particles or nuclear particles. Each of these
two overarching goals carries with it their own complications and
considerations, creating a variety of techniques to attempt to study the
reaction in question. For some reactions of interest within the domain of
low-energy nuclear astrophysics, both or these experimental directions are
possible.

Gamma-ray detection suffers from background radiation producing $\gamma$ rays
that are picked up by the detector. This background radiation is produced by
the construction materials within the laboratory in question, cosmic radiation,
and the $\gamma$-ray producing reactions involving contaminants within the
target material. The solutions for reducing the background from each of these
sources are varied, from surrounding the detector with additional detectors to
create an active shield that rejects detections that appear in both detectors
within some window of time (the anti-coincidence method), to the extreme case
of placing the laboratory deep underground to use hundreds of meters of the
Earth's crust as a passive shield against cosmic radiation. None of the
techniques, however, addresses the inherent limitation of $\gamma$-ray
detection: the low efficiency of the detectors that must be used. This problem
is exasperated by the extremely low cross sections common for reactions of
astrophysical interest, making studies away from nuclear reasonances uncommon.
The other irreducible problem with these techniques is when the produced
$\gamma$ ray of interest is close in or the same energy as a prominent
background line.
%, such as the 511~keV $e^+e^-$ annihilation peak.

Alternatively, the produced nuclear particles may be detected instead with a
high-efficiency detector. The various experimental requirements for these
experiments will be discussed, along with the additional considerations that
must be accounted for. For convenience, the rest of this discussion will assume
that the reaction of interest is radiative alpha capture ($(\alpha,\gamma)$),
where the reaction is being studied in reverse kinematics, except where
otherwise noted.

% This previous paragraph is weird... I'm not sure how to transition here

St. George (Stong Gradient Electromagnetic Online Recoil separator for capture
Gamma ray Experiments) is a recoil separator designed to primarily study
$(\alpha,\gamma)$ reactions at low energy using stable beams provided by St.
Ana (Stable beam Accelerator for Nuclear Astrophysics). The design of the
separator was guided by the principles to be discussed and on the experience
gained from previous recoil separators.


\subsection{Motivation}

Recoil mass separation was conceived as an alternate way to measure the cross
sections of radiative capture reactions. These reactions had previously been
studied by detecting the produced $\gamma$ rays, subject to the limitations
previously discussed. The heavy reaction product can instead be detected by a
detector situated behind the target, assuming that the target is thin enough to
allow the produced recoils to leave the target. In this thin target case, the
incident beam will likely pass through the target as well, making it a source
of background at the detector plane. In the cases of interest for nuclear
astrophysics, this background count rate could be $\times 10^{15}$ that of the
particles of interest and may cause damage to the detector.

The produced recoils may be filtered out from the incident beam by
electromagnetic elements situated between the target and the detector. The
interaction between the heavy incident beam with mass $A$ and linear momentum
$p$ and the $\alpha$ particles within
the target produces a heavy compound nucleus with mass $A + 4$ and momentum
$p$. Ignoring the effect of the emitted $\gamma$ ray on the momentum and
assuming that there is no spread in the momentum, the use of electrostatic
elements can separate the recoils from the beam based on their different
magnetic and electric rigidities, defined as
\[
    B\rho\rm{ [Tm] } = \frac{p}{q} = \frac{\sqrt{2mT}}{q}
\]
and
\[
    E\rho\rm{ [MV] } = \frac{pv}{q} = \frac{2T}{q},
\]
respectively, where $q$ is the charge, $T$ is the kinetic energy, $m$ is the
mass, and $v$ is the velocity of the particle. With a single momentum and
velocity (or kinetic energy) selected for, the recoil particles of interest can
be uniquely identified by the optical system. The design of recoil separators
make use of this relatively simple idea as the basis of their design. Despite
this, there have been relatively few recoil separators that have been brought
into service due to the complexities of their design and operation that are not
adequately taken into account in this description.


\subsection{Background Sources}

A recoil separator primarily utilizes two different methods to reduce the
background at the detector system: the separation of isotopes based on their
rigidities to reduce the number of incident beam particles that reach the
detector, and the direct detection of the heavy nuclear particles compared to
the detection of the $\gamma$ rays particles produced in the reaction at the
target location. These reduction techniques have been briefly introduced.

Within the separator, the rejected incident beam particles (those which do not
have the desired rigidities $B\rho$ and $E\rho$) are still traveling within
the separator. Their final rejection and the reduction of their induced
background at the detector plane must be performed by some physical element
interior to the separator. A common choice of element are slits: physical
barriers that stop any particle which strikes them located within the
separator. The physical gap between the slits allow the desired recoil
particles to pass between them while stopping the unwanted beam particles from
continuing down the separator. The location of these slits is dependent on the
design and the beam optics of the separator.

While these slits stop errant beam particles, there is a possibility of those
particles scattering off of the edge of the slit if the beam properties are
not properly matched to the internal position or gap of the slits. These
scattering events, depending on where in the separator they occur in relation
to the other magentic and electric elements and to the detector system, may
allow some passage of beam particles to the detector plane and must be either
rejected before that plane or taken into account in the operation of the
detector system. The scattering events may also take place off of the interior
of the vacuum chamber of the separator, the residual vacuum within the vacuum
chamber, or diagnostic equipment installed to aide the experimenter.

The reaction also does not take place at a single energy. The incident beam
invariably will have an energy spread associated with its production and will
lose energy within the target before reacting with the target particles. The
produced reaction products will necessarily not be mono-energetic, and
additional energy spread will result from the subsequent interaction with the
residual target. The reaction products also undergo either a single photon
emission or a gamma cascade which will alter the reaction products' momentum
vectors, adding futher spread to the momentum and energy of the products that
must be detected at the detector plane. Additional sources of background
reduction beyond the separator must be included due to the realities of
performing the experiment.


\subsubsection{Detection System}

The detector system following the separator may be used to provide this
additional required beam and background suppression. Due to the separation of
the recoil particles by their rigidity, a detector sensitive to both the energy
and the momentum of the particles can unique identify those particles. There
are multiple ways in which this can be achieved, but one common design is to
make use of a \textit{time-of-flight} (TOF) detector to determine the velocity of the
particles reaching the detector, and an energy-sensitive detector, measuring
the total energy deposited in the detector, to provide this identification.
Depending on the resolutions of the two systems, the recoils of interest can
be identified by the recorded TOF and $E$. More importantly, those other
particles that happen to reach the detector would have drastically different
TOF or $E$ or both, allowing those detections to be rejected as they did not
originate from recoil particles.

The two coupled detectors would be operated in coincidence, such that those
events that create a timing signal within the TOF detector must also create an
energy signal in the following energy-sensitive detector in order for the
event to be recorded. This operation further suppresses errant background
counts at the detector system. The timing window must be matched to the
properties and response of the detectors in order to ensure that the
coincidence measurements from the paired detectors arise from a single
particle.


\subsection{Recoil Separators}
\label{sec:prevwork}

% The implementation of recoil separation into experimental campaigns has been
% explored recently at a number of facilities, primarily with a focus on studying
% astrophysically relevant reactions. When considering each separator facility and
% the capabilities of that system, the separator must be considered alongside the
% target, detector, accelerator, source, and other essential elements within the
% entire laboratory.
% St.\ GEORGE (Strong Gradient Electromagnetic Online Recoil separator for
% capture Gamma ray Experiments, henceforth simply \emph{St.\ George}), is one
% recent recoil separator, installed at the Nuclear Science Laboratory (NSL)
% at the University of Notre Dame~\cite{Couder2008}.
% The design for St.\ George is based on knowledge gained by
% designing, constructing, and using these previous
% separator systems for scientific research.
% Where these systems are similar and
% differ from the system in place at the University of Notre Dame will be briefly
% discussed. Additionally, a brief description of the St.\ George separator will
% be included here and further discussed in Sec.~\ref{sec:stg}.

The use of recoil separators to study radiative capture reactions has been
explored recently at a number of facilities. The design of St. George is based
on the knowledge gained from the design, construction, and operation of these
previous recoil separator systems. The entire system, inclusive of the beam
source, target, and detector, must be discussed as a whole when evaluating the
capabilities of a given separator.


\subsubsection{CalTech Separator}

% The design and use of recoil separators for astrophysical studies was first
% pursued by Smith \textit{et al.}~\cite{Smith1991}. The requirements of radioactive
% beam studies required the need to develop new detection systems, especially
% considering the effect of differing beam property limits (intensity, purity,
% and emittance) and the desire to detect the produced nuclei to further probe
% the astrophysical conditions. As the initial feasibility system to provide a
% technical proof-of-concept, many of the design choices made and techniques
% used have been adopted by following separators. These include the use of a Wien
% filter for velocity selection, dipole magnets for momentum selection, an
% electrostatic deflector for energy selection, and the use of a gaseous target.
% Additionally, the use of a gamma-ray detector in coincidence with the final
% recoil detection and beam monitoring with an offset Si detector at the target
% location are also common choices that have been adapted at the other separators.

The design and use of recoil separators for nuclear astrophysics research was
pioneered by Smith *et al.*. This separator was a proof-of-concept design to
determine the feasibility of performing reaction studies with this technique.


\subsubsection{ARES}

The Astrophysics REcoil Separator (ARES) was built at Louvain-la-Neuve to
study $(\rm{p},\gamma)$ and $(\alpha,\gamma)$ reactions using radioactive
incident beams provided by the CYCLONE44 cyclotron~\cite{Angulo2001}.
Self-supporting solid
targets, containing the required H or He, were used for the reaction
studies. The system is designed with a single magnetic dipole for momentum
selection and a Wien filter for velocity selection, along with multiple
magnetic quadrupoles to maintain the transportation of the to the detector
system. The condensed and limited size of the separator is based on the
constraints of the experimental hall~\cite{Couder2003}.
The detector system consists of a single
$\Delta E − E$ telescope which separates out the reaction products from the
remaining incident beam particles. The initial test of the separator used a
stable incident beam to compare to results obtained by other methods within
the lab, and the focus of the initial work was on low-lying resonances of
astrophysical interest.


\subsubsection{DRAGON}

The DRAGON recoil separator at TRIUMF-ISAC was built for the same reasons as ARES,
but differs in the actual construction and usage of the separator.
% Success of separator
The separator
itself uses two large magnetic dipoles for momentum separation and to electric
dipoles for energy selection~\cite{Engel2005}. The separator also contains
steering elements within the beamline to aide in transporting the recoils to
the detector plane. The extended gas target is surrounded by a large BGO
gamma-ray detector for coincidence purposes

\subsubsection{ERNA}



\subsubsection{St. GEORGE}
The separator consists of six dipole magnets, eleven quadrupole magnets, and a
Wien filter. The separator was designed to accept recoils with a maximum
energy and angular spread of $\Delta E/E = \pm7.5\%$ and
$\Delta\theta = \pm40$~mrad, respectively, and to provide a mass separation
of $m/\Delta m = 100$ and beam suppression of a factor $\geq 10^{15}$. Combined
with the HIPPO (High-Pressure Point-like target) supersonic gas jet target,
St.\ George will be primarily used to study low energy $(\alpha,\gamma)$
reactions using stable beams.



\section{Beam Optics}

% quadrupole and dipole fields...?
