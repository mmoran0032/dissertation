
\chapter{COMMISSIONING ST.\ GEORGE}
\label{ch:commissioning}

The St.\ George recoil separator was designed to study a set of
$(\alpha,\gamma)$ reactions of astrophysical importance. To study these
reactions, an energy acceptance of $\Delta E/E = \pm7.5$\,\% and an
angular acceptance of $\Delta\theta = \pm40$~mrad were necessarily based
on the kinematic properties of those reactions~\cite{Couder2008}. When
the separator is set such that it achieves the optimal ion optics for
the system, the separator exhibits full acceptance of the recoils of
interest while maximizing the beam rejection; these settings must be
experimentally determined. These settings are the currents and voltages
required to operate the electromagnetic elements, which may differ from
those directly determined from the ion optics design. Measuring the
acceptance achieved through various tunes and techniques is one way to
determine if the desired transport properties have or have not been
achieved experimentally. Recently, the energy acceptance was studied
separately from the angular acceptance to provide a relation between the
predicted and experimentally determined field values~\cite{Meisel2017}.
The angular acceptance was then studied through a variety of methods,
both in conjunction and without a corresponding energy acceptance
requirement. Measuring the acceptance of a separator is paramount to
using it for experimental measurements.

Two primary commissioning campaigns, determining the angular and energy
acceptances for a given desired global setting of the separator, have
been undertaken. The two global settings are: (i) the designed
parameters for St.\ George for transporting heavy recoil products
produced by $(\alpha,\gamma)$ reactions in inverse kinematics through
the entire separator, and (ii) the modified parameters for transporting
$\alpha$ particles produced by $({\rm p},\alpha)$ reactions in forward
kinematics to focal plane $F_2$. From each global setting, the separator
elements can be scaled based on the desired particle's magnetic
(Eq.~\ref{eq:brho}) and electric (Eq.~\ref{eq:erho}) rigidities. The
experimental settings for the separator achieving the required transport
properties (acceptance and rejection) must be determined over a wide
range of rigidities in order to ensure that the scaled settings still
maintain the desired properties. Scaling the settings over a large range
must make assumptions about \emph{how} those elements scale which may or
may not be representative of the experimental situation. Thus,
determining the acceptance and rejection for a variety of particle
rigidities that adequately cover the allowed rigidities and those
expected for astrophysical reactions is a requirement for the full
commissioning of the separator system. For the modified tune settings to
measure $({\rm p},\alpha)$ cross sections, the required field settings
were determined by using a direct $\alpha$ beam at the expected
energies. This situation is a special case due to the relatively small
rigidity phase space covered by the experimental plan and the
availability of a direct beam of the desired species, which most likely
is not the case for many astrophysically-important reactions.

For the designed inverse kinematics separator settings, various beams
spanning a large region within the desired rigidity phase space were
used. Test beams included light beams (\nuc{1}{H} and \nuc{4}{He}) due
to their ease of production and heavier beams (\nuc{16}{O} and
\nuc{20}{Ne}) to simulate transporting heavy reaction products through
the separator. The available test beams and their properties were based
on the capabilities of the 5U and its ion source, including but not
limited to the species in question, the charge state, energy, and
intensity of the beam, and the overall stability of the system. Here
stability is diagnosed by the beam current at the diagnostic equipment:
a current with constant mean and small variance over time is considered
a ``stable'' beam. This stability is based on the entirety of the
experimental system, from the initial beam produced at the ion source to
the power supplies used for the various tuning elements along the beam
line. The commissioning work was divided between focusing on the energy
or angular acceptance relatively independent of the other. For some of
the angular acceptance measurements, a small energy spread was included
to better recreate the conditions under which St.\ George will be used.


\section{Theoretical and Experimental Considerations}
\label{sec:cosy}

% Possibly move this to Chapter 2...

St.\ George was modeled using COSY Infinity (henceforth \emph{COSY}), a
beam optics and transport language developed at Michigan State
University~\cite{COSY}. The initial ion optics solution for the
separator was calculated by Drs.\ Couder and Berg at the University of
Notre Dame to maximize the angular and energy acceptance for a
point-like target located prior to the separator. Optimization of the
individual elements' properties allowed the separator to achieve the
previously-stated energy and angular acceptances, create an achromatic
focus at the mass slits (focal plane $F_2$), and transport all recoils
to the final detector focal plane $F_3$ (see Section~\ref{sec:stg}).
Each magnetic element is represented by a single command within the
code, defining the type and properties of the desired element. The three
types of elements used within St.\ George\----{}dipoles, quadrupoles,
and the Wien filter\----{}require different sets of values to be
defined. The recoil envelope, consisting of a number of sample recoil
properties used as representative rays, for the final designed
configuration is shown in Fig.~\ref{fig:raytrace}. For the example
shown, the quadrupole pole tip fields are given in
Table~\ref{tab:poletip}, where negative values represent a quadrupole
focusing in the $y$-direction. The pole tip fields for $(\alpha,\gamma)$
experiments are for the test particles shown, while those for $({\rm
p},\alpha)$ experiments are specific to this work. The actual fields
used will depend on the rigidity of the desired particle to transport
through the separator and can be scaled from these values.

\textbf{IT WOULD BE NICE TO HAVE SOME INDICATION OF TRANSVERSE AND
LONGITUDINAL DIMENSIONS}

\begin{figure}
    \begin{center}
        \centerline{
            \includegraphics[width=0.8\textwidth]{figures/raytrace.png}}
        \caption[Horizontal and vertical rays through St.\
            George]{Horizontal (upper plot) and vertical (lower plot)
            rays through St.\ George. Recoil \nuc{41}{Sc} rays are shown
            in black and beam \nuc{40}{Ca} rays are shown in red.
            % The beam energy is $E_{\textrm{beam}} =
            % 15.99\pm0.001$~MeV, and the recoil energy is
            % $E_{\textrm{recoil}} = 15.6\pm1.155$~MeV.
            The beam rigidities are $B\rho = 0.331$~Tm and $E\rho =
            2.907$~MV, and the recoil rigidities are $B\rho = 0.331$~Tm
            and $E\rho = 2.836$~MV. Both the beam and recoil are in the
            $11^+$ charge state. Differing charge states would lead to a
            greater separation in the two beam profiles. The COSY
            calculation assumes that the recoil particles are spread
            within an acceptance range of $\Delta E/E \approx7.5$\,\%
            and $\Delta\theta = 40$~mrad. The transverse scale is highly
            exaggerated to show detail.}
        \label{fig:raytrace}
    \end{center}
\end{figure}

% INCLUDE THIS TO BREAK (NOT SURE IF THIS IS THE BEST WAY)
\newpage
The initial ion optics solution creates a transport map for particles
passing through the entire separator that can be analyzed independently
of the ray traces and provide the mathematical backing to the particles'
trajectories within St.\ George. The transport map is dependent on the
quantities
\begin{equation}
    \label{eq:cosyvars}
    \begin{split}
        r_1 &= x \\
        r_3 &= y \\
        r_5 &= l = -(t - t_0)v_0\gamma/(1 + \gamma) \\
        r_7 &= \delta_m = (m - m_0)/m_0
    \end{split}
    \quad\quad
    \begin{split}
        r_2 &= a = p_x/p_0 \\
        r_4 &= b = p_y/p_0 \\
        r_6 &= \delta_K = (K - K_0)/K_0\\
        r_8 &= \delta_z = (z - z_0)/z_0,
    \end{split}
\end{equation}
where $a$ and $b$ are treated similarly to angles as a first
approximation within each plane, $\delta_K$ is the relative energy
difference from the desired energy $K_0$, $\delta_m$ is the relative
mass difference from the desired mass $m_0$, and $\delta_z$ is the
relative charge from the desired charge state $z_0$~\cite{COSY}. These
values are related to the parameters in the transport map (see
Sec.~\ref{sec:ch1-beam-optics}), but defined for use within the code.
The desired quantities are the values used to calculate the magnetic
(Eq.~\ref{eq:brho}) and electric (Eq.~\ref{eq:erho}) rigidity, and thus
set the fields of the elements within St.\ George, of the particle to be
transported through the entirety of the separator. The time of flight
difference $l$ is not considered in analyzing the separator. Within the
transport map, aberrations up to fourth order were calculated, with
terms up to third order found to affect the design~\cite{Couder2008}.

The original ion optics calculation describes the fringe fields of the
optical elements using the default parameters provided by COSY. A change
in the shape of the fringe field can change the trajectory of the
particles within the separator, as the total field that the particle
interacts with changes in magnitude. Since the fringe fields used to
find the ion optics solution and those created by the actual magnetic
elements within St.\ George may be different, the required field
strength may also be different. Field maps for each of the magnetic
elements were measured by the production company in order to understand
the dependence of the field strength and fringe fields at differing
excitations of the magnet. The pole tip fields for a given particle
rigidity, determined by the current set point for that magnet, must be
found experimentally. The procedures for each of the three different
types of elements necessarily differ based on what diagnostic equipment
is available.

The magnetic settings for the dipole magnets, determined by the rigidity
of the particles, can be determined by observing the trajectory of the
particles within the separator. This trajectory can be directly observed
using diagnostic equipment aligned with the optical axis. The final
setting of the dipole magnets must be found by observing the effect of
the focusing quadrupole magnets following the dipole. If the beam is
aligned with the magnetic optical axis of the quadrupoles, the off-axis
steering effects will be minimized. This fine adjustment to the dipole
fields is within a small window of current settings for the dipole near
the coarse value found through direct observation of the beam.

Setting the Wien filter can be done in a similar manner. The electric
field strength is determined from the known energy of the particle, the
known charge state, and the desired bending radius of the filter. This
bending radius is the radius of the particle's trajectory if only the
electric or magnetic field were operating. Thus, the electric field
strength can be set to an exact value, requiring the magnetic field to
be set to match the properties of the electric field. The strength of
the magnetic field is set such that the bending radii of the two fields
are the same and so that the particle beam continues along the optical
axis in the same manner as described previously with the standard
dipoles. The Wien filter was designed with magnetic field clamps at the
entrance and exit of the filter to match the magnetic fringe field to
the electric fringe field. These clamps must be adjusted to their proper
positions before the magnetic field can be set.

The magnetic quadrupoles require a more complex procedure in order to
set their fields to the desired values. This complication arises from
the inability to directly observe the trajectory of the focused
particles within the separator at all possible angles and energies
concurrently, and at multiple points within the separator itself. If
these trajectories match the trajectories expected from the beam optics
calculation, then the acceptance and rejection properties of the
separator would be the desired values. Due to the inability to do this,
the quadrupole tuning procedure must be adapted to work with the
diagnostic equipment available. The full procedure is outlined in
\ref{sec:tuning_stg}.


\section{Separator Properties}

% St George was installed the same year that I started grad school
The elements, power supplies, and supports were provided by Bruker
Biospin and installed in 2011. The separator design requirements for the
strengths of the optical elements were based on the maximum beam energy
of the older KN single-ended Van de Graff accelerator and the possible
charge states produced by its internal ion source. The 5U and ion source
have similar properties to this system. The power supplies for the
magnets provide highly stable direct currents for each magnet
individually, with $dI/I \approx 10^{-4}$ for the quadrupoles and $dI/I
\approx 10^{-5}$ for the dipoles. The upper current limit is different
for each magnet. The separator uses a robust water cooling system able
to maintain the required $80\pm2$~\degree{}F magnet temperature for the
entire system. The system is able to maintain the temperature even when
all magnets are at their maximum currents for extended periods of time.

The Wien filter electrode power supplies are set separately based on
their voltage, with voltage stability $dV/V \approx 10^{-5}$ in the
range commonly used for experiments. The upper limits for these power
supplies are $\pm110$~kV, with voltages below $\approx 70$~kV used
during previous work. In order to reach electric potentials near the top
of the range, the voltages need to be slowly ramped up to ``condition''
the plates at the higher voltage. This conditioning is required due to
minor imperfections on the surface of the electrostatic plates,
differences in the residual vacuum within the vacuum chamber, and
buildup of C deposits on the plates and interior walls. Directly setting
the plates to higher voltages without conditioning the plates would lead
to large and frequent discharges of the electrostatic plates, preventing
the Wien filter from being used in stable running conditions. For
voltages above $\approx 50$~kV, the plates were conditioned at settings
at least 10~kV above the desired set point to provide a stable running
condition. For lower potentials, no conditioning is necessary unless the
vacuum chamber was recently vented (exposed to atmospheric pressure
gases).

The properties of each of the electromagnetic elements (entrance and
exit apertures, length, maximum field strength, good field region, etc.)
were determined within the ion optics solution to transport the desired
recoils, and built to match those specifications. Higher order
corrections to the particle trajectory were achieved by shaping the
entrance and exit faces of the dipoles instead of using higher order
multipole magnets~\cite{Couder2008}. Additionally, the shape of the Wien
filter electrostatic plates were designed such that the electric and
magnetic fields, including the fringe fields, were closely matched.


\subsection{Magnetic Fringe Fields and Effective Field Lengths}

Detailed two-dimensional magnetic field maps for multiple excitations of
each magnet were provided by Bruker. The field maps allow a check on the
good field region for each magnet and provide a description of the
fringe fields. Field strengths at each location (distance along the beam
axis and at a radial distance from the beam axis) are measured in mT.
From this data, the shape of the fringe field and the effective field
length of the magnetic elements can be determined. The effective field
length, defined as
\begin{equation}
    \label{eq:efl}
    L = \frac{1}{B_0}\int_{-\infty}^{\infty} B(z)\, \textrm{d}z,
\end{equation}
where $B_0$ is the field strength at the center of the magnet, is the
field length if the field were described with a pure ``hard edge'' or
Heaviside function at the entrance and exit, i.e.\ no fringe fields. As
these values are essential to setting the necessary values for the
quadrupoles, the analysis of the field maps focused on these elements.

A single edge fringe field is described by the Enge function given by
\begin{equation}
    \label{eq:enge}
    E(z) \equiv \frac{1}{1 +
        \exp\left[\sum_{i=0}^{N-1}{a_i}(\frac{-z}{D})^{i}\right]},
\end{equation}
where $a_i$ are the desired expansion coefficients, $D$ is the aperture
diameter, and $z$ is the longitudinal distance~\cite{Baartman2007}. The
formulation above is used within COSY to describe user-defined fringe
fields. For a short magnet, which the St.\ George quadrupoles can be
considered to be, the entrance and exit fringe fields are not completely
independent of each other, since the fringe fields extend into the
central region of the magnet. Instead of fitting each fringe field
separately, we can instead fit the entirety of the magnetic field
profile using a combined ``short'' quadrupole function, in terms of the
Enge function, given by
\begin{equation}
    \label{eq:shortquad}
    k(z) = k_0\left[E(L/2 + z) + E(L/2 - z) - 1\right],
\end{equation}
where $k_0$ is a scaling parameter for the central field and $L$ is the
effective field length~\cite{Baartman2007}. This formulation assumes a
symmetric field profile, as both the entrance and exit fringe fields are
modeled with the same Enge function.

Using the field maps provided by Bruker, we can determine the Enge
coefficients and the effective field lengths for our magnets. For all
calculations, since the field near the center of the magnet is
relatively weak, the fields within 2~cm in the radial direction of the
central axis were not used for determining either the effective field
length or the Enge coefficients. Additionally, the effective field
length and the shape of the fringe field were assumed to not differ with
different magnet excitations, so all available data were used for each
magnet at the same time. An example using the penultimate quadrupole
$Q_{10}$ of the normalized fields used and the resulting fit is shown in
Fig.~\ref{fig:enge_fit}.

\begin{figure}[h]
    \begin{center}
        \centerline{
            \includegraphics[width=0.85\textwidth]{figures/enge_fit.png}}
        \caption[Normalized field with Enge fit]{Normalized field and
            the resultant fit to the fringe field for the example
            quadrupole $Q_{10}$. The parameters of the fit are given in
            Table~\ref{tab:enge}.}
        \label{fig:enge_fit}
    \end{center}
\end{figure}

The effective field lengths were calculated directly from the field maps
by integrating along the $z$-direction for each radial distance
provided. Since the maximum field strength for a given magnet current
varies depending on the distance from the center, the individual
``traces'' of the magnetic field along the $z$-axis were normalized.
This normalization is shown in Eq.~\ref{eq:efl} as the constant factor
outside of the integral. The integration was performed using the
Simpson's Rule routine provided by the SciPy Python
package~\cite{SciPy}. An average of these lengths was used. Differences
between the calculated effective field length and those used within the
initial ion optics solution were within 2\,\%.

Using the same normalized field ``traces'' along the $z$-axis, the Enge
coefficients describing the shape of the fringe field may be determined.
The field profiles at each radial distance were fit simultaneously.
Using the default Enge coefficients as the initial parameter guesses,
the summed mean squared error between the data and
Eq.~\ref{eq:shortquad} was minimized using the Nelder-Mead downhill
simplex minimization (see \cite{Simplex}) provided by
SciPy~\cite{SciPy}. The additional factor $k_0$ was included in the fit,
but is not needed when defining a fringe field within COSY. The process
was repeated for each quadrupole separately. The updated Enge
coefficients and their comparison to the default values used by COSY for
$Q_{10}$ can be seen in Table~\ref{tab:enge}, and the difference in the
shape of the fringe field can be seen in Fig.~\ref{fig:enge_comparison}.

\begin{table}[h]
    \begin{center}
        \caption{ENGE COEFFICIENTS FOR $Q_{10}$ COMPARED TO COSY DEFAULTS}
        \begin{tabular}{c S[table-format=2.8]S[table-format=2.6]}
            \toprule
            \midrule
            \textbf{Coefficient} & \textbf{$Q_{10}$ Values} &
                \textbf{COSY Defaults} \\
            \midrule
            $k_0$ &  0.99731489 & \\
            $a_0$ &  0.37255261 &  0.296471 \\
            $a_1$ &  6.18699778 &  4.533219 \\
            $a_2$ & -5.55514115 & -2.270982 \\
            $a_3$ &  6.96210851 &  1.068627 \\
            $a_4$ & -4.82581328 & -0.036391 \\
            $a_5$ &  1.3135787 &  0.022261 \\
            \bottomrule
        \end{tabular}
        \label{tab:enge}
    \end{center}
\end{table}

\begin{figure}[h]
    \begin{center}
        \centerline{
            \includegraphics[width=0.85\textwidth]{figures/enge_comparison.png}}
        \caption[Comparison between fringe fields]{Comparison between
            fringe fields for the example quadrupole $Q_{10}$. The COSY
            default parameterization for the fringe field is the dashed
            orange line, and the fitted fringe field is the solid blue
            line. Due to the different parameters used for the fringe
            field, the shape is qualitatively different, which will
            affect the actual setting of the magnetic element. Refitting
            the magnetic field in St.\ George within the COSY program
            can be used to guide this procedure.}
        \label{fig:enge_comparison}
    \end{center}
\end{figure}

% \newpage
In some cases, the field maps were not recorded far enough away from the
center of the magnet for the fitting routine to converge, primarily due
to the field not adequately reaching zero. In those cases, ``dummy''
points of zero field were pre\---{} and post\---{}pended to the
individual ``traces'' at distances greater than 5~m from the center of
the magnet to aide in convergence.

The default COSY coefficients for the fringe field were compared against
the data and shown to not adequately describe the field maps. The summed
mean squared error when using the short quadrupole formalization and the
default COSY parameters was significantly larger than that found through
the minimization routine, and the difference was shown to be
statistically significant. A visual comparison between the two models
for $Q_{10}$ is shown in Fig.~\ref{fig:enge_comparison}.

These new terms describing the effective field length and the fringe
field can be used to provide a more realistic beam optics solution for
St.\ George. Within the COSY description of the separator, the magnetic
fields for the quadrupoles were defined with the updated fringe field
parameters. The beam optics solution was refit by minimizing a cost
function defined in terms of our desired transfer matrix properties at
our focal planes, such as the mass and energy dispersion at the
post-Wien filter focal plane $F_2$, the beam spot size at the final
detector plane $F_3$, and other properties. From this recalculated
solution, new pole tip fields for the focusing quadrupoles are
determined which can provide a more accurate estimate of the actual
experimental field required for the experiment.


\section{Energy and Angular Acceptance}
\label{sec:commissioning}

The energy and angular acceptances of St.\ George were determined
experimentally through a series of experimental campaigns using multiple
rigidities. The energy acceptance without a corresponding angular
acceptance was shown to exceed the designed acceptance at zero degrees,
with a measured energy acceptance of $\Delta E/E = \pm 8$\,\% for ten
different beam rigidities covering the phase space region for recoils
created by reactions of astrophysical importance~\cite{Meisel2017}. The
angular acceptance has been shown to meet the desired
$\Delta\theta = \pm 40$~mrad in limited cases with an energy spread of
$\Delta E/E = \pm 3$\,\%. The full total acceptance where the angle and
energy are allowed to deviate up to their maximum design limits has not
yet been measured for the entirety of St.\ George, with work ongoing.

Within the following discussion, the term ``test beam'' will be used in
reference to an incident beam produced by the 5U with a desired
rigidity. These test beams are defined by the selected beam particle,
energy, and charge state, which determines the magnetic and electric
rigidity. Test beams with different rigidities were chosen to adequately
cover the possible rigidity phase space during the commissioning
experiments (see Fig.~\ref{fig:rigidity-phase-space}). These beams were
chosen to provide particles with the desired rigidity and with beam
currents in the range of $0.5 - 3$~$\mu$A in order for the diagnostic
equipment to properly measure the beam. Additionally, those beams that
commonly had highly stable 5U and ion source running conditions over
extended times were selected to reduce beam preparation steps.

Acceptance measurements first probed the energy acceptance within the
designed $B\rho-E\rho$ phase space. The rigidity phase space limits,
along with measured acceptances and ranges for proposed future
experiments is shown in Figure~\ref{fig:rigidity-phase-space}. The
regions of astrophysical interest are accessible by various test beams
that can be produced by the 5U, allowing the phase space to be
adequately studied.

\begin{figure}[h]
    \begin{center}
        \centerline{\includegraphics[width=0.8\textwidth]%
            {figures/rigidity_phase_space.png}}
        \caption[Designed $B\rho-E\rho$ rigidity phase space for St.\
            George]{Designed $B\rho-E\rho$ rigidity phase space for St.\
            George. Stars represent rigidities that have been shown to
            have the full $\Delta E/E = 8$\,\% energy acceptance.
            Reactions shown are probable first experiments using St.\
            George that use beam energies accessible with the 5U:
            \nuc{14}{N} at $E_{\rm{beam}}\approx 0.7-5.0$~MeV (solid
            blue line), \nuc{3}{He} at $E_{\rm{beam}}\approx
            0.25-1.2$~MeV (dashed orange line), and \nuc{12}{C} at
            $E_{\rm{beam}}\approx 3.0-10.0$~MeV (dotted green line).
            These energy ranges cover some of the astrophysically
            important ranges for the given reactions. Adapted
            from~\cite{Meisel2017}.}
        \label{fig:rigidity-phase-space}
    \end{center}
\end{figure}


\subsection{Beam Tuning and Properties}
\label{sec:tuning}

The commissioning runs followed a similar procedure for beam preparation
using the 5U and the transport line. The beam rigidities were chosen to
cover a region within the phase space limits of the separator that cover
recoils produced through reactions of astrophysical interest. Both light
(\nuc{1}{H} and \nuc{4}{He}) and heavier (\nuc{16}{O} and \nuc{20}{Ne})
beams were used to probe different regions of that phase space. Angular
acceptance runs to date have only used lighter beams. The energy
uncertainty of the beam is approximately 0.3~keV, and a conservative
value of 0.5~keV will be used to account for the lack of a direct
measurement with the exact experimental setup used herein.

Beam preparation can be divided into two segments: preparing the
incident test beam with no energy difference and no angular difference
($\Delta E = 0$\,\% and $\Delta\theta = 0$~mrad) to enter into St.\
George along the central magnetic optical axis, and transporting that
beam along the central magnetic optical axis within St.\ George. The
following procedures were used for all acceptance measurements, with
differences being minor. The diagnostic equipment described in
Section~\ref{sec:diagnostic} was essential to performing the beam
preparation steps and their use is highlighted below.

\subsubsection{Before St.\ George}

Test beam preparation before St.\ George is required to prepare the beam
for the experimental requirements of the commissioning runs: the beam
\begin{enumerate}[leftmargin=2\parindent]
% \begin{enumerate}[\indent(1)]  % I'm not sure which will work
    \item must enter along the magnetic optical axis;
    \item must have a narrow waist point with a circular cross section
        at the target location;
    \item must have a focus that is not highly divergent; and
    \item must be stable with low energy uncertainty and relatively high
        current.
\end{enumerate}
These requirements are fulfilled through the operation of both the 5U
and the transport beam line, with the last two requirements primarily
beneficial from an experimental standpoint. The beam divergence is the
maximum angle of the particle trajectory caused by the focusing elements
along the transport line, and is related to the magnetic fields used to
focus the beam and the properties of the beam, such as beam extent and
rigidity. A graphical description of these dependencies is shown in
Fig.~\ref{fig:divergence}.

\begin{figure}[h]
    \begin{center}
        \centerline{\includegraphics[width=0.8\textwidth]%
            {figures/quad_focus_divergence.png}}
        \caption[Sketch of beam divergence due to focusing
            strength]{Sketch of the beam divergence caused by the
            focusing strength of a quadrupole magnet. Increasing
            darkness corresponds to increasing strength of the magnetic
            field. When focusing in a single direction, the apparent
            beam spot (shown on the right) must diverge in the other
            direction. The requirements of the tune will determine what
            beam shape some distance from the quadrupole is required.}
        \label{fig:divergence}
    \end{center}
\end{figure}

The chosen beam intensity is dependent on which diagnostic equipment
will be used. The isolated Faraday cups cannot read current below
50~$e$nA when read through the logarithmic amplifier at the console, and
the current can't be above 20-30~$e\mu$A as the cups are not currently
water cooled and a high intensity and focused beam may melt some of the
components. This upper current limit was not approached during the
tests, since the cups were used in tandem with the quartz viewers. The
quartz viewers are limited to beam currents of a maximum of 3-5~$e\mu$A,
as higher currents risk heating up the quartz to a high enough
temperature to cause them to shatter or melt. Since four of the quartzes
are also barriers between the high ($10^{-8}$~torr) vacuum within St.\
George and atmosphere, this limit must be carefully avoided. In
practice, currents between 500~nA and 4~$\mu$A were used, based on the
exact properties of the ion source for that particular run, the beam
species, and the locations of slits on the primary transport line used
to reduce the beam current.

The procedure for aligning the test beam to the magnetic optical axis is
described below. Major subsections of the procedure will begin with a
short title in bold to guide the reader. The elements on the main
transport line that may be necessary to adjust are the switching magnet
with the $X_6$ steerer, and the $Y_5$ and $Y_6$ steerers (locations
shown in Figure~\ref{fig:5U}). The steerers are labeled as such based on
their position along the main transport line. Steerers $X_6$ and $Y_6$
are part of the same physical steerer but can be operated independently.
Additionally, the quadrupole triplet directly before the target location
will be necessary for final tuning.

\textbf{Aligning the beam to St.\ George's optical axis:}
The desired test beam is transported down the St.\ George transport line
and monitored with the Faraday cup at the target location, called the
\emph{target cup} for beam current stability. Diagnostic equipment
before the target location are used as an aide to transport the beam and
ensure that it has the desired properties. If necessary, the beam
current is reduced. The quadrupole triplet is not used at this point,
since the beam may not be entering the element along its magnetic
optical axis.

The beam is sent into St.\ George. With no field in $Q_1$, $Q_2$, and
$B_1$, the beam hits the quartz viewer at the 0\degree{} exit port of
the magnetic vacuum chamber, called the \emph{$B_1$ quartz}. If the beam
does not strike the quartz, then the final set of steering elements
needs to be adjusted to send the beam into the quartz.

\textbf{Checking for steering:}
Quadrupoles $Q_1$ and $Q_2$ are adjusted independently of each other,
and the resulting motion of the beam on the quartz is recorded. If the
beam is aligned with the central magnetic optical axis, the quadrupole
will only focus the beam and not shift its position on the quartz, i.e.\
the spread in the beam will change but not its central position. The two
quadrupoles must be adjusted independently of each other as any induced
steering from a beam misalignment in one may be counteracted by a
misalignment in the other quadrupole. If the beam is steered by either
quadrupole, the steering elements preceding the offending quadrupole
are adjusted to reduce that steering. Commonly, the elements that steer
in the same direction as the focusing direction of the quadrupole (i.e.\
$Y_5$ and $Y_6$ for $Q_1$, the switching magnet and $X_6$ for $Q_2$)
will provide the largest improvement to the steering. Minor corrections
to all preceding steering elements may be required as the non-steering
solution is approached.

A quadrupole steers a beam when the beam enters the magnetic element
misaligned with the optical magnetic axis. Assuming that the element is
brought from zero to defined strength, the focal length of the
quadrupole changes from $\infty$ to a length $f$. The effect on the beam
is that those regions of the beam away from the optical axis are brought
to pass through this focal point. When a beam is aligned with the
magnetic optical axis, there is an equal amount of the beam on either
side of this optical axis, so the beam spot will narrow along the
focusing axis of the quadrupole. The beam will also extend along the
other axis. If the beam is not aligned with the optical axis, this beam
motion to the focal point will be viewed as a lateral motion along the
focusing axis of the quadrupole. A sketch of this effect can be seen in
Figure~\ref{fig:steering}.

\begin{figure}[h]
    \begin{center}
        \centerline{\includegraphics[width=0.8\textwidth]%
            {figures/quad_steering.png}}
        \caption[Sketch of quadrupole steering of misaligned
            beam]{Sketch of the steering caused by a quadrupole magnet
            due to the beam being misaligned to the magnetic optical
            axis. Increasing darkness corresponds to increasing magnetic
            field strength. From the misalignment, the beam spot appears
            to move along the focusing axis at some distance. The
            scanning motion of the spot can be used as a diagnostic
            tool to determine the beam alignment. If the beam spot is
            not moving in the focus direction as the magnetic strength
            is changed, the beam is aligned to that quadrupole's
            magnetic axis.}
        \label{fig:steering}
    \end{center}
\end{figure}

The goal for adjusting the steering elements before St.\ George is to
have each quadrupole induce no steering on the beam. In practice, each
change to the steering elements either increases or decreases the amount
of steering in the direction of that element. The crossover point, where
the beam switches from steering left to steering right for example, can
be used to restrict the possible phase space of steerer values, as the
beam must have a zero deflection position between those two extremes.

A single quadrupole may induce steering in both directions based on the
beam conditions. For example, a beam that is misaligned in both the $+x$
and $+y$ direction entering into a quadrupole focusing in the $y$ plane
will be steered in both the $+x$ and $-y$ direction. When minimizing
steering in a single direction, the other direction must be periodically
checked to ensure that a minimal steering solution is reached for both
directions at the same time. At the end of this process, the beam is not
deflected when the field strength for either $Q_1$ or $Q_2$ is increased
or decreased independently of the other quadrupole. The beam may be said
to be entering St.\ George along the optical magnetic axis. Due to the
short distance between the first quadrupole doublet and the $B_1$
quartz, it may be necessary to increase the sensitivity of the steering
to ensure that we are as aligned as possible to the axis.

\textbf{Increase sensitivity:}
The beam is then sent further into St.\ George, first to the $B_2$
quartz located within the beamline then the $B_3$ quartz. The steering
of $Q_1Q_2$ is again checked in the same fashion as before. As these
quartzes are located further from the quadrupoles, they give a higher
sensitivity to steering effects from misalignment than just using the
$B_1$ quartz at the trade-off that the the quadrupoles can only be set
to lower field strengths. Since the quartz is further from the focusing
elements, the same focusing strength will create a larger beam spot on
the quartz viewer. This effect can be seen in Figure~\ref{fig:steering}.

These additional checks require $B_1B_2$ to have field. While these two
dipoles must have an exact field strength when performing acceptance
measurements or an experiment, at this point their fields only need to
be coarsely set such that the beam strikes the desired quartz. While the
higher order corrections from these magnets do play a role in the
direction and focusing of the beam, that contribution has no effect on
determining beam alignment within the quadrupoles.

The steering elements are adjusted in the same fashion to minimize
steering in $Q_1Q_2$. Since this steering was minimized during the
previous step, these adjustments should be minimal. It may be necessary
to have a weak field in $B_3$ in order to see the beam on the $B_3$
quartz, due to possible machining misalignments of the port that the
quartz is attached to and the residual magnetic field within the dipole.

\textbf{Include the quadrupole triplet:}
As the last focusing element before St.\ George, the quadrupole triplet
(henceforth simply the \emph{triplet}) is the final adjustable element
to determine the beam properties when entering the separator. The
triplet is used to focus the beam to a small spot at the target
location, a requirement for both experiments and acceptance
measurements. As it and $Q_1Q_2$ should lie on the same magnetic optical
axis, its steering must also be checked and minimized if its use is
desired for the present experiment. It was not used in all cases as the
coarse target focus provided by the previous quadrupole doublets on the
main transport line were deemed sufficient.

Before moving the beam off of the $B_1$ quartz, the steering effects of
the triplet must be characterized in the same fashion as $Q_1Q_2$.
During the steering minimization steps, both the triplet and $Q_1Q_2$
must both be minimally steering before moving forward.

Due to minor misalignments between the triplet and $Q_1Q_2$, it is
usually not possible to have all elements non-steering at the same time.
In these cases, the steering of $Q_1Q_2$ should take precedence while
having the triplet minimally steering. While the steering of the beam
prior to the target location is important, experimentally the steering
of the individual elements within the triplet cancel or nearly cancel
each other out when the triplet is minimally steering, reducing that
problem.

At this point, the main transport line has been tuned to prepare a
well-focused and well-aligned beam entering into St.\ George. These
elements are not to be touched during the rest of the tuning process.
The triplet, due to the possibility of it having minor steering effects,
must also have zero field for the remainder of the steering checks, and
will be turned on for the actual measurement.

\subsubsection{Within St.\ George}
\label{sec:tuning_stg}

Once the test beam has been aligned to enter the separator along the
magnetic optical axis, it must also be aligned to the magnetic optical
axes of all of the quadrupoles within the separator. This alignment is
done using only the dipoles $B_{1-6}$ and the WF. Any minor misalignment
in the vertical direction should have been corrected during the previous
steps, but there is the possibility that there will be vertical steering
within the separator, both from that misalignment and effects from the
dipoles and quadrupoles. The procedure for this second alignment is
straightforward, as the only elements used to adjust the steering of the
quadrupoles are the two dipoles immediately prior.

\textbf{Tuning to the Wien Filter:}
With the beam striking the $B_3$ quartz, quadrupoles $Q_{3-5}$ are
checked for steering. The primary focus of these steering checks will be
on $Q_3$ and $Q_5$ which focus in the horizontal plane. The magnetic
fields within $B_1B_2$ are adjusted to make these quadrupoles
non-steering or minimally steering. The field precision is on the order
of 0.1~G, read back by the Hall probes.

Due to potentially small misalignments in the St.\ George quadrupoles in
relation to each other, it is commonly not possible to have $Q_{1-5}$
non-steering simultaneously (see \cite{Meisel2017}). In these cases,
minimal steering can be achieved through $Q_{3-5}$ by adjusting $B_1B_2$
when $Q_1Q_2$ are non-steering. At this point, dipoles $B_1B_2$ are set
to the value corresponding to the magnetic rigidity of the particle and
to maintain the test beam alignment to the optical axis.

Dipoles $B_3B_4$ are brought up to their rough field value to send the
beam through the WF and onto either the WF quartz or the $B_5$ quartz.
The quadrupoles $Q_{6-9}$ are checked for steering, adjusting $B_3B_4$
to minimize the steering. The focus at this point is on getting a
non-steering solution for $Q_6Q_7$, as $Q_8Q_9$ can also be corrected by
the Wien filter. Since there is some residual magnetic field within the
Wien filter, the electric field is brought up to compensate for this
bending to keep the test beam along the optical axis through $Q_8Q_9$.
The field required is calculated by determining the bending radius
caused by the residual magnetic field and creating the equivalent
bending radius in the opposite direction for the particle's $E\rho$.

As the beam envelope has expanded, it will be necessary to bring
$Q_{1-5}$ to their desired values in order to check the steering of the
remaining quadrupoles. Since these quadrupoles have been shown to be
minimally steering, their effect on the beam trajectory through the
remainder of St.\ George should be negligible. It may be necessary to
have a weak field in $B_5$ in order to see the beam on the $B_5$ quartz
for the same reasons as explained previously for $B_3$.

\textbf{Setting the Wien Filter:}
For the incident test beam's rigidity, the properties of the beam, such
as the energy and charge state, are well known or exactly known. Thus,
the electric rigidity $E\rho$ is also well known when tuning for a set
bending radius. For St.\ George, the Wien filter (WF) bending radius is
$\rho_{\rm WF} = 4.348$~m. The electric dipole within the WF is set for
this rigidity and held constant for the remainder of the tuning process.

The magnetic field is set similarly to the other dipoles: to minimize
the steering induced by the next set of quadrupoles ($Q_8Q_9$). Since
the test beam was aligned to the optical magnetic axes of this
quadrupole doublet in the previous step, the WF magnetic dipole must
return the beam to this orientation. The preceding magnetic quadrupoles
$Q_{1-7}$ must also be set to their required values. These values are
found by scaling the calculated tune based on the rigidity of the test
beam; in cases where an optimal solution was found experimentally that
differs from the COSY calculated values, those \emph{in situ} optimized
values are used as the baseline. The quadrupoles are required to be set
since the elements within St.\ George up to and including the Wien filter
work to separate the beam particles by mass, so setting them mimics the
situation during an experiment.

The magnetic field for the WF is read back using a Hall probe located on
the pole face. The field can be set precisely and related to the fields
in the other dipoles. Once the magnetic field is set such that $Q_8Q_9$
do not steer the beam, the full WF is set.

\textbf{Tuning through the detector chamber:}
Dipoles $B_5B_6$ are set to their rough values based on the $B\rho$ of
the test beam, sending the beam through the detector chamber and onto
the last quartz, called the \emph{detector quartz}. As before, due to
the size and shape of the beam envelope, $Q_8Q_9$ must be set to the
required values. The final two quadrupoles $Q_{10}Q_{11}$ are checked
for steering, and $B_5B_6$ are adjusted to minimize that steering.

Since the test beam is traveling through the detector chamber, the
entire detection system must be pulled out of the way of the beam.
Magnetic shields have been placed below the MCP constructs to remove the
effect of the magnetic fringe fields on the beam
deflection~\cite{MoralesDNP}. Once $Q_{10}Q_{11}$ are non-steering, the
test beam is fully aligned to the optical magnetic axis of St.\ George.

\subsubsection{Collimator and Target Position}

The 2~mm diameter collimator at the target location (see
Section~\ref{sec:target}) is used for setting the triplet to the proper
values. A narrow waist beam at the target location is a requirement to
achieve the maximum angular and energy acceptance for St.\ George. With
the collimator in place, the triplet is adjusted such that the beam
transmission, defined as the ratio between the beam currents before and
after the collimator as read by two separate Faraday cups, is maximized
and ideally close to 100\,\%.

Since the target chamber may rotate around its central axis, it is
possible for the location of the collimator to become slightly
misaligned between runs. Additionally, the triplet may induce some minor
steering at the target location, potentially moving the focal point
radially from the optical magnetic axis. The target collimator position
is then not a fixed value but must also be tuned to maximize
transmission. Once the collimator position is found, the target position
is immediately known. Empirically, a small range of possible values for
the rotation and extension of the target ladder have been found to be
optimal, restricting the search space when tuning.

For acceptance measurements, the collimator is used to create an object
point at the target location. Once the beam preparation is complete, it
is retracted from the beamline. In situations where a target foil is
used as a ``degrader'' to provide an angular and energy spread, the
relative distance between the collimator and the foil is used to
properly align the foil to the beam.

\subsubsection{Additional Considerations}
The steering minimization routine can never fully eliminate the beam
steering because of possible misalignments of the magnetic elements and
the properties of the beam. Especially of concern would be the vertical
steering of the $y$-focusing quadrupoles ($Q_{1,\,4,\,7,\,8,\,11}$), as
the alignment of the beam to their axis is primarily affected by the
beam preparation steps before the beam enters St.\ George. These
quadrupoles may steer in the vertical direction despite the best efforts
of the operator. As there are no elements within St.\ George that could
correct for this, the steering effect of these quadrupoles may not be
able to be eliminated.

As the process is repeated, the required field strengths for the
magnetic dipoles within St.\ George will be better known, and additional
global settings for St.\ George will be iteratively updated to provide a
better starting point for minimization. Over time, the magnetic field
strengths will be known with enough accuracy across a range of rigidity
values that a simple scaling to the selected beam energy can provide the
final set points for the magnets without much further tuning.

\subsection{Energy Acceptance}

The energy acceptance of St.\ George at $\Delta\theta = 0$~mrad was
measured to be $\Delta E/E = \pm 8$\,\% for ten different rigidities
(see Fig.~\ref{fig:rigidity-phase-space} and \cite{Meisel2017}). The
measurements took place before angular acceptance target chamber was
built and will be remeasured as part of a total acceptance measurement
campaign. Test beam rigidities were chosen to cover an adequate region
within the designed phase space near the rigidities expected for recoils
of astrophysical interest and based on the restrictions imposed by the
5U and ion source.

For a given set of field settings for a test beam at $\Delta E = 0$ that
provide 100\,\% transmission between the target cup and a Faraday cup
located within the detector chamber at focal plane $F_3$, the separator
is said to accept an energy difference if the test beam is changed to
that different energy and still have 100\,\% transmission between those
two cups. To state that St.\ George has an energy acceptance of $\Delta
E/E = \pm 8$\,\%, a single set of fields for the elements within the
separator transmitted 100\,\% of the test beam between the two cups when
its energy was changed within that energy change.

The procedure for measuring the energy acceptance of a single rigidity
is outlined below. The slits located at the post-WF focal plane $F_2$
were used to define a beam center. As the tune for a given recoil is
supposed to be achromatic at this location, these slits were used as
both a diagnostic on the path of the beam and a check of this
requirement during the measurements. Note that the tuning process for
these measurements did not make use of the in-beam quartz viewers at
$F_1$ and $F_2$ since they had not yet been installed.

\textbf{Initial setup:}
After tuning a beam along the optical magnetic axis as described in
\ref{sec:tuning}, all elements within St.\ George are at a given field.
The dipole elements, including the WF, are not touched. The transmission
between the target cup and the $F_3$ cup is measured. If the
transmission is 100\,\%, the beam energy was changed. If not, then the
quadrupoles were retuned to transmit 100\,\% of the test beam between
the two cups.

Quadrupole retuning was done systematically to prevent over- or
under-focusing the beam at any location within St.\ George. With the
beam on the $F_3$ cup, each quadrupole was adjusted individually to
determine what field is required to transmit 100\,\% of the beam to the
cup. After finding that field, the difference is recorded and the
quadrupole is returned to its original value. This process is repeated
for every quadrupole acting independently. If a single quadrupole could
not achieve 100\,\% transmission on its own, it was not included in the
next step. Assuming $N$ quadrupoles adjusted by $\Delta B_i$ to give
100\,\% transmission, the individual quadrupoles $Q_i$ were changed by
$\Delta B_i / N$. This approach usually resulted in achieving 100\,\%
transmission for the $\Delta E = 0$ case.

The quadrupole adjustment described was used at every step if the tune
was shown to not transmit 100\,\% of the test beam. Previous settings of
the quadrupoles were recorded to map regions of field strengths were
100\,\% transmission was achieved for different energy changes.

\textbf{Changing energy:}
The beam energy was changed to $E = 0.92 \cdot E_0$
($\Delta E/E = -8$\,\%) by changing the accelerator. The transport
beamline was scaled automatically to account for the change in rigidity.
The beam was shown to enter into St.\ George along the optical magnetic
axis by putting the fields within $Q_1$, $Q_2$, and $B_1$ to zero and
checking the steering of the first two quadrupoles. Since this energy
change is minor, in most cases only the switching magnet needed to be
changed. The magnets $Q_1Q_2B_1$ were brought back to their required
values and transmission between the two cups was checked.

If the beam was fully transmitted to the $F_3$ cup, the settings for
St.\ George were said to have an energy acceptance of $\Delta E/E =
-8$\,\%. The beam energy was then changed to $E = 1.08 \cdot E_0$
($\Delta E/E = + 8$\,\%), following the same procedure, and transmission
was checked. If the beam also was fully transmitted, the separator tune
was said to have an energy acceptance of $\Delta E/E = \pm 8$\,\% and
the measurement was complete.

Where 100\,\% transmission was not achieved, the quadrupole scaling
described previously was used. The new tune was recorded, and the beam
energy was returned to $\Delta E = 0$ to check transmission. This
process was continued until all three energy points had 100\,\%
transmission for a single setting of St.\ George. During this cycling,
referring to previous values was used to prevent correcting the tune in
one direction at one energy only to change back to the previous tune at
another energy.

Since the $F_2$ slits were placed around the beam center, achieving
100\,\% transmission was only possible if test beam had a nearly or
completely achromatic focus following the WF, one of the requirements
for normal operation of the separator.

\textbf{Additional measurements:}
For subsequent energy acceptance measurements, instead of using the COSY
predicted values, an energy acceptance tune scaled based on the magnetic
rigidity $B\rho$ of the new test beam was used for the initial
quadrupole settings. If the difference in $B\rho$ was sufficiently
small, the required adjustments to the quadrupole fields were minimal,
speeding up the measurement process. As more individual energy
acceptance measurements were made, the scaling based on $B\rho$ became
more robust to slight differences in beam preparation and species.

Once the ten rigidities within the astrophysically interesting phase
space of the separator were measured, work moved to measuring the
angular acceptance.


\subsection{Angular Acceptance}

As of this writing, the angular acceptance of St.\ George has been
measured to be $\Delta\theta = \pm 40$~mrad in the horizontal and
vertical planes for a single rigidity. The acceptance was shown by
ensuring 100\,\% transmission when deflecting the beam 40~mrad in each
direction, and quadrupole adjustments followed the same procedure as
during the energy acceptance measurements. The measurement was done
without a corresponding energy acceptance, and without the requirement
that the test beam be focused at the focal plane $F_2$ following the WF
and without the beam passing through the slit opening at that location
for all deflection angles. The measurement was then a single ``proof of
concept'' that an angular acceptance could be measured using the new
diagnostic and control equipment installed.

There are multiple solutions to measuring the angular acceptance, both
and without an accompanying energy acceptance measurement. Multiple
procedures, each with their own tradeoffs and outcomes, were attempted.
The final choice of each measurement option will depend on how the
measurement will be integrated into the final acceptance measurements
and commissioning work. For completeness, the procedures for the
options explored are outlined below.

\textbf{Deflector plates only:}
A test beam is tuned to provide a non-steering beam with 100\,\%
transmission between the target and $F_3$ cups. The deflector plates
(see Section~\ref{sec:target}) are rotated so that they deflect the beam
in a single plane. The horizontal plane was commonly chosen first. Since
the entrance aperture for the target cup is larger than 40~mrad, it does
not intercept any of the beam when it is deflected. Angles between 0 and
40~mrad were used and the current on the $F_3$ cup was monitored. The
maximum angle that provided 100\,\% transmission was recorded.

If the maximum angle achieved was not 40~mrad, the quadrupoles were
tuned in the same fashion as for the energy acceptance measurement but
with the deflector plate set to an angle greater than was accepted such
that the beam is still partially captured by the cup. The changes to the
quadrupole fields were recorded, and all quadrupoles that could provide
100\,\% transmission were scaled to new values. The beam was returned to
$\Delta\theta = 0$ to ensure that the new tune still provided 100\,\%
transmission in this case, and the deflection was changed.

A single plane was checked for $\pm 40$~mrad first before switching to
the other plane, and any retuning was done to also transmit 100\,\% of
the beam to the final cup. The deflector was also rotated to check the
other plane, and the quadrupoles retuned to provide 100\,\%. In general,
this procedure did not provide 100\,\% transmission when deflecting a
test beam up to 40~mrad in the four cardinal directions. This procedure
was used for the single full angular acceptance measurement.

Additionally, since the angular and energy acceptance is dependent on
the beam size and shape at focal plane $F_2$, the WF quartz was used to
aide in tuning $Q_{1-7}$ to their proper values. The beam should move
minimally at this location when deflected up to the maximum 40~mrad in
any direction. The beam profile is required to be horizontally narrow
for the highest mass separation, requiring the vertical extent to be
large. Using this intermediate quartz slightly improved the ability to
tune the separator but did not allow for a full angular acceptance
measurement to be performed.

\textbf{Degrader foil:}
The limiting factor in using the deflector plates as the only angular
change is that each direction must be looked at independently. Assuming
the plates are aligned to deflect in the horizontal direction, only one
direction (left or right from the beam's perspective) can be viewed at a
time without some manual adjustment to the deflector plate power supply.
The cyclic problem of correcting the beam trajectory only to remove that
correction becomes harder to avoid. Since the plates can only deflect
along a single plane, the additional unknowns of removing a large
angular acceptance along a difference by making changes on the current
plane also decreased the possibility of success.

At the target location, Al foils of different thicknesses were placed to
degrade the beam, creating a spread in angle and energy at the same
time. Foil thicknesses were matched with beam properties to fall within
the anticipated $\Delta E/E = \pm 8$\,\% and $\Delta\theta = \pm
40$~mrad acceptances of St.\ George. Since the foils also induce an
energy loss for the test beam, the separator dipoles needed to be
properly scaled down to the correct values after the test beam (without
foil in place) was aligned to the magnetic optical axis. The scaling
required accurate and precise measurements of the foil thicknesses.
Thicknesses ranged from $100-250$~$\mu$g/cm$^2$, and \nuc{1}{H} and
\nuc{4}{He} test beams in the energy range of $0.9-2.0$~MeV were used.

Using the WF quartz, the test beam was tuned to have the correct phase
space properties at $F_2$. The degraded test beam is emitted into the
separator within a phase space determined by its interaction with the
foil, allowing the magnets to be tuned without relying on the slow
change between deflection angles and directions and including the minor
energy acceptance measurement. Currently, no full angular acceptance
measurements have been made past $F_2$.

% Dalmore 12 Year - The Exchange

\textbf{Reaction Measurement:}
Additional measurements have been made of the angular acceptance with an
energy acceptance and a nearly achromatic focus at the $F_2$ focal
plane. These measurements were for the altered settings for transporting
$\alpha$ particles from $(\textrm{p},\alpha)$ reactions. The
measurements are a different ``proof of concept'' for the angular
acceptance measurements by verifying a $\Delta\theta = \pm 40$~mrad
acceptance with the deflector plates before using a foil to produce the
full angular spread. In this case (see
Ch.~\ref{ch:experimental-procedure}), the transported particles are the
reaction product $\alpha$ particles, verified using a direct test beam
of $\mnuc{4}{He}^{2+}$. The transported reactions products within the
$\approx 45$~mrad cone limited by the target Faraday cup were
transported to $F_2$ and detected with the Si detector.


\section{Considerations}

Full acceptance measurements require a fine detailed understanding of
the operation of St.\ George. Previous work has provided the initial
understanding on providing a large energy acceptance of at least $\Delta
E/E = \pm 8$\,\% and angular acceptances near $\Delta\theta = \pm
40$~mrad. Combined measurements have been limited to a large energy
acceptance and small angular acceptance or vice versa. Current work is
ongoing on providing an improved understanding of the operation of St.\
George, particularly in setting the quadrupole fields.

A full commissioning of the separator system requires the gas target,
separator, and detection system to be operated in parallel and
well-understood. The current status of each of these discrete systems is
varied. The Hippo gas target has been tested in a prior configuration,
and work has been started to redesign the upper chamber to improve the
possibility for monitoring incident beam current and using a $\gamma$
detector in coincidence with the final detector system. The combined
$E_{\rm TOTAL}}$~vs.~TOF detection system has been shown to work for
surface sources. Silicon detectors are known to be very robust, and the
Si detector and acquisition system has been used for a successful
measurement with St.\ George for $(\rm{p},\alpha)$ measurements. The
separator status has been explored earlier in this chapter. Final
verification of the separator will be measuring the
\react{\mnuc{14}{N}}{\alpha}{\gamma}{\mnuc{18}{F}} cross section in
inverse kinematics at energies where the cross section is well known.

The target chamber used for the commissioning work and the experimental
campaign is different than that which will be used during a fully
featured St.\ George experimental campaign, namely the Hippo supersonic
helium gas jet target. Hippo will be used for $(\alpha,\gamma)$
experiments following the completion of the commissioning work. The
specifics of that gas target are discussed elsewhere (see
\cite{Kontos2012} and \cite{Meisel2016}). Due to the differences between
the commissioning chamber and the design of the gas target, some
specifics of beam tuning and preparation (see Sec.~\ref{sec:tuning})
will inevitably change as experimental work transitions between
commissioning and reaction research work.

Due to the ongoing nature of the commissioning work and the intricate
link between the systems\textemdash{}accelerator, target, separator, and
detector\textemdash{}involved, the primary \textit{in situ} test of the
complete system will be the cross section measurement of the \alpa{}
reaction near two low-energy resonances. This reaction study will
provide an assessment of the angular acceptance of the separator with a
small energy acceptance. This reaction was chosen since the properties
of the reaction are well-known, allowing for the properties of the
separator to be determined with minimal external uncertainty.
