% Kings County 2.5 yr peated bourbon (Brooklyn, NY)

\chapter{ANALYSIS}
\label{ch:analysis}

The angular acceptance of St.\ George can be determined for a given
energy by comparing the expected yield based on the known reaction
properties to the actual counts measured at the detector plane. Here, we
use ``energy'' as the central energy of the recoil distribution, which
is based on the incident beam energy and the measured target properties.
The acceptance of St.\ George is based on the tune of the separator for
a given recoil energy, meaning that each run that occurs at a different
beam energy, and thus a different tune for St.\ George, has an
associated angular acceptance. The goal is to verify the angular
acceptance for particle rigidities within the possible phase space
anticipated for the produced recoils of astrophysical interest (see
Fig.~\ref{fig:rigidity-phase-space}). From the experimental data, the
properties of the resonance can be determined if the angular and energy
acceptances are known, allowing St.\ George to be used for nuclear
astrophysics research.

The experimental work to determine the angular acceptance is one aspect
of commissioning the separator and has been one of the most challenging
aspects of the work. The \alpa{} reaction runs performed provided an
\emph{in situ} test of the angular acceptance by focusing on
well-understood resonances within the cross section. The analysis of the
experimental results was performed using libraries within the Python
scientific stack: NumPy~\cite{NumPy}, SciPy~\cite{SciPy},
Matplotlib~\cite{Matplotlib}, Pandas~\cite{Pandas}, and
PyMC3~\cite{PyMC3}. The routines have been packaged within an analysis
framework, \verb+pyne+, to improve the processing and reproducibility of
experimental results. See Appendix~\ref{ch:pyne} for details.

\section{Target Properties}
\label{sec:target-properties}

% measuring foil thickness should be a sentence or two cite stopping
% power website for SRIM for the SRIM uncertainty (around 3%) H and He
% should be similar there is enough data in the energy range in question
% to justify it

A self-supporting \nuc{27}{Al} target was used for the entirety of the
experiment. The target thickness was measured using an offline detector
station and a mixed \nuc{241}{Am}/\nuc{148}{Gd} alpha-particle source.
The target thickness was measured by observing the energy loss of the
two alpha peaks and using a Monte Carlo method to determine the value
and uncertainty. The measured target thickness is
$63.8^{+2.0}_{-1.7}$~$\mu$g/cm\squared{} (95\,\% confidence interval),
which does not include the uncertainty due to SRIM, and can be used to
determine the energy loss for other particles at any energy.

In our energy range of interest for both protons and $\alpha$ particles,
the uncertainty in the stopping power in the SRIM results is
approximately 3-5\,\%, based on the spread in experimental values from
different data sources~\cite{Paul2001}.
To be conservative, an uncertainty of 5\,\% is
adopted for all calculations. When taking the SRIM uncertainty into
account, our measured thickness is
$63.8^{+6.7}_{-6.6}$~$\mu$g/cm\squared{} (95\,\% confidence interval).
For subsequent calculations, the thickness used was sampled from this
distribution.

% see foil_thickness/target_thickness_20180403.ipynb

The uncertainties present in each step of the procedure laid out above
are automatically propagated forward due to the methods chosen. The
stochastic nature of the process allows the influence of the base
assumptions of the underlying data (e.g. the counts in each bin are
drawn from a Poisson distribution) to be seamlessly brought forward
without the need of cumbersome mathematics that can potentially hide
incorrect assumptions about the values in question, such as that all of
the data is normally distributed.

For most of the following calculations, the number of target nuclei per
square centimeter is used instead of the thickness in
$\mu$g/cm\squared{}. Our target thickness is then $1.42^{+0.15}_{-0.15}
\times 10^{18}$~nuclei/cm\squared{} (95\,\% confidence interval),
including our SRIM uncertainty. This value is useful for calculating the
energy loss of the proton through the target, as that relies on the
number density of the target and the stopping power of the material. The
energy loss of the beam will be discussed in the next section.

% update foil_thickness/target_thickness_201804XX.ipynb to include the
% SRIM uncertainties directly

\section{Beam Properties}
\label{sec:beam-properties}

The incident proton beam was produced by the 5U and delivered to the St.
George target area. The beam energy and resolution were determined
through a series of accelerator and beamline commissioning experiments
performed before this experiment was performed.

% Similar experiments showed that the energy resolution of the beam is
% approximately 300 keV. To be conservative, a value of 500 keV was used
% for all runs.

The beam energy was determined from the calibration of the 5U analyzing
magnet performed during a different experiment. During the experiment,
the magnetic changes were performed slowly such that the magnetic field
did not appreciably drift during the runs. The energy resolution can
also be determined from the calibration runs, where the resolution is
given by the energy width of the leading edge of the resonance scan.
Values of approximately 300 eV were commonly observed, with a
conservative value of 500 eV adopted for this experiment since no direct
energy calibration was performed with our specific experimental setup.
The uncertainty in the analyzing magnet field is accounted for within
this uncertainty and is not considered separately.

% While the uncertainty in the field strength recorded is not completely
% negligible, it is included within the beam energy resolution and is
% not considered separately.

The beam current was relatively stable during the experiment. During the
longer runs, the beam current was measured every 15 minutes in order to
monitor its change during the run. For each run, the current uncertainty
was determined by the measured values for cases where multiple current
measurements were performed, or 5\,\%. For all runs, the final current
uncertainty was between 5 and 12\,\%. Ideally, an offset Si detector at
the target location would be used to monitor the beam current during the
entirety of the run by measuring the current of the scattered beam
particles at a fixed angle. As this setup was not available for the
target chamber, periodic direct measurements of the current using the
Faraday cup at the target location were required to measure the beam
intensity.

\section{Detector Properties}
\label{sec:detector-properties}

A 16-strip Si detector was used to detect the produced alpha particles
during the experiment. A calibration run was performed following the
experiment using the same detector and data acquisition settings as used
during the experiment. A \nuc{241}{Am}/\nuc{148}{Gd} mixed alpha source
was used for calibrating the energy conversion and energy resolution of
each strip separately. All of the strips were similar with approximately
2~keV/bin for the calibration and approximately 2.75\,\% (90~keV) for
the energy resolution. The poor energy resolution of the detector
resulted from the lower-than-optimal bias voltage setting used during
the experiment.

In the calibration runs, only the two highest intensity peaks could be
resolved above the background. The alpha peaks also exhibit long
low-energy tails such that the particles produced in the \alpa{}
reaction are smeared out in energy. For the experimental run, an energy
threshold was set to exclude incident proton counts, where counts
appearing above the threshold are considered to be from alpha particles.
That threshold was set by the following:

\begin{equation}
    E_{\rm proton} + 3 \sigma_{\rm beam} + 3 \sigma_{\rm resolution}
\end{equation}

The detector efficiency was not directly measured and assumed to be
100\,\%. Efficiency measurements performed during the commissioning work
supporting the St.\ George detector system resulted in efficiencies
above 99\,\% for all strips.

\subsection{Simulating the Energy Spectrum}
\label{sec:05-simulating-spectrum}

A simulation of the expected energy spectrum at the target location was
performed using a combination of SRIM Monte Carlo results and Python.
The simulation looked at the known energy loss within the target of the
incident beam, the expected cross section within the energy limits of
the target, the energy loss of the produced alpha particles through the
remainder of the target, and the energy resolution of the detector to
generate an expected energy spectrum. The procedure for this simulation
is as follows:


\textbf{Step 1:}
  An energy deviation $\Delta_{E,{\rm p}}$ drawn from the ${\rm
  Normal}(0,\sigma)$ distribution (where $\sigma$ is the beam energy
  resolution) for 2000 particles. This energy deviation is the
  difference in energy from the central energy $E_{{\rm p},0}$.

\textbf{Step 2:}
  SRIM files for the central energy were generated for fractional depths
  within the target, where the output is the beam energy profile at that
  target depth.

\textbf{Step 3:}
  For each simulated particle, a depth within the target is randomly
  generated based on the underlying reaction cross section covered by
  the energy range of the incident particles. This depth generation will
  favor depths that energetically match portions of the cross section
  with larger values. The underlying cross section was generated using
  the AZURE2 $R$-matrix code~\cite{AZURE2, deBoer2017}.

\textbf{Step 4:}
  For each particle, an energy $E_d$ is randomly generated from the
  energy distribution obtained from the SRIM Monte Carlo run to that
  depth within the target. The final beam energy $E_{{\rm p},f} = E_d +
  \Delta_{E,0}$ is saved for each particle.

\textbf{Step 5:}
  The beam energy is converted to the produced alpha particle energy
  through the kinematic equation
  % for some reason, using mnuc didn't work...?
  \[
      E_{\alpha,0} = Q + E_{\rm p}\left(
          1 - \frac{m_{\rm p}}{m_{\rm p} + m_{\mnuc{27}{Al}}} %m_{{}^{27}\rm{Al}}}
      \right),
  \]
  where $Q$ is the $Q$-value of the reaction (see
  Sec.~\ref{sec:01-capture-reactions}).

\textbf{Step 6:}
  The difference between the produced alpha energy $E_{\alpha,0}$ and
  the alpha energy used to generate the SRIM files is recorded as
  $\Delta_{E,\alpha}$.

\textbf{Step 7:}
  For each particle, an exit energy $E_{\alpha,f}$ is generated based on
  the SRIM Monte Carlo files. The exit energy is drawn from the
  distribution of exit energies for a target thickness equal to the
  remainder of the target thickness from the depth generated in Step 5.

\textbf{Step 8:}
  The energy of the alpha particle entering St.\ George is determined
  from $E_{\alpha} = E_{\alpha,f} + \Delta_{E,\alpha}$, or adding the
  energy deviation back to the generated energy.

This procedure generates an $\alpha$-particle energy spectrum for the
particles entering St.\ George based on the known properties of the
target, cross section, and beam. Using the known energy resolution of
the detector, this spectrum can be converted into the detected spectrum
\[
    E_{\alpha,{\rm detector},i} \sim {\rm Normal}(E_{\alpha,i},\sigma_{\rm detector}),
\]
where $i$ is for each individual particle within the final distribution.
An example of the output of this procedure is given in
Fig.~\ref{fig:sim-spectra}
% ,
% where the agreement between the location and width of the alpha peak
% can be seen in the normalized spectra. Note that the low energy
% tailing of the detected particles is not modeled in our simulated
% spectrum, as we don't know the full characteristics for the detector
% response.

\begin{figure}[h]
    \begin{center}
        \centerline{\includegraphics[width=0.75\textwidth]%
            {figures/simulated_spectrum.png}}
        \caption[Simulated detector
            $\alpha$ energy spectrum]{Simulated detector $\alpha$ energy
            spectrum, following the procedure outlined in
            Sec.~\ref{sec:05-simulating-spectrum}. The blue filled
            distribution is generated by the procedure, and the orange
            line is the counts detected at one of the central strips.
            The two distributions are normalized to unit area. There is
            a qualitative match between the two descriptions, with the
            simulated spectrum showing fatter tails but a symmetric
            distribution. The rough re-creation of the distribution from
            an ``uncertainty first'' perspective is indicative of the
            potential future use of the procedure.}
        \label{fig:sim-spectra}
    \end{center}
\end{figure}


\section{Additional Parameters}
\label{sec:05-additional-parameters}

Additional inputs into the final calculation of the acceptance of St.
George are the cross section determined from an R-matrix fit on several
low-lying resonances, the stopping power of protons in aluminum from
SRIM, the run time, and the counts at the detector. For those parameters
that are derived from external programs (AZURE2 and SRIM), the
uncertainty is assumed to be negligible. The uncertainty in the time was
assumed to be 10 seconds for those runs that only lasted for a single
15-minute span, and higher for those runs that required the periodic
measurement of the beam current which resulted in stopping the incident
beam for an unspecified duration of time.

The counts at each detector were the sum of all events above the
threshold defined by the beam energy and detector resolution. The counts
are Poisson distributed, with the length of time for the run was such
that the uncertainty from the counts at the detector was not above
5\,\%, with most runs having a count uncertainty of a much lower value.
The direct uncertainty of the counts at the detector is partially
convolved with the run time; a lower counting uncertainty requires a
longer run time and potentially a larger time uncertainty.

The direct beam reduction by St.\ George must be on the order of
$10^{10}-10^{14}$ in order to avoid damaging the Si detector and to
measure lower value regions of the cross section. This requirement is
within the designed capabilities of St.\ George when tuned for heavy
recoil transmission to the final detector plane, but must be verified
experimentally due to the altered tune and different detector plane
required for this experiment. During the experiment, count rates at the
detector were monitored, and potential counts from the direct proton
beam were excluded from the final counts with the energy discriminator
previously described.

\section{Final Acceptance Measurements}
\label{sec:final-acceptance-measurements}

The acceptance of St.\ George can be found for each energy value by
comparing the detected counts to the expected yield for that incident
beam energy. The yield is found from:
\begin{equation}
    Y(E) = \frac{N_r}{N_b},
\end{equation}
where $N_r$ is the number of reaction products produced and $N_b$ is the
number of incident beam particles. We can determine $N_b$ from the beam
current and the total run time. The value for $N_r$ is determined by the
total counts at the detector (for the experimental yield) or the
integration of target and cross section properties following
\begin{equation}
    \label{eq:yield}
    Y(E_0) = \int^{E_0}_{E_0 - \Delta E_t} \frac{\sigma(E)}{\epsilon(E)}\,\rm{d}E,
\end{equation}
where $E_0$ is the incident beam energy and $\Delta E_t$ is the target
thickness in energy for the incident beam. In both cases, the detector
efficiency and St.\ George transport efficiency are 100\,\%, as
previously discussed.

\textbf{SHOW THE DERIVATION FOR BELOW}

If we assume that the angular acceptance is symmetric, the angular
acceptance in mrad is given by
\begin{equation}
    \label{eq:acceptance}
    \theta = \arccos\left(1 - 2 \frac{Y_{\rm experiment}}
        {Y_{\rm theory}}\right).
\end{equation}
The angular acceptance can be calculated in this way for each run
individually. The process for calculating the uncertainty bounds is
given by the following, repeated 2000 times to have enough confidence in
the final values:

\textbf{Step 1:}
  The beam energy, beam current, and time are sampled from a normal
  distribution ${\rm Normal}(\mu,\sigma)$, where $\mu$ and $\sigma$ are
  for the value (energy, current, or time) in question.

\textbf{Step 2:}
  The incident number of particles is calculated from the current and
  time.

\textbf{Step 3:}
  The target thickness in energy is calculated from finding the stopping
  power at the incident beam energy from the SRIM tables, and sampling
  from the distribution of target thicknesses in terms of
  atoms/cm\squared{}.

\textbf{Step 4:}
  The yield is determined by integrating Eq.~\ref{eq:yield} between the
  entrance energy and the lower energy given by that entrance energy
  minus the target thickness.

\textbf{Step 5:}
  The experimental yield is drawn from a poisson distribution
  ${\rm Poisson}(c)$, where $c$ is the number of counts detected.

\textbf{Step 6:}
  The acceptance for the iteration is calculated by
  Eq.~\ref{eq:acceptance}.

The distribution of values generated by the process above can be used to
find the acceptance and confidence intervals for the run in question.
Each run has an acceptance described by its distribution, which is the
run for that particular setting of St.\ George.
