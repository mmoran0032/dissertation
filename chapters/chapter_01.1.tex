\chapter{INTRODUCTION}

The elements making up the universe were formed during a variety of processes,
beginning with Big Bang Nucleosynthesis (BBN) that formed the lightest
elements. Those elements common to life on Earth were primarily formed through
burning processes inside of stars, grouped together under the title of
Stellar Nucleosynthesis. Depending on the conditions within the stellar
environment, which are characterized by macroscopic qualities about the star
(temperature, pressure, mass, etc.) and the elemental composition of the
stellar interior where the burning process takes place, the reactions
accessible to the nuclei within the star differ. The creation and destruction
of different elements and isotopes may be inhibited or enhanced by these
differing conditions, and the study of these processes at the nuclear level
has spawned the field of nuclear astrophysics in order to understand the inner
workings of these stars.

The study of these reactions has increasingly taken place within nuclear
accelerator laboratories, which can attempt to replicate the conditions inside
of stars in order to measure the qualities of the reactions as they take place
in the cosmos. Individual reactions may be isolated by the choice of the
accelerated beam particle and its properties, and the choice of target material
and its properties. As the underlying cross section governing the observed
properties of the reaction may change rapidly within a narrow band of energy,
immense amounts of work and effort has been expended in order to detect these
changes with enough precision in order to be confident that the cross section
has been accurately described. Whole classes of reactions and detection methods
have been devised to study those reactions previous out-of-reach to experiment,
either due to the energies in question or the complexities of detecting the
desired particles.

In order to properly measure a single cross section, a [COMPLETE?] knowledge of
the relevant physical processes and experimental methods is paramount. The
astrophysical motivation for studying certain reactions and the experimental
techniques devised to study these reactions will be explored.

\section{Nuclear Astrophysics}
\label{sec:ch01-nuclear-astrophysics}

The specific and directed study of those nuclear reactions that have an effect
on the properties or life cycle of celestial bodies can be grouped under the
umbrella term \emph{nuclear astrophysics}. These reactions may take place
during the lifecycle of a star, during a cataclysmic event within the universe
such as a black hole merger or gamma ray burst, or at the beginning of the
universe itself. Additionally, the decay of various isotopes can also play a
major role within this domain, either as part of a sequence of reactions or
independently. The entire field of nuclear astrophysics was conceived in the
seminal papers [B2FH] and [Other], which have been used as a basis for much of
the work following.

For astrophysical reactions, the properties of the environment can play a major
role in how quickly the reaction proceeds or if it is even energetically
allowed. Cross sections for many of these reactions rapidly decrease at lower
energies, ...

Cross sections, resonances, how temperature and density affect reactions, etc

\subsection{Hydrogen Burning}

The processes by which \nuc{1}{H} can be combined to form \nuc{4}{He} are
called \emph{hydrogen burning processes}, or simply \emph{hydrogen burning}.

\subsubsection{$pp$ Chains}

Stars similar to our Sun fuse hydrogen through the $pp$ chains, which are
described by the reaction sequences
\begin{align*}
    \rm{PP-1}& \mreact{\rm{p}}{\rm}{e^+\nu_e}{\mnuc{2}{H}} \
    \rm{PP-2}& \
    \rm{PP-3}&
\end{align*}
The net effect of these reaction sequences is the reaction
\[
    4\rm{p} \rightarrow \mnuc{4}{He} + 2e^+ + 2\nu_e,
\]
with the primary difference being what intermediate nuclei the reaction
progresses through. Each of the identified $pp$-chain sequences releases the
same amount of energy (XXX MeV) and become energetically favorable in different
temperature ranges inside of stars.

\subsubsection{CNO Cycles}

The Carbon-Nitrogen-Oxygen (CNO) cycle is an additional pathway for stable
hydrogen burning making use of catalytic nuclei to produce \nuc{4}{He} through
the reaction sequences
\begin{align*}
    \rm{CNO-1}& \mreact{}{}{}{} \
    \rm{CNO-2}&
\end{align*}
The CNO cycles dominate the energy production of heavy stars with properties
[MASS and TEMPERATURE].

\subsection{Helium Burning}
?
\subsection{Neutron Capture Processes}

s-process Nucleosynthesis
- end state of CNO is more 14N than anything else
- Breakout reaction feeds into s-process by generating seed nuclei
- s-process progresses in AGB stars over long timescales
- produces high mass nuclei near the valley of stability

- where does 27Al(p,a) fit in?
- hydrostatic hydrogen burning in AGB stars
- explosive hydrogen burning in a nova
- np-process, sp-process, r-process, etc?

?

%%% end section %%%

\section{Beam Optics}
\label{sec:ch01-beam-optics}

The understanding of how recoil separators work is grounded in the theory of
beam optics which describe the effect of electric and magnetic fields on moving
charged particles. These moving particles are focused and directed by the
electromagnetic elements, and their action on particles can be modeled and
optimized to transport particles with various properties down a beam line to a
desired location.

- parts of beam optics?
- Louisville theorom (conservation of phase space)
- connection to properties (resolution?)
- references to code/COSY?

\subsection{Considerations}
\subsection{Resolution}

%%% end section %%%

\section{Recoil Separation}
\label{sec:ch01-recoil-separation}

The experimental process in which the reaction products produced by a direct
beam can be filtered out from that direct beam in order to be detected is
called \emph{recoil separation} or alternatively \emph{recoil mass separation}.
A recoil separator is the system, consisting of a sequence of electromagnetic
elements, designed to perform this task.

\subsection{Motivation}

Recoil mass separation was conceived as an alternate way to measure the cross
sections of radiative capture reactions. These reactions had previously been
studied by detecting the produced $\gamma$ rays, subject to the limitations
previously discussed. The heavy reaction product can instead be detected by a
detector situated behind the target, assuming that the target is thin enough to
allow the produced recoils to leave the target. In this thin target case, the
incident beam will likely pass through the target as well, making it a source
of background at the detector plane. In the cases of interest for nuclear
astrophysics, this background count rate could be $\times 10^{15}$ that of the
particles of interest and may cause damage to the detector.

The produced recoils may be filtered out from the incident beam by
electromagnetic elements situated between the target and the detector. The
interaction between the heavy incident beam with mass $A$ and linear momentum
$p$ and the $\alpha$ particles within
the target produces a heavy compound nucleus with mass $A + 4$ and momentum
$p$. Ignoring the effect of the emitted $\gamma$ ray on the momentum and
assuming that there is no spread in the momentum, the use of electrostatic
elements can separate the recoils from the beam based on their different
magnetic and electric rigidities, defined as
\[
    B\rho\,\rm{[Tm]} = \frac{p}{q} = \frac{\sqrt{2mT}}{q}
\]
and
\[
    E\rho\,\rm{[MV]} = \frac{pv}{q} = \frac{2T}{q},
\]
respectively, where $q$ is the charge, $T$ is the kinetic energy, $m$ is the
mass, and $v$ is the velocity of the particle. With a single momentum and
velocity (or kinetic energy) selected for, the recoil particles of interest can
be uniquely identified by the optical system. The design of recoil separators
make use of this relatively simple idea as the basis of their design. Despite
this, there have been relatively few recoil separators that have been brought
into service due to the complexities of their design and operation that are not
adequately taken into account in this description.

\subsubsection{Radiative Capture}
\subsection{Experimental Considerations}
\subsubsection{Inverse Kinematics}
\subsection{Recoil Separator Facilities}
The use of recoil separators to study radiative capture reactions has been
explored recently at a number of facilities. The design of St. George is based
on the knowledge gained from the design, construction, and operation of these
previous recoil separator systems. The entire system, inclusive of the beam
source, target, and detector, must be discussed as a whole when evaluating the
capabilities of a given separator.

\subsubsection{CalTech Separator}
% The design and use of recoil separators for astrophysical studies was first
% pursued by Smith \textit{et al.}~\cite{Smith1991}. The requirements of radioactive
% beam studies required the need to develop new detection systems, especially
% considering the effect of differing beam property limits (intensity, purity,
% and emittance) and the desire to detect the produced nuclei to further probe
% the astrophysical conditions. As the initial feasibility system to provide a
% technical proof-of-concept, many of the design choices made and techniques
% used have been adopted by following separators. These include the use of a Wien
% filter for velocity selection, dipole magnets for momentum selection, an
% electrostatic deflector for energy selection, and the use of a gaseous target.
% Additionally, the use of a gamma-ray detector in coincidence with the final
% recoil detection and beam monitoring with an offset Si detector at the target
% location are also common choices that have been adapted at the other separators.

The design and use of recoil separators for nuclear astrophysics research was
pioneered by Smith *et al.*. This separator was a proof-of-concept design to
determine the feasibility of performing reaction studies with this technique.

\subsubsection{DRAGON}
The DRAGON recoil separator at TRIUMF-ISAC was built for the same reasons as ARES,
but differs in the actual construction and usage of the separator.
% Success of separator
The separator
itself uses two large magnetic dipoles for momentum separation and to electric
dipoles for energy selection~\cite{Engel2005}. The separator also contains
steering elements within the beamline to aide in transporting the recoils to
the detector plane. The extended gas target is surrounded by a large BGO
gamma-ray detector for coincidence purposes

\subsubsection{ARES}
The Astrophysics REcoil Separator (ARES) was built at Louvain-la-Neuve to
study $(\rm{p},\gamma)$ and $(\alpha,\gamma)$ reactions using radioactive
incident beams provided by the CYCLONE44 cyclotron~\cite{Angulo2001}.
Self-supporting solid
targets, containing the required H or He, were used for the reaction
studies. The system is designed with a single magnetic dipole for momentum
selection and a Wien filter for velocity selection, along with multiple
magnetic quadrupoles to maintain the transportation of the to the detector
system. The condensed and limited size of the separator is based on the
constraints of the experimental hall~\cite{Couder2003}.
The detector system consists of a single
$\Delta E − E$ telescope which separates out the reaction products from the
remaining incident beam particles. The initial test of the separator used a
stable incident beam to compare to results obtained by other methods within
the lab, and the focus of the initial work was on low-lying resonances of
astrophysical interest.

\subsubsection{ERNA}
% Look up ERNA design/commission paper(s)
The ERNA (acronym) recoil separator at (location)...

\subsubsection{St.\ George}
The separator consists of six dipole magnets, eleven quadrupole magnets, and a
Wien filter. The separator was designed to accept recoils with a maximum
energy and angular spread of $\Delta E/E = \pm7.5\%$ and
$\Delta\theta = \pm40$~mrad, respectively, and to provide a mass separation
of $m/\Delta m = 100$ and beam suppression of a factor $\geq 10^{15}$. Combined
with the HIPPO (High-Pressure Point-like target) supersonic gas jet target,
St.\ George will be primarily used to study low energy $(\alpha,\gamma)$
reactions using stable beams.
