\chapter{INTRODUCTION}

\section{Nuclear Astrophysics}

The specific and directed study of those nuclear reactions that have an effect
on the properties or life cycle of celestial bodies can be grouped under the
umbrella term \emph{nuclear astrophysics}. These reactions may take place
during the lifecycle of a star, during a cataclysmic event within the universe
such as a black hole merger or gamma ray burst, or at the beginning of the
universe itself. Additionally, the decay of various isotopes can also play a
major role within this domain, either as part of a sequence of reactions or
independently.

\subsection{Hydrogen Burning}

The processes by which \nuc{1}{H} can be combined to form \nuc{4}{He} are
called \emph{hydrogen burning processes}, or simply \emph{hydrogen burning}.

\subsubsection{$pp$ Chains}
\subsubsection{CNO Cycles}
\subsection{Helium Burning}
?
\subsection{Neutron Capture Processes}
?
\subsection{Quantities of Interest}
Cross section, reaction rate, $R$-matrix formalism?

%%% end section %%%

\section{Beam Optics}
?

%%% end section %%%

\section{Recoil Separation}

The experimental process in which the reaction products produced by a direct
beam can be filtered out from that direct beam in order to be detected is
called \emph{recoil separation} or alternatively \emph{recoil mass separation}.
A recoil separator is the system, consisting of a sequence of electromagnetic
elements, designed to perform this task.

\subsection{Motivation}
\subsection{Experimental Considerations}
\subsection{Recoil Separator Facilities}
\subsubsection{CalTech Separator}
\subsubsection{DRAGON}
\subsubsection{ARES}
\subsubsection{ERNA}
\subsubsection{St.\ George}
