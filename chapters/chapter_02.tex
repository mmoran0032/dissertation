\chapter{EXPERIMENTAL SETUP}
\label{ch:02-experimental-setup}

All experimental work was performed at the Nuclear Science Laboratory
(NSL) at the University of Notre Dame, using the St.\ Ana 5U-4
accelerator and St.\ George. Commissioning work for the separator began
in 2014 and is currently on-going (see Section~\ref{sec:commissioning}).
The experiment discussed herein consisted of three separate runs, each
one week long, in December 2016 and February 2017. The first run tested
a proposed close location for the detection system, the second run
characterized the additional magnets required for a far location for the
detection system, and the third run was the primary data collection run.

The accelerator provided a high intensity proton beam to St.\ George for
the first and third runs, and a high intensity $\mnuc{4}{He}^{+}$ beam
for the second characterization run. A 16-strip Si detector was
installed at the two proposed focal planes ($F_1$ and $F_2$, see
Fig.~\ref{fig:stgeorge}) to detect the produced $\alpha$-particles from
the \alpa{} reaction. Incident beam rejection was on the order of
$1.5\times 10^{12}$ for the runs at the resonance peak.


\section{The St.\ Ana Accelerator and Transport Line}
\label{sec:ch02-5U}

St.\ ANA (Stable beam Accelerator for Nuclear Astrophysics) is a 5~MV
vertical, single-ended pelletron accelerator at the NSL, providing
high-intensity stable beams to a number of experimental setups within a
dedicated target room. The original designation of the accelerator by
the manufacturer, National Electrostatics Corporation, was the 5U-4,
denoting the five individual acceleration sections and the four charging
chains, causing the accelerator to commonly be referred to as the
\textit{5U}. Both names will be used interchangeably throughout this
work.

\begin{figure}
    \begin{center}
        \centerline{\includegraphics[width=0.8\textwidth]%
            {figures/schematic_5U_transport.png}}
        \caption[Schematic of the 5U accelerator]{Schematic of the 5U
            accelerator and transport beamline. The steering elements
            \emph{XY\#} and focusing quadrupole doublets \emph{QD\#} are
            used the direct and focus the beam to St.\ George. The
            transport line is used to prepare the beam for experiments
            and is not adjusted after that preparation. Adapted from
            \cite{Meisel2017}.}
        \label{fig:5U}
    \end{center}
\end{figure}

The 5U, shown schematically in Fig.~\ref{fig:5U}, accelerates a high-intensity
ion beam produced by the Nanogan Pantechnik ECR source to the desired
energy by energizing the terminal shell to high voltage. The
acceleration tube, extending from the bottom of the shell to the bottom
of the accelerator, steps that voltage down through a series of
resistors and plates, creating an electric field gradient along the tube
that accelerates the positively charged ion beam down and out of the
accelerator. The beam is then bent around a 90\degree dipole magnet,
called the analyzing magnet, and through a pair of vertical slits,
called the analyzing slits, that provide energy identification of the
beam. In addition, the current on the slits can be used as a feedback
control system to regulate the voltage on the terminal shell when the
accelerator is being run in ``slit control mode,'' providing a highly
stable beam with a small energy spread. A recently performed
\react{\mnuc{27}{Al}}{\textrm{p}}{\gamma}{\mnuc{28}{Si}} resonance
scan~\cite{Gilardy2017} was used to determine the beam energy spread of
$\sigma_{\textrm{beam}} \approx 0.3$~keV near a beam energy of
$E_{\textrm{beam}} = 1320$~keV. The 5U also contains focusing,
directional, and diagnostic elements used to help tune the accelerator
to provide a stable and well-behaved beam.

The transport beamline directs the analyzed beam to the desired
experimental area through use of focusing and directional elements: two
magnetic quadrupole doublets maintain a focused beam along the beamline;
the magnetic dipole switching magnet directs the beam down one of the
available experimental beamlines; and magnetic steerers shift and turn
the beam within the beamline. The steerers act to maintain the beam
along the magnetic optical axis of the quadrupoles and maximize the
amount of beam being transported down the beamline from the accelerator.
Sets of diagnostic equipment are also installed at various locations
along the beamline to help monitor and restrict the beam, and the entire
system is kept at a high ($\approx 8\times 10^{-8}$~torr) vacuum through
use of multiple turbo-molecular pumps located along the beamline.

The final section of the transport line is between the switching magnet
and the entrance to St.\ George. This section of beamline prepares the
beam to have the required characteristics at the target location and
contains a single magnetic steerer (horizontal and vertical) and a
magnetic quadrupole triplet along with diagnostic equipment. The final
magnetic steerer is used to finalize the alignment of the beam, and the
magnetic quadrupole triplet creates a small, well-focused beam at the
target location.


\section{The St.\ George Recoil Separator}
\label{sec:stg}

The St.\ George is a recoil mass separator at the NSL~\cite{Couder2008}
and one of the experimental beamlines accessible to the 5U. The design
and operation is based on previous recoil separators (see
Sec.~\ref{sec:prevwork}). The separator consists of six dipole magnets,
eleven quadrupole magnets, and a Wien filter. The separator was designed
to accept recoils with a maximum energy and angular spread of $\Delta
E/E = \pm7.5\%$ and $\Delta\theta = \pm40$~mrad, respectively, and to
provide a mass separation of $m/\Delta m = 100$ and beam suppression of
a factor $\geq 10^{15}$. At lower angular spreads, the transported ions
may have a larger energy spread limited by the good field region within
the third quadrupole $Q_3$. The design was guided by the desire to
efficiently measure $(\alpha,\gamma)$ reactions with a beam mass up to
$A = 40$ with high beam intensities and to fit the separator within the
physical limitations of the target room at the NSL~\cite{Couder2008}.

The separator is designed to transport recoils within a rigidity phase
space relevant for measuring $(\alpha,\gamma)$ reaction cross sections.
The transportation of these ions is achieved by setting the magnetic and
electrostatic elements within the separator, based on the rigidity of
the ions. The design parameters restrict the magnetic
(Eq.~\ref{eq:brho}) and electric (Eq.~\ref{eq:erho}) rigidities within
the ranges $0.1 \leq B\rho \leq 0.45$~Tm and
$E\rho \leq 5.7$~MV~\cite{Couder2008}

\subsection{Subsections of St.\ George}

\begin{figure}[t]
    \begin{center}
        \centerline{\includegraphics[width=0.95\textwidth]%
            {figures/st_george.png}}
        \caption[Layout of the St.\ George recoil separator]{Layout of
            the St.\ George recoil separator. Quadrupoles are identified
            by \emph{Q}, dipoles by \emph{B}, and focal planes by
            \emph{F}. Section labels are placed near the dipole doublet
            within that section, and boundaries are the intervening
            focal planes. Adapted from Reference~\cite{Couder2008}.}
        \label{fig:stgeorge}
    \end{center}
\end{figure}

The layout of St.\ George is shown in Fig.~\ref{fig:stgeorge}. The
separator is divided into three sections, based on their purpose: the
charge selection stage; the mass selection stage; and the clean-up
stage. The stages are separated by the focal planes. Each of these
stages will be discussed in turn.

The entire separator is tuned for a single $B\rho-E\rho$ rigidity,
defined by the central energy of the recoils and its mass and charge
state (see Eqs.~\ref{eq:brho} and \ref{eq:erho}). For commissioning
purposes, a direct incident beam was used as a ``test beam'' to tune the
separator without having to perform a reaction. The tuned particles will
travel down the central magnetic axis of the separator. Particles that
differ in energy or angle from these central particles will travel
through a different path (see Fig.~\ref{fig:raytrace}).

The first section is the so-called \textit{charge selection} stage,
consisting of the first quadrupole doublet ($Q_1Q_2$) and the first
dipole doublet ($B_1B_2$). The doublet $Q_1Q_2$ focuses the recoils
through the dipole pole gap, and the doublet $B_1B_2$ provides the first
rejection of beam due to the difference in magnetic rigidity from the
desired recoils. After the first focal plane $F_1$, a single recoil
charge state will be transported through the remainder of the separator.
Horizontal slits may be placed at this focal plane to aide in the
rejection of incident beam particles that have undergone a charge
exchange event. Slits were not used during the commissioning work or
during the primary experiment. At the end of this section, both beam and
recoil particles are present within the separator.

The next section is the so-called \textit{mass selection} stage, which
contains the magnetic quadrupole triplet ($Q_3Q_4Q_5$), the second
magnetic dipole doublet ($B_3B_4$), the second magnetic quadrupole
doublet ($Q_6Q_7$), and the Wien Filter (WF). This section's primary
purpose is to reject the incident beam and create an achromatic focus at
the second focal plane $F_2$. This focus is horizontally narrow at the
focal plane to aide in the rejection of the beam. At this focal plane, a
set of horizontal slits, called the mass slits, are placed to reject the
remainder of the incident beam. The focused recoil particles will pass
through the center gap between the slits. The mass resolution of the
separator depends on the particle distributions of the beam and recoil
being focused and spatially separated at this position, and that the
tail of the beam distribution minimally overlaps with the recoil
distribution.

\begin{figure}[t]
    \begin{center}
        \centerline{\includegraphics[width=0.95\textwidth]%
            {figures/wien_filter.png}}
        \caption[Schematic of the St.\ George Wien filter]{Schematic of
            the focusing and bending properties of the Wien filter
            installed in St.\ George. The crossed electric ($E$) and
            magnetic ($B$) fields would deflect a given particle in
            opposite directions, such that a particle passing through
            with the desired velocity would remain unreflected. This
            simplistic description ignores the effect of the particles
            in question existing within an envelope of energies and
            angles. Adapted from \cite{Couder2008}.}
        \label{fig:wienfilter}
    \end{center}
\end{figure}

The Wien filter operates by having crossed electric and magnetic fields,
oriented such that individually they would bend a particle beam in
opposite horizontal directions, as shown in Figure~\ref{fig:wienfilter}.
From the beam's perspective, the electric field bends the beam to the
right while the magnetic field bends the beam to the left. The fields
are provided by a pair of electrostatic plates within the vacuum chamber
and a magnetic dipole outside of the chamber. A Wien filter is set to
allow for a single velocity to pass through the center of the element
undeflected, according to $v = E/B$. For St.\ George, the Wien filter
provides the final mass selection for our recoils, allowing the beam
particles to be filtered out from the recoil particles, since the mass
difference between the two ($\Delta m = 4$ when performing
$(\alpha,\gamma)$ reactions) translates into a velocity difference.

The simplified description above is not entirely accurate, since our
beam and recoils exist within a phase space envelope confined by ranges
of allowed positions, angles, and energies. Since we inherently have an
energy distribution of our recoils deriving from the convolution of the
beam energy uncertainty with the target losses of the beam and recoils
and the energy change arising from the $\gamma$ ray emission, we do not
have a mono-energetic recoil distribution passing through the center of
the WF. When taken as a whole, the elements up to and including the WF
create the proper beam and recoil properties to reject the incident beam
based on their mass difference.

At this point, our beam envelope can be thought of as just our
\textit{recoil} envelope, where the particles still being transported
through St.\ George are just those recoils produced in the reaction at
the target location. In reality, the envelope still contains some beam
particles, either through scattering off of the interior of the beam
pipe, diagnostic elements, or the residual vacuum, or through charge
changing events with the residual vacuum. Due to these factors, we
cannot place the final detection system right after the Wien filter and
instead need additional elements to further reduce the background.

This final section of St.\ George is the so-called \textit{clean-up}
stage, where the phase space of the recoils passing through the mass
slits are matched with the phase space of the detection system,
providing the last rejection of the incident beam particles. This
section consists of, in order, a quadrupole doublet ($Q_8Q_9$), the last
dipole doublet ($B_5B_6$), and a final quadrupole doublet
($Q_{10}Q_{11}$). These magnets transport the recoil particles that
passed through the mass slits at $F_2$ through the detection system
installed at the detector focal plane $F_3$. Since the detection system
has a defined physical acceptance size, these magnets must reduce the
physical extent of the recoil envelope within this space.


\subsection{Diagnostic Equipment}
\label{sec:diagnostic}

\begin{figure}[t]
    \begin{center}
        \centerline{\includegraphics[width=0.95\textwidth]%
            {figures/st_george_diagnostics.png}}
        \caption[Locations of diagnostic equipment]{Locations of various
            diagnostic equipment within St.\ George. The slits (denoted
            \emph{S}) are installed at entrance and exit ports of
            dipoles and at the Wien filter focal plane. The isolated
            Faraday cups (denoted \emph{FC}) are located near focal
            planes and saddle points within the separator. The quartz
            camera viewers (denoted \emph{QV}) are located at the exit
            ports of the dipoles and within the beam line at focal
            planes. Adapted from Reference~\cite{Couder2008}.}
        \label{fig:diagnostic}
    \end{center}
\end{figure}

To aide tuning the beam, additional diagnostic equipment has been
developed and installed at various points along the separator. This
diagnostic equipment can be divided between three basic types: Faraday
cups, slits, and quartz viewers. These first two equipment types are
present within other beamlines and the primary transport beamline, while
the third was adapted for St.\ George based on principles encountered
when working with other beamlines. The positions of this equipment is
denoted in Figure~\ref{fig:diagnostic}.

Faraday cups are beamstops that also provide the user with the beam
current being captured by the cup. The cup consists of three main parts:
the shield, the suppression, and the cup itself. This structure is
attached to a linear motion drive, letting the cup be positioned in or
out of the beam, and isolated from the beamline. The shield primarily
protects the suppression from being hit by incident beam but, due to its
isolation, is also a point to read out the current and determine the
physical extent of the beam at that point. Ideally, all of the beam
would enter the cup portion, allowing the Faraday cup to determine the
complete beam current at that location. Using the cup in this way as a
tuning aide is possible since the cup locations were selected based on
the beam optics properties, as those locations correspond to waist
points of the beam. The suppression is necessary since electrons are
emitted with energies in the rough range of $20-100$~eV when the beam
strikes the physical cup. By biasing the suppressor to $-300$~eV, those
electrons are directed back toward the cup, giving an accurate reading
of the beam current actually hitting the cup.

Slits are an additional way to determine the passing beam current but
also provide information about the spatial size of the beam. These slits
are Ta plates attached to a linear motion, allowing the position to be
controlled and determined from the exterior of the beamline. Each slit
is isolated from the beamline, allowing the current hitting the slit to
be read out at the console or to some other diagnostic program. The
limiting factors for using these slits as a diagnostic device is their
sensitivity, since unlike Faraday cups the slits used within St.\ George
are not suppressed. The slits are then used as rough spatial diagnostics
within the separator itself.

The quartz viewers used within St.\ George are divided between two
types: an exterior quartz located at various exit ports, and an interior
quartz that must be removed from the beamline. The locations of these
camera systems are at the 0\degree{} exit ports of dipoles $B_1$, $B_3$,
and $B_5$, the end of the detector chamber, and at focal planes $F_1$
and $F_2$. The quartz are used for beam alignment and tuning the
magnetic and electric elements of St.\ George, as explained in
Section~\ref{sec:commissioning}. When beam strikes the quartz material,
the quartz fluoresces and the camera mounted immediately behind records
the image and transmits it to the control console. Since this
fluorescence is dependent on the beam intensity, minimum currents of
200~nA are used to ensure that the beam shape can be accurately
identified. Additionally, high intensity beams can melt the quartz
material if left impinging on the system for extended periods of time,
so maximum currents in the range of $1.5-2.5$~$\mu$A were used,
depending on the actual beam particle selected.


\section{Detector System and Data Acquisition}
\label{sec:detector}

The full St.\ George detector system consists of a pair of micro channel
plate based time-measurement detectors and a single 16-strip Si detector
for energy deposition. Combined, this detection system provides particle
identification by way of the \textit{Time\--{}of\--{}Flight vs.\ Total
Energy} method. As residual beam particles can make it to the detector
plane, either through scattering or charge exchange events, particle
identification is required to provide final discrimination and beam
suppression (see, for example, \cite{James1988, Angulo2001, Engel2005,
DiLeva2008} and others). As this experiment did not require the use of
the full detection system, only the Si detector will be discussed
further.

The Si detector is a Canberra PIPS (Passivated Implanted Planar Silicon)
model PF-16CT-58*58-300EB, which has a detector area of $58\times 58$~mm
and is divided into 16 individual strips. The 16 strips allow for
spatial resolution in one dimension, commonly taken to be the horizontal
direction. For the experiment discussed herein, the detector was
installed at focal plane $F_2$. Care was taken to ensure that the quartz
viewer at that same location did not interfere with the operation of the
detector.

The electronics for the experiment are relatively simple, needing only
the energy signals from each of the 16 strips. A Mesytec MSCF-16 shaping
filter amplifier provided amplification and signal shaping following the
preamplifier, and a Caen V785 32-channel multi-event peak sensing ADC
transformed those signals for the data acquisition system. Power to the
detector was provided by Mesytec MHV-4 high precision bias supply unit,
and was biased up to +40~V. Leakage current from the detector during the
run was $\approx 1.4$~$\mu$A, but this high value was later shown to be
due to the cable shielding internal to the beamline contacting the
shielding installed to protect the detector when fully retracted. During
the energy calibration runs where no detector shielding was present, the
leakage current was $\approx 0.3$~$\mu$A. The electronics used are the
same as for a standard St.\ George experiment, allowing for an
additional test of a subsystem of the full acquisition system. Data was
recorded using the VM-USB crate connected to the St.\ George DAQ
computer.


\section{Target Chamber}
\label{sec:target}

A commissioning target chamber was designed and built specifically for
running the commissioning experiments and solid target studies. The
chamber makes use of two of the turbo-molecular pumps from the Hippo
target to provide a high ($\approx 7\times 10^{-8}$~torr) vacuum at the
beginning of the separator and around the target location. The
commissioning chamber consists of a pair of electrostatic plates within
a rotating chamber, a target ladder, and a Faraday cup. The rotating
chamber allows the combined target ladder and electrostatic plates to
rotate through a range of $\approx 160\degree$ (limited by the physical
space around the target location) while maintaining a high vacuum, thus
the target location does not need to be vented in order to change the
angle of the target ladder and plates.

The electrostatic plates can be powered up to a maximum of 10~kV each
from two single-phase high voltage power supplies and are operated
remotely using an Arduino-based controller. The properties of the plates
(length, width, separation, etc.) were determined such that the produced
electric field would be as homogeneous as possible within the limited
space and provide a beam deflection of up to 40~mrad from the target
location, within the physical limits of the chamber. Experimentally,
deflections of 45~mrad have been achieved, allowing a full angular phase
space sweep to be performed. Switching the polarity of the deflector
plates must be done at the power supplies. Since the test beam particles
have a set electric rigidity $E\rho$, the maximum deflection of the
particles may be limited by the upper voltage of the power supply. When
not in use, the current striking the deflector plates could be monitored
at the control console to use as an additional beam diagnostic.

The target ladder contains a 6.35~mm diameter collimator and a 2.06~mm
collimator, separated by 11.1~mm. The large collimator is primarily used
for mounting self-supporting solid targets, while the smaller collimator
is primarily used for beam alignment and focusing purposes. The ladder
is mounted on a high precision, manually controlled linear motion drive.
The ladder may be fully removed from the beamline, and the central axis
is aligned with the physical location of the jet target within Hippo. A
thin Al foil was mounted for the experiment described herein.

The Faraday cup following the target chamber is an isolated FN-style
cup. The back of the cup can be actuated open to allow beam to pass into
the separator. The Faraday cup allowed for beam with a maximum
deflection of $\approx 45$~mrad to enter St.\ George, as measured using
the deflector plates. This larger angular acceptance at the beginning of
the separator ensured that the diagnostic equipment did not have a
detrimental affect on the experiment and based the final acceptance at
$F_2$ to be based on the specific tune of the separator. During some
tests, the deflector plates were seen to intercept some of the beam when
deflecting to 45~mrad, but since this value is again larger than the
designed acceptance of St.\ George, this was deemed to not be
detrimental.
