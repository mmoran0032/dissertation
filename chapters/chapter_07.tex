
\chapter{CONCLUSIONS}
\label{ch:conclusions}

The work supporting the commissioning of the St.\ George recoil
separator can be divided into two distinct sections: direct beam studies
and reaction studies. Direct beam studies were used to verify the
$\Delta E/E = 7.5$\,\% energy acceptance across a range of electric and
magnetic rigidities and provide the first studies of the the angular
acceptance properties of the separator. The use of multiple beams at
different magnetic and electric rigidities allowed the parameter space
that St.\ George is designed to transport to be probed with direct
beams, even though those beams are not the exact recoil particles that
will be transported during a cross section measurement. The use of
reaction studies to support the commissioning work allows for the
determination of the angular \textit{and} energy acceptances jointly for
the rigidity of the produced particle, replicating the situation when
measuring the proposed $(\alpha,\gamma)$ reactions of astrophysical
interest.

The use of the \alpa{} reaction as this first such case was to verify
the $\Delta\theta = 40$~mrad angular acceptance under experimental
conditions and the explore the possibility of using St.\ George to
study additional reactions at zero degrees. The use of the separator
under these conditions can support multiple experimental areas within
the Nuclear Science Laboratory and opens up the possibility of the
separator to be used for a wider range of astrophysically important
reactions. These reaction studies may be performed under a similar
experimental setup, where the detector system is positioned just beyond
the Wien filter, or where the entirety of St.\ George is used for the
incident beam rejection.

\section{Next Steps}
\label{sec:next-steps}

As a further test of the angular acceptance capabilities of St.\ George,
the experiment discussed herein can be repeated by sending the produced
$\alpha$ particles to the final detector plane. The transmission of
these particles through the entirety of the separator would be able to
use the standard ion transport tune for the separator, avoiding the
requirement of determining an additional set of magnetic field
strengths. A test under these conditions would be a verification of the
angular acceptance under similar conditions as will be seen for
$(\alpha,\gamma)$ experiments during the regular operation of the
separator. The required work to determine the correct field settings for
the final third of St.\ George was beyond the scope of this work, but
will be required for any future study using the detection system at the
end of St.\ George.

In cases where the detection system is placed just beyond the Wien
filter at focal plane $F_2$, care must be taken to properly ground and
shield the cables connecting the detector to the DAQ. This experiment
used a suboptimal bias voltage, resulting in additional noise at low
energy and a drastic decrease in detector resolution. These facts
combined prevented the direct identification of the produced $\alpha$
particles and the final rejection of the remaining incident protons such
that an energy threshold was used to discriminate between the two
particles.

For each of the sources of uncertainty, a solution can be implemented to
reduce those uncertainties in future experiments. These solutions can be
both procedural, where the process to measure a reaction is altered, and
physical, where diagnostic equipment could be utilized to provide a
better understanding of the underlying assumptions. Some of the
uncertainty will be irreducible, but the reducible portion of the
uncertainty can be minimized in future experiments.

The first such change is to include an offset Si detector at the target
location. This offset detector would be used to monitor the live beam
current by measuring the elastically-scattered incident beam and
relating that to the incident beam current. The scattered particles
would be detected at a small, fixed angle where the count rate at that
detector can be expressed in terms of the incident beam current. The
scattered particles can also be used as a measure of the energy of the
incident beam beyond what the accelerator is calibrated to deliver. By
using this detector, the beam current and energy do not have to be
estimated based on periodic measurements or settings of the accelerator,
which allows for the contribution of these factors to be more precisely
known. The live monitoring will also reduce the timing uncertainty, as
interruptions during the data collection process would be eliminated in
the best-case scenario, which adjusts the procedure through which a
cross section is measured. Finally, a full acceptance check, where the
energy and angle are adjusted within the entire range of values, can
provide a limit to the systematic uncertainty caused by the potential
for an asymmetric angular acceptance.

For follow-up experiments, there are multiple options that can provide
an improved outcome for the experiment, based on the limitations and
findings presented within. The possibilities of utilizing St.\ George
for experiments past the planned and designed $(\alpha,\gamma)$
reactions allows for a wider set of potential astrophysical studies to
be performed. When combined with a future supersonic jet gas target, the
full TOF-vs.-$\Delta E$ detector, and the full separator system, St.\
George will quickly become a premier facility for nuclear astrophysics.


\section{Closing Thoughts}
\label{sec:closing-thoughts}

Fully commissioning a recoil separator requires a vast amount of
coordinated resources and time, and is something that would not be
possible without a large and motivated team behind it. The attention to
detail required for each individual magnet setting and the process
through which that magnet is set can only be accomplished through
extreme rigor on the experimental side as well as on the logging and
information dissemination side. By being explicit about the choices
taken and the results, the progress of the team and the experimental
work can be ensured in situations where that problem arises again in the
future. Lacking in any of these areas can lead to an increase in the
time required to commission a single section of the separator.

Once sections are commissioned, experimental work building off of the
sections that are now experimentally accessible and understood should be
done to ensure that the understanding of that section is accurate and to
provide an avenue for data analysis for the experimenters involved, both
to maintain morale and to provide for the possibility of publishing
incremental improvements. This experimental work is the outcome of
including these experimental ``checkpoints'' into the grander
commissioning work required for St.\ George, and should be included as
the additional components of the St.\ George system, especially the
addition of the gas target, moving forward.

The next reactions that will be probed by St.\ George are
\react{\mnuc{3}{He}}{\alpha}{\gamma}{\mnuc{7}{Be}},
\react{\mnuc{14}{N}}{\alpha}{\gamma}{\mnuc{18}{F}},
\react{\mnuc{15}{N}}{\alpha}{\gamma}{\mnuc{19}{F}},
\react{\mnuc{12}{C}}{\alpha}{\gamma}{\mnuc{16}{O}}, and
\react{\mnuc{18}{O}}{\alpha}{\gamma}{\mnuc{22}{Ne}} among others. These
reactions require that St.\ George and its gas target and detector
systems be fully commissioned. Due to the necessity of separating out
the heavy reaction products from the incident beam, where the two
particles have rigidities closer to each other, the final segment of
St.\ George must be utilized. Final particle identification augmented by
the full detector system will improve the beam suppression up to and
potentially above the $10^{15}$ level, which will be required for some
of the reactions. Each of these additional reactions will be used both
for their astrophysical importance as well as testing the properties of
St.\ George. For instance, the reaction
\react{\mnuc{18}{O}}{\alpha}{\gamma}{\mnuc{22}{Ne}} has an angular and
energy spread near the maximums (39.2~mrad and 7.8\%, respectively),
when measured at $E_{\rm beam} = 1.94$~MeV (lab energy). Additional
reactions that test the limits of the separator are outlined in
\cite{Couder2008}.

The work described sets the initial footing for these following tests.
The cumulative usage and understanding of the separator iteratively
improves the operation of the system for all subsequent runs.
Eventually, the improved magnetic field settings will minimize the long
tuning process, allowing for future experiments to spend less time
tuning between runs and more time measuring the low count rate,
off-resonance regions of the cross section. These regions can be
extremely informative for the $R$-matrix formulation of the cross
section, and using St.\ George provides the unique ability to probe
these regions.

The design of St.\ George has also been adopted for SECAR (SEparator for
CApture Reactions), a new recoil mass separator being installed at the
Facility for Rare Isotope Beams (FRIB) at Michigan State
University~\cite{Berg2018}. SECAR is designed to measure
$({\rm p},\gamma)$ and $(\alpha,\gamma)$ reactions in inverse
kinematics. Of particular interest are the design of the Wien filters,
the use of similar beam optics codes during the design process, and the
utilization of a recirculating gas target to provide a pure target. The
continuing commissioning of St.\ George will have the added benefit of
improving procedures at SECAR, including the use of the camera systems
developed to help tune the test beams. Both systems will benefit from
the continued collaboration between the two teams.

As commissioning a recoil separator relies on more than just the
separator itself, the work requires a flexible and multi-faceted
approach to readying the system for experimental work. The results
presented within on the initial commissioning work and the use of St.\
George for additional reaction studies is promising for the future of
the facility.
