\chapter{INTRODUCTION}

The elements making up the universe were formed during a variety of processes,
beginning with Big Bang Nucleosynthesis (BBN) that formed the lightest
elements. Those elements common to life on Earth were primarily formed through
burning processes inside of stars, grouped together under the title of
Stellar Nucleosynthesis. Depending on the conditions within the stellar
environment, which are characterized by macroscopic qualities about the star
(temperature, pressure, mass, etc.) and the elemental composition of the
stellar interior where the burning process takes place, the reactions
accessible to the nuclei within the star differ. The creation and destruction
of different elements and isotopes may be inhibited or enhanced by these
differing conditions, and the study of these processes at the nuclear level
has spawned the field of nuclear astrophysics in order to understand the inner
workings of these stars.

The specific and directed study of those nuclear reactions that have an effect
on the properties or life cycle of celestial bodies can be grouped under the
umbrella term \emph{nuclear astrophysics}. These reactions may take place
during the lifecycle of a star, during a cataclysmic event within the universe
such as a black hole merger or gamma ray burst, or at the beginning of the
universe itself. Additionally, the decay of various isotopes can also play a
major role within this domain, either as part of a sequence of reactions or
independently. The entire field of nuclear astrophysics was conceived in the
seminal papers [B2FH] and [Other], which have been used as a basis for much of
the work following.

For astrophysical reactions, the properties of the environment can play a major
role in how quickly the reaction proceeds or if it is even energetically
allowed. Cross sections for many of these reactions rapidly decrease at lower
energies, requiring increasingly sophisticated detection methods in order for
the energy dependence of the cross section to be determined. Due to the
decreasing cross sections, work has primarily focused on strong resonances in
this low-energy regime.

\section{Stellar Burning}

Stars in hydrostatiic equilibrium can produce energy through a number of
different reaction channels based on the mass, temperature, and isotopic
enrichment of the stellar interior where the burning process takes place. The
net result of stellar burning is the fusion of lighter isotopes into heavier
isotopes and the release of energy. A single star may undergo multiple distinct
burning stages during its lifecycle, with each subsequent burning stage
occuring at progressively hotter temperatures until a point at which the energy
produced cannot maintain hydrostatic equilibrium.

The isotopic abundances in the stellar interior at a given point in time is
based on the initial abundances within the interstellar medium when the cloud
condensed into the star, and the reaction channels that are available in the
stellar interior during the lifespan of the star.

\subsection{Low-Mass Hydrogen Burning}

The fusion of four \nuc{1}{H} nuclei into a single \nuc{4}{He} nucleus is
called \emph{hydrogen burning}, and can take place through numerous reaction
sequences at within vastly different temperature ranges. The overall outcome
of the reaction chain given by
\[
    4\mnuc{1}{H} \rightarrow{} \mnuc{4}{He} + 2e^+ + 2\nu_e
\]
is the conversion of \nuc{1}{H} into \nuc{4}{He} and the release of
approximately 26.7~MeV in energy from that fusion. The underlying processes
that are undergone within stars under steady state conditions are grouped under
\emph{hydrostatic hydrogen burning}. Common environments for such burning
processes are core hydrogen burning in stars similar to our sun in mass and
metallicity, and in hydrogen burning shells within asymptotic giant branch
(AGB) stars. The differences in the isotopic composition and temperature of the
stellar interior can allow for different processes to take place.

Stars similar to our sun fuse hydrogen through the proton-proton ($pp$) chains,
which are described by the reaction sequences
\begin{align*}
    \rm{PP-1:\,}& \mreact{\rm{p}}{\rm{p}}{e^+\nu_e}{
        \mreact{\mnuc{2}{H}}{\rm{p}}{\gamma}{
        \mreact{\mnuc{3}{He}}{\mnuc{3}{He}}{2\rm{p}}{\mnuc{4}{He}}}} \\
    \rm{PP-2:\,}& \mreact{\mnuc{3}{He}}{\alpha}{\gamma}{\mnuc{7}{Be}}
        \left(e^-,\bar{\nu}_e\right)\mreact{\mnuc{7}{Li}}{\rm{p}}{\alpha}{\mnuc{4}{He}} \\
    \rm{PP-3:\,}& \mreact{\mnuc{7}{Be}}{\rm{p}}{\gamma}{\mnuc{8}{B}}
        \left(\beta^+\right)\mnuc{8}{Be}\left(\alpha\right)\mnuc{4}{He},
\end{align*}
where the PP-2 and PP-3 chains are branches off from the PP-1 and PP-2 chains,
respectively, at the point after the initial nuclei is created within the
chain. Typical temperatures for this burning process are on the order of
$T_6 \approx 8-55$, which the core temperature of our sun ($T_6 = 15.6$) falls
squarely within~\cite{Iliadis} and others?.

The Carbon-Nitrogen-Oxygen (CNO) cycle is an additional pathway for stable
hydrogen burning accessible when the stellar interior has been enriched with
heavier nuclei. The relative abundances of the catalytic elements C, N, O, and
F will change based on the relative reaction rates. The CNO cycles are
described by the cyclic reaction sequences
\begin{align*}
    \rm{CNO-1:\,}& \mreact{\mnuc{12}{C}}{\rm{p}}{\gamma}{\mnuc{13}{N}}
        \left(\beta^+\right)\mreact{\mnuc{13}{C}}{\rm{p}}{\gamma}{
            \mreact{\mnuc{14}{N}}{\rm{p}}{\gamma}{\mnuc{15}{O}}}
        \left(\beta^+\right)\mreact{\mnuc{15}{N}}{\rm{p}}{\alpha}{\mnuc{12}{C}} \\
    \rm{CNO-2:\,}& \mreact{\mnuc{14}{N}}{\rm{p}}{\gamma}{\mnuc{15}{O}}
        \left(\beta^+\right)\mreact{\mnuc{15}{N}}{\rm{p}}{\gamma}{
        \mreact{\mnuc{16}{O}}{\rm{p}}{\gamma}{\mnuc{17}{F}}}\left(\beta^+\right)
        \mreact{\mnuc{17}{O}}{\rm{p}}{\alpha}{\mnuc{14}{N}} \\
    \rm{CNO-3:\,}& \mreact{\mnuc{15}{N}}{\rm{p}}{\gamma}{
        \mreact{\mnuc{16}{O}}{\rm{p}}{\gamma}{\mnuc{17}{F}}}\left(\beta^+\right)
        \mreact{\mnuc{17}{O}}{\rm{p}}{\gamma}{\mnuc{18}{F}}\left(\beta^+\right)
        \mreact{\mnuc{18}{O}}{\rm{p}}{\alpha}{\mnuc{15}{N}} \\
    \rm{CNO-4:\,}& \mreact{\mnuc{16}{O}}{\rm{p}}{\gamma}{\mnuc{17}{F}}\left(\beta^+\right)
        \mreact{\mnuc{17}{O}}{\rm{p}}{\gamma}{\mnuc{18}{F}}\left(\beta^+\right)
        \mreact{\mnuc{18}{O}}{\rm{p}}{\gamma}{
            \mreact{\mnuc{19}{F}}{\rm{p}}{\alpha}{\mnuc{16}{O}}}
\end{align*}

For the CNO cycles, the temperature of the interior of the star and the
individual reaction rates in question determine which cycle dominates as well
as which catalytic isotope will, in steady state conditions, have the highest
fractional abundance. These differing abundances based on the properties of the
star can have large consequences on the burning phases at higher mass and
temperature due to the isotopic enrichment available.

As the burning progresses, an inert core primarily consisting of \nuc{4}{He} is
produced. Due to the lower temperatures in the core at this stage in stellar
evolution, no He burning takes place. This inert core may ignite during the
red giant stage of stellar evolution, at which point the He within the outer
shell of this inert core ignites and helium burning becomes an open channel
for stellar energy production.

\subsection{Helium Burning}

The primary nucleosynthesis products of Helium burning are \nuc{12}{C} and
\nuc{16}{O}, produced through the \emph{triple-$\alpah$ process} given by
\react{\mnuc{4}{He}}{\alpha\alpha}{\gamma}{\mnuc{12}{C}} and
\react{\mnuc{12}{C}}{\alpha}{\gamma}{\mnuc{16}{O}},
respectively~\cite{Aliotta2016}. The relative abundance ration C/O between
these two isotopes greatly affects the subsequent evolution of the star at the
end of the He burning phase. He burning progresses first through the
triple-$\alpha$ process until such a time that enough \nuc{12}{C} has built up
within the He core, at which point the production of \nuc{16}{O} can
begin~\cite{Aliotta2016}.

He burning may take place in varied stellar envvironments. Following the end of
H burning within main sequence stars, the stellar interior compresses since not
enough energy is being produced. This compression increases the pressure and
temperature near the stellar core to the point that He burning becomes
energetically favorable~\cite{CarrollOstlie}. The star expands, transitioning
from a main sequence star to a red giant star, and maintains hydrostatic
equilibrium due to the increased outward pressure produced by He burning.

Asymptotic Giant Branch (AGB) stars are a phase of stellar evolution following
the turnoff from the main sequence after initial H burning
completes~\cite{CarrollOstlie}. The stellar interior contains two distinct
burning regions: H and He. These regions become alternatingly active as the
reaction sequences progress within each region~\cite{Aliotta2016}. During the
early AGB phase, the H burning shell is nearly inert until the point when
mixing occurs between the H- and He-rich regions near the center of the
star~\cite{CarrollOstlie}. The star then enters the thermal pulse AGB phase,
defined by a reactivated H burning region and periodic He shell flashes caused
by the rapid introduction of He from the H burning shell onto the top layer of
the He burning shell. The flash drives the H burning shell outward, cooling it
and reducing the burning rate, which in turn ends the shell flash. As the
He burning subsides, the H burning shell can reactivate, restarting the
cycle~\cite{CarrollOstlie}. Temperatures required for this cyclic process are
in the $T_6 = 20 - 60$ temperature range, where termeratures at the higher end
create a more-efficient environment for the CNO reactions~\cite{Boeltzig2016}.

During these shell flashes, additional He burning reactions activate. Two such
reactions, \react{\mnuc{13}{C}}{\alpha}{\rm{p}}{\mnuc{16}{O}} and
\react{\mnuc{22}{Ne}}{\alpha}{\rm{n}}{\mnuc{25}{Mg}}, are sources for the
strong and weak $s$-process, respectively~\cite{Aliotta2016}. The strong
$s$-process is responsible for approximately half of the cosmic abundances for
isotopes with mass $A \geq 90$, while the weak $s$-process contributes to the
isotopic abundances for isotopes with mass within the range
$60 \leq A \leq 90$~\cite{Aliotta2016}.

\subsection{Additional Burning Processes}

While the above reation sequences play a large role in the energy production of
stars, additional burning sequences are available for more massive and hotter
stars. These burning cycles can be through of in similar terms to
the CNO cycles, where baseline processes convert $4H\rightarrow\mnuc{4}{He}$
and additional reactions provide a ``breakout'' channel to higher burning
processes. The reaction \alpa{} is the final step in the MgAl burning cycle
(shown in Figure~\ref{fig:mgal}) that provides for the cycling of the catalytic
nuclei. These burning cycles activate at elevated temperatures in the range of
$T_6 = 60 - 100$ within massive AGB stars~\cite{Boeltzig2016}.

\begin{figure}[t]
    \begin{center}
        \label{fig:mgal}
        \centerline{\includegraphics[width=0.95\textwidth]%
            {figures/cno_nena_mgal.pdf}}
        \caption[Schematic of burning cycles]{Schematic of burning cycles in
            the $12 \leq A \leq 27$ mass range. These catalytic cycles play a
            large role in various abundance measurements and burning processes
            in red giant stars. The reaction in question is indicated. Figure
            adapted from \cite{Boeltzig2016}.}
    \end{center}
\end{figure}


%%% end section %%%

\section{Resonant Capture Reaction Theory}

The outcome of a particular reaction is tied to the underlying properties of
the reaction and the nuclei in question. For \alpa{}, the reaction progresses
through the compound nucleus \nuc{28}{Si}, as shown in the diagram below.

[FIGURE]

The energy levels within \nuc{28}{Si} govern how the reaction progresses. When
the incident energy of the proton beam directly matches an energy level within
\nuc{28}{Si}, the reaction is described as a \emph{resonant capture} reaction,
where the likelihood of that reaction taking place is higher due to the energy
of the incident beam and the energy level within the compound nucleus. The
likelihood of the reaction, extended across the entire energy domain, is the
\emph{cross section} of the reaction. This cross section, for a given reaction,
is a measure of the likelihood of the reaction for any inicident beam energy,
and is related to the properties of the energy levels within the compound
nucleus. Additionally, if a reaction is being studied at a particular angle for
the emitted particle, in this case the produced $\alpha$, the cross section
also includes information about this likelihood.

The properties of the cross section, and thus the underlying energy levels
within the compound nucleus, depend on the properties of the energy levels,
specifically the spin-parity of the nuclear state. The spin and parity of the
nuclear states available for resonant reactions are determined by the quantum
mechanical selection rules from the spin, parity, and angular momentum of the
target and projectile nuclei, given as
\[
    \left|J_{\rm{p}} - J_{\mnuc{27}{Al}}\right|
    \leq J_{\mnuc{28}{Si}} \leq
    \left|J_{\rm{p}} + J_{\mnuc{27}{Al}}\right|
\]
and
\[
    \pi_{\mnuc{28}{Si}} = \pi_{\rm{p}}\pi_{\mnuc{27}{Al}}(-1)^{\ell_{\rm{p}}},
\]
where the properties of the compound nuclear state in \nuc{28}{Si} are
$J_{\mnuc{28}{Si}}$ and $\pi_{\mnuc{28}{Si}}$. Given that the ground state of
\nuc{27}{Al} is $5/2+$ and the spin-parity of protons are $1/2+$, the only
accessbile spins for the reaction are $J = 2\rm{ and }3$. The parity of the
allowed states depends on the angular momentum of the incident proton beam.

From the level state in \nuc{28}{Si}, we can perform the same calculations to
determine the allows states that can decay into $\mnuc{24}{Mg} + \alpha$. Since
both the ground state of \nuc{24}{Mg} and the spin-parity of $\alpha$ is $0+$,
only natural parity states within \nuc{28}{Si}
($J^{\pi} = 0+,\, 1-,\, 2+,\, \ldots$) are accessible, which in turn limits the
allowed states for the entrance channel $\mnuc{27}{Al} + \rm{p}$.

\subsection{Gamow Energy}

Within each stellar environment and reaction of interest, the energy range
where the reaction will take place within the star is described as the Gamow
peak.

- description of why this is the peak (electric potential barrier and thermal increase)

- peak energy and width

- cite energies of interest for ALPA

\subsection{Astrophysical $S$-factor}

The cross section for charged particle reactions drops off precipitously as
the energy decreases due to repulsion from the Coulomb barrier. At the energy
decreases, the cross section drops by orders of magnitude, which can make
analysis of low energy phenomena more difficult. A solution to this problem is
to define the \emph{astrophysical $S$-factor} which helps to mitigate this
problem.

The $S$-factor is defined in reference to the cross section $\sigma(E)$ as
\[
    S(E) = E\sigma(E)e^{2\pi\eta},
\]
where $\eta = EQUATION$ is the Sommerfield parameter.


\subsection{Angular Correlation}
The state within \nuc{28}{Si} that gets populated during the reaction also
determines the angular correlation of the reaction productions

By taking these contributions into account, the cross section of the reaction
can be determined for a given incident beam energy and a desired exit angle for
the reaction products.

- angular correlations, note that our reaction shows some for some reasonance

- cite the old paper

\subsection{Reaction Rate}
Under astrophysical conditions, the quantity of interest is the nuclear
reaction rate evvaluated at the temperature of the stellar environment. The
reaction rate is related to the temperature and the energy-dependent cross
section $\sigma(E)$ through
\begin{equation}
\label{eq:reaction-rate}
    N_A\langle\sigma\nu\rangle_{01} = N_A\left(\frac{8}{\pi m_{01}}\right)^{1/2}
        \frac{1}{(kT)^{3/2}}\int_0^{\infty} E\sigma(E)e^{-E/kT}\,\rm{d}E,
\end{equation}
where $m_{01} = m_0m_1/(m_0 + m_1)$ is the reduced mass of the particle
pair~\cite{Iliadis}. In cases where a single narrow resonance dominates the
cross section, Eq.~\ref{eq:reaction-rate} may be simplified to
\[
    N_A\langle\sigma\nu\rangle = N_A\left(\frac{2\pi}{m_{01}kT}\right)^{3/2}
        \hbar^2e^{-E_r/kT}\omega\gamma,
\]
where $E_r$ is the resonance energy and $\omega\gamma$ is the resonance
strength~\cite{Iliadis}. The resonance strength is defined in reference to the
partial widths of the reaction $\omega\gamma = \omega\Gamma_a\Gamma_b/\Gamma$,
where $a$ and $b$ are the entrance and exit channels. In cases where several
isolated resonances contribute to the reaction rate, their contribution may be
summed.


%%% end section %%%

\section{Recoil Separation}
\label{sec:ch01-recoil-separation}

The experimental process in which the reaction products produced by a direct
beam can be filtered out from that direct beam in order to be detected is
called \emph{recoil separation} or alternatively \emph{recoil mass separation}.
A recoil separator is the system, consisting of a sequence of electromagnetic
elements, designed to perform this task.

\subsection{Motivation}
\label{ssec:recoil-separation-motivation}

Recoil mass separation was conceived as an alternate way to measure the cross
sections of radiative capture reactions. These reactions had previously been
studied by detecting the produced $\gamma$ rays, subject to the limitations
previously discussed. The heavy reaction product can instead be detected by a
detector situated behind the target, assuming that the target is thin enough to
allow the produced recoils to leave the target. In this thin target case, the
incident beam will likely pass through the target as well, making it a source
of background at the detector plane. In the cases of interest for nuclear
astrophysics, this background count rate could be $\times 10^{15}$ that of the
particles of interest and may cause damage to the detector.

The produced recoils may be filtered out from the incident beam by
electromagnetic elements situated between the target and the detector. The
interaction between the heavy incident beam with mass $A$ and linear momentum
$p$ and the $\alpha$ particles within
the target produces a heavy compound nucleus with mass $A + 4$ and momentum
$p$. Ignoring the effect of the emitted $\gamma$ ray on the momentum and
assuming that there is no spread in the momentum, the use of electrostatic
elements can separate the recoils from the beam based on their different
magnetic and electric rigidities. The magnetic rigidity is defined as
\begin{equation}
    \label{eq:brho}
    B\rho\textrm{ [Tm]} = \frac{p}{q} = \frac{\sqrt{2mT}}{q},
\end{equation}
where $p$, $q$, $m$, and $T$ are the momentum, charge state, mass, and kinetic
energy of the desired particle, respectively. The magnetic rigidity
defines the trajectory of the particle's movement within a homogeneous magnetic
field of strength $B$ along a circular path with radius $\rho$. Similarly, the
electric rigidity is defined as
\begin{equation}
    \label{eq:erho}
    E\rho\textrm{ [MV]} = \frac{pv}{q} = \frac{2T}{q}
\end{equation}
with the same variable definitions as before, and defines the circular
trajectory a particle takes within an electric field of strength $E$.

With a single momentum and
velocity (or kinetic energy) selected for, the recoil particles of interest can
be uniquely identified by the optical system. The design of recoil separators
make use of this relatively simple idea as the basis of their design. Despite
this, there have been relatively few recoil separators that have been brought
into service due to the complexities of their design and operation that are not
adequately taken into account in this description.


\subsubsection{Radiative Capture}
\subsection{Experimental Considerations}
\subsubsection{Inverse Kinematics}
\subsection{Recoil Separator Facilities}
The use of recoil separators to study radiative capture reactions has been
explored recently at a number of facilities. The design of St. George is based
on the knowledge gained from the design, construction, and operation of these
previous recoil separator systems. The entire system, inclusive of the beam
source, target, and detector, must be discussed as a whole when evaluating the
capabilities of a given separator.

\subsubsection{CalTech Separator}
% The design and use of recoil separators for astrophysical studies was first
% pursued by Smith \textit{et al.}~\cite{Smith1991}. The requirements of radioactive
% beam studies required the need to develop new detection systems, especially
% considering the effect of differing beam property limits (intensity, purity,
% and emittance) and the desire to detect the produced nuclei to further probe
% the astrophysical conditions. As the initial feasibility system to provide a
% technical proof-of-concept, many of the design choices made and techniques
% used have been adopted by following separators. These include the use of a Wien
% filter for velocity selection, dipole magnets for momentum selection, an
% electrostatic deflector for energy selection, and the use of a gaseous target.
% Additionally, the use of a gamma-ray detector in coincidence with the final
% recoil detection and beam monitoring with an offset Si detector at the target
% location are also common choices that have been adapted at the other separators.

The design and use of recoil separators for nuclear astrophysics research was
pioneered by Smith *et al.*. This separator was a proof-of-concept design to
determine the feasibility of performing reaction studies with this technique.

\subsubsection{DRAGON}
The DRAGON recoil separator at TRIUMF-ISAC was built for the same reasons as ARES,
but differs in the actual construction and usage of the separator.
% Success of separator
The separator
itself uses two large magnetic dipoles for momentum separation and to electric
dipoles for energy selection~\cite{Engel2005}. The separator also contains
steering elements within the beamline to aide in transporting the recoils to
the detector plane. The extended gas target is surrounded by a large BGO
gamma-ray detector for coincidence purposes

\subsubsection{ARES}
The Astrophysics REcoil Separator (ARES) was built at Louvain-la-Neuve to
study $(\rm{p},\gamma)$ and $(\alpha,\gamma)$ reactions using radioactive
incident beams provided by the CYCLONE44 cyclotron~\cite{Angulo2001}.
Self-supporting solid targets, containing the required H or He, were used for
the reaction studies. The system is designed with a single magnetic dipole for
momentum selection and a Wien filter for velocity selection, along with
multiple magnetic quadrupoles (one triplet and two doublets) to maintain the
transportation of the to the
detector system. The condensed and limited size of the separator is based on
the constraints of the experimental hall~\cite{Couder2003}. The detector system
consists of a single $\Delta E − E$ gas telescope which separates out the reaction
products from the remaining incident beam particles. The initial test of the
separator used a stable incident beam to compare to results obtained by other
methods within the lab, and the focus of the initial work was on low-lying
resonances of astrophysical interest.

\subsubsection{ERNA}
% Look up ERNA design/commission paper(s)
The European Recoil separator for Nuclear Astrophysics (ERNA) at (location)...

\subsubsection{St.\ George}
The separator consists of six dipole magnets, eleven quadrupole magnets, and a
Wien filter. The separator was designed to accept recoils with a maximum
energy and angular spread of $\Delta E/E = \pm7.5\%$ and
$\Delta\theta = \pm40$~mrad, respectively, and to provide a mass separation
of $m/\Delta m = 100$ and beam suppression of a factor $\geq 10^{15}$. Combined
with the HIPPO (High-Pressure Point-like target) supersonic gas jet target,
St.\ George will be primarily used to study low energy $(\alpha,\gamma)$
reactions using stable beams.


%%% end section %%%

\section{Beam Optics}
\label{sec:ch01-beam-optics}
% still unsure how much will go in here...probably just the absolute basics
% check out later chapters, because I have some stuff in there

The understanding of how recoil separators work is grounded in the theory of
beam optics which describe the effect of electric and magnetic fields on moving
charged particles. These moving particles are focused and directed by the
electromagnetic elements, and their action on particles can be modeled and
optimized to transport particles with various properties down a beam line to a
desired location.

- parts of beam optics?

- Louisville theorom (conservation of phase space)

- connection to properties (resolution?)

- references to code/COSY?

\subsection{Considerations}
\subsection{Resolution}
