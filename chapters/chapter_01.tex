\chapter{INTRODUCTION}

The elements making up the universe were formed during a variety of processes,
beginning with Big Bang Nucleosynthesis (BBN) that formed the lightest
elements. Those elements common to life on Earth were primarily formed through
burning processes inside of stars, grouped together under the title of
Stellar Nucleosynthesis. Depending on the conditions within the stellar
environment, which are characterized by macroscopic qualities about the star
(temperature, pressure, mass, etc.) and the elemental composition of the
stellar interior where the burning process takes place, the reactions
accessible to the nuclei within the star differ. The creation and destruction
of different elements and isotopes may be inhibited or enhanced by these
differing conditions, and the study of these processes at the nuclear level
has spawned the field of nuclear astrophysics in order to understand the inner
workings of these stars.

The specific and directed study of those nuclear reactions that have an effect
on the properties or life cycle of celestial bodies can be grouped under the
umbrella term \emph{nuclear astrophysics}. These reactions may take place
during the lifecycle of a star, during a cataclysmic event within the universe
such as a black hole merger or gamma ray burst, or at the beginning of the
universe itself. Additionally, the decay of various isotopes can also play a
major role within this domain, either as part of a sequence of reactions or
independently. The entire field of nuclear astrophysics was conceived in the
seminal papers [B2FH] and [Other], which have been used as a basis for much of
the work following.

For astrophysical reactions, the properties of the environment can play a major
role in how quickly the reaction proceeds or if it is even energetically
allowed. Cross sections for many of these reactions rapidly decrease at lower
energies, requiring increasingly sophisticated detection methods in order for
the energy dependence of the cross section to be determined. Due to the
decreasing cross sections, work has primarily focused on strong resonances in
this low-energy regime.

\section{Stellar Burning}

Stars in hydrostatiic equilibrium can produce energy through a number of
different reaction channels based on the mass, temperature, and isotopic
enrichment of the stellar interior where the burning process takes place. The
net result of stellar burning is the fusion of lighter isotopes into heavier
isotopes and the release of energy. A single star may undergo multiple distinct
burning stages during its lifecycle, with each subsequent burning stage
occuring at progressively hotter temperatures until a point at which the energy
produced cannot maintain hydrostatic equilibrium.

Describe stellar burning connection to stellar evolution

Cite a bunch of stuff here

\subsection{Hydrogen Burning}

The fusion of four \nuc{1}{H} nuclei into a single \nuc{4}{He} nucleus is
called \emph{hydrogen burning}, and can take place through numerous reaction
sequences at within vastly different temperature ranges. The overall outcome
of the reaction chain given by
\[
    4\mnuc{1}{H} \rightarrow{} \mnuc{4}{He} + 2e^+ + 2\nu_e
\]
is the conversion of \nuc{1}{H} into \nuc{4}{He} and the release of
approximately 26.7~MeV in energy from that fusion. The underlying processes
that are undergone within stars under steady state conditions are grouped under
\emph{hydrostatic hydrogen burning}. Common environments for such burning
processes are core hydrogen burning in stars similar to our sun in mass and
metallicity, and in hydrogen burning shells within asymptotic giant branch
(AGB) stars. The differences in the isotopic composition and temperature of the
stellar interior can allow for different processes to take place.

Stars similar to our sun fuse hydrogen through the proton-proton ($pp$) chains,
which are described by the reaction sequences
\begin{align*}
    \rm{PP-1:}& \mreact{\rm{p}}{\rm{p}}{e^+\nu_e}{
        \mreact{\mnuc{2}{H}}{\rm{p}}{\gamma}{
        \mreact{\mnuc{3}{He}}{\mnuc{3}{He}}{2\rm{p}}{\mnuc{4}{He}}}}} \
    \rm{PP-2:}& \mreact{\mnuc{3}{He}}{\alpha}{\gamma}{\mnuc{7}{Be}}
        \left(e^-,\bar{\nu}_e\right)\mreact{\mnuc{7}{Li}}{\rm{p}}{\alpha}{\mnuc{4}{He}} \
    \rm{PP-3:}& \mreact{\mnuc{7}{Be}}{\rm{p}}{\gamma}{\mnuc{8}{B}}
        \left(\beta^+\right)\mnuc{8}{Be}\left(\alpha\right)\mnuc{4}{He},
\end{align*}
where the PP-2 and PP-3 chains are branches off from the PP-1 and PP-2 chains,
respectively, at the point after the initial nuclei is created within the
chain. Typical temperatures for this burning process are on the order of
$T_6 \approx 8-55$, which the core temperature of our sun ($T_6 = 15.6$) falls
squarely within~\cite{Iliadis} and others?.

The Carbon-Nitrogen-Oxygen (CNO) cycle is an additional pathway for stable
hydrogen burning accessible when the stellar interior has been enriched with
heavier nuclei. The relative abundances of the catalytic elements C, N, O, and
F will change based on the relative reaction rates. The CNO cycles are
described by the cyclic reaction sequences
\begin{align*}
    \rm{CNO-1:}& \mreact{\mnuc{12}{C}}{\rm{p}}{\gamma}{\mnuc{13}{N}}
        \left(\beta^+\right)\mreact{\mnuc{13}{C}}{\rm{p}}{\gamma}{
            \mreact{\mnuc{14}{N}}{\rm{p}}{\gamma}{\mnuc{15}{O}}}
        \left(\beta^+\right)\mreact{\mnuc{15}{N}}{\rm{p}}{\alpha}{\mnuc{12}{C}} \
    \rm{CNO-2:}& \mreact{\mnuc{14}{N}}{\rm{p}}{\gamma}{\mnuc{15}{O}}
        \left(\beta^+\right)\mreact{\mnuc{15}{N}}{\rm{p}}{\gamma}{
        \mreact{\mnuc{16}{O}}{\rm{p}}{\gamma}{\mnuc{17}{F}}}\left(\beta^+\right)
        \mreact{\mnuc{17}{O}}{\rm{p}}{\alpha}{\mnuc{14}{N}} \
    \rm{CNO-3:}& \mreact{\mnuc{15}{N}}{\rm{p}}{\gamma}{
        \mreact{\mnuc{16}{O}}{\rm{p}}{\gamma}{\mnuc{17}{F}}}\left(\beta^+\right)
        \mreact{\mnuc{17}{O}}{\rm{p}}{\gamma}{\mnuc{18}{F}}\left(\beta^+\right)
        \mreact{\mnuc{18}{O}}{\rm{p}}{\alpha}{\mnuc{15}{N}} \
    \rm{CNO-4:}& \mreact{\mnuc{16}{O}}{\rm{p}}{\gamma}{\mnuc{17}{F}}\left(\beta^+\right)
        \mreact{\mnuc{17}{O}}{\rm{p}}{\gamma}{\mnuc{18}{F}}\left(\beta^+\right)
        \mreact{\mnuc{18}{O}}{\rm{p}}{\gamma}{
            \mreact{\mnuc{19}{F}}{\rm{p}}{\alpha}{\mnuc{16}{O}}}
\end{align*}

For the CNO cycles, the temperature of the interior of the star and the
individual reaction rates in question determine which cycle dominates as well
as which catalytic isotope will, in steady state conditions, have the highest
fractional abundance. These differing abundances based on the properties of the
star can have large consequences on the burning phases at higher mass and
temperature due to the isotopic enrichment available.

As the burning progresses, an inert core primarily consisting of \nuc{4}{He} is
produced. Due to the lower temperature at this stage in stellar evolution, no
burning of this core takes place.

\subsection{Helium Burning}
% maybe make a section called hydrostatic burning to talk about this and
% H burning

The primary nucleosynthesis products of Helium burning are \nuc{12}{C} and
\nuc{16}{O}, produced through the \emph{triple-$\alpah$ process} given by
\react{\mnuc{4}{He}}{\alpha\alpha}{\gamma}{\mnuc{12}{C}} and
\react{\mnuc{12}{C}}{\alpha}{\gamma}{\mnuc{16}{O}},
respectively~\cite{Aliotta2016}. The relative abundance ration C/O between
these two isotopes greatly affects the subsequent evolution of the star at the
end of the He burning phase. He burning progresses first through the
triple-$\alpha$ process until such a time that enough \nuc{12}{C} has built up
within the He core, at which point the production of \nuc{16}{O} can
begin~\cite{Aliotta2016}.

\subsection{Asymptotic Giant Branch Stellar Burning}

Asymptotic Giant Branch (AGB) stars are the phase of stellar evolution
following...

The stellar interior contains two distinct burning regions: H and He. these
regions become alternatingly active as the reaction sequences progress within
each region~\cite{Aliotta2016}.

Thermal pulse, describe cycling (orange Intro to Astro)

The reaction \react{\mnuc{13}{C}}{\alpha}{\rm{p}}{\mnuc{16}{O}} is the primary
production source of the neutrons for the main and strong $s$-process, which
are the underlying processes determining about half of the cosmic abundances
for isotopes with mass $A \geq 90$. Additionally, the reaction
\react{\mnuc{22}{Ne}}{\alpha}{\rm{n}}{\mnuc{25}{Mg}} may also be activated to
produce the neutrons for the weak $s$-process, producing isotopes in the mass
$60 \leq A \leq 90$ range~\cite{Aliotta2016}.

Should we discuss what the $s$-process is here, or can we ignore it?


While the above reation sequences play a large role in the energy production of
stars, additional burning sequences are available for more massive and hotter
stars.

-   main sequence + stellar evolution (basics)
-   Describe Asymptotic Giant Branch stars
-   Discuss burning regimes (termal pulse, s-process, \alpa{}/MgAl, etc)


\section{Resonant Capture Reaction Theory}
-   Cross sections
-   Reaction rates
-   R-Matrix formalism (?, maybe not needed since I'm not doing a full fit)


%%% end section %%%

\section{Recoil Separation}
\label{sec:ch01-recoil-separation}

The experimental process in which the reaction products produced by a direct
beam can be filtered out from that direct beam in order to be detected is
called \emph{recoil separation} or alternatively \emph{recoil mass separation}.
A recoil separator is the system, consisting of a sequence of electromagnetic
elements, designed to perform this task.

\subsection{Motivation}

Recoil mass separation was conceived as an alternate way to measure the cross
sections of radiative capture reactions. These reactions had previously been
studied by detecting the produced $\gamma$ rays, subject to the limitations
previously discussed. The heavy reaction product can instead be detected by a
detector situated behind the target, assuming that the target is thin enough to
allow the produced recoils to leave the target. In this thin target case, the
incident beam will likely pass through the target as well, making it a source
of background at the detector plane. In the cases of interest for nuclear
astrophysics, this background count rate could be $\times 10^{15}$ that of the
particles of interest and may cause damage to the detector.

The produced recoils may be filtered out from the incident beam by
electromagnetic elements situated between the target and the detector. The
interaction between the heavy incident beam with mass $A$ and linear momentum
$p$ and the $\alpha$ particles within
the target produces a heavy compound nucleus with mass $A + 4$ and momentum
$p$. Ignoring the effect of the emitted $\gamma$ ray on the momentum and
assuming that there is no spread in the momentum, the use of electrostatic
elements can separate the recoils from the beam based on their different
magnetic and electric rigidities, defined as
\[
    B\rho\,\rm{[Tm]} = \frac{p}{q} = \frac{\sqrt{2mT}}{q}
\]
and
\[
    E\rho\,\rm{[MV]} = \frac{pv}{q} = \frac{2T}{q},
\]
respectively, where $q$ is the charge, $T$ is the kinetic energy, $m$ is the
mass, and $v$ is the velocity of the particle. With a single momentum and
velocity (or kinetic energy) selected for, the recoil particles of interest can
be uniquely identified by the optical system. The design of recoil separators
make use of this relatively simple idea as the basis of their design. Despite
this, there have been relatively few recoil separators that have been brought
into service due to the complexities of their design and operation that are not
adequately taken into account in this description.

\subsubsection{Radiative Capture}
\subsection{Experimental Considerations}
\subsubsection{Inverse Kinematics}
\subsection{Recoil Separator Facilities}
The use of recoil separators to study radiative capture reactions has been
explored recently at a number of facilities. The design of St. George is based
on the knowledge gained from the design, construction, and operation of these
previous recoil separator systems. The entire system, inclusive of the beam
source, target, and detector, must be discussed as a whole when evaluating the
capabilities of a given separator.

\subsubsection{CalTech Separator}
% The design and use of recoil separators for astrophysical studies was first
% pursued by Smith \textit{et al.}~\cite{Smith1991}. The requirements of radioactive
% beam studies required the need to develop new detection systems, especially
% considering the effect of differing beam property limits (intensity, purity,
% and emittance) and the desire to detect the produced nuclei to further probe
% the astrophysical conditions. As the initial feasibility system to provide a
% technical proof-of-concept, many of the design choices made and techniques
% used have been adopted by following separators. These include the use of a Wien
% filter for velocity selection, dipole magnets for momentum selection, an
% electrostatic deflector for energy selection, and the use of a gaseous target.
% Additionally, the use of a gamma-ray detector in coincidence with the final
% recoil detection and beam monitoring with an offset Si detector at the target
% location are also common choices that have been adapted at the other separators.

The design and use of recoil separators for nuclear astrophysics research was
pioneered by Smith *et al.*. This separator was a proof-of-concept design to
determine the feasibility of performing reaction studies with this technique.

\subsubsection{DRAGON}
The DRAGON recoil separator at TRIUMF-ISAC was built for the same reasons as ARES,
but differs in the actual construction and usage of the separator.
% Success of separator
The separator
itself uses two large magnetic dipoles for momentum separation and to electric
dipoles for energy selection~\cite{Engel2005}. The separator also contains
steering elements within the beamline to aide in transporting the recoils to
the detector plane. The extended gas target is surrounded by a large BGO
gamma-ray detector for coincidence purposes

\subsubsection{ARES}
The Astrophysics REcoil Separator (ARES) was built at Louvain-la-Neuve to
study $(\rm{p},\gamma)$ and $(\alpha,\gamma)$ reactions using radioactive
incident beams provided by the CYCLONE44 cyclotron~\cite{Angulo2001}.
Self-supporting solid targets, containing the required H or He, were used for
the reaction studies. The system is designed with a single magnetic dipole for
momentum selection and a Wien filter for velocity selection, along with
multiple magnetic quadrupoles (one triplet and two doublets) to maintain the
transportation of the to the
detector system. The condensed and limited size of the separator is based on
the constraints of the experimental hall~\cite{Couder2003}. The detector system
consists of a single $\Delta E − E$ gas telescope which separates out the reaction
products from the remaining incident beam particles. The initial test of the
separator used a stable incident beam to compare to results obtained by other
methods within the lab, and the focus of the initial work was on low-lying
resonances of astrophysical interest.

\subsubsection{ERNA}
% Look up ERNA design/commission paper(s)
The European Recoil separator for Nuclear Astrophysics (ERNA) at (location)...

\subsubsection{St.\ George}
The separator consists of six dipole magnets, eleven quadrupole magnets, and a
Wien filter. The separator was designed to accept recoils with a maximum
energy and angular spread of $\Delta E/E = \pm7.5\%$ and
$\Delta\theta = \pm40$~mrad, respectively, and to provide a mass separation
of $m/\Delta m = 100$ and beam suppression of a factor $\geq 10^{15}$. Combined
with the HIPPO (High-Pressure Point-like target) supersonic gas jet target,
St.\ George will be primarily used to study low energy $(\alpha,\gamma)$
reactions using stable beams.


%%% end section %%%

\section{Beam Optics}
\label{sec:ch01-beam-optics}
% still unsure how much will go in here...probably just the absolute basics
% check out later chapters, because I have some stuff in there

The understanding of how recoil separators work is grounded in the theory of
beam optics which describe the effect of electric and magnetic fields on moving
charged particles. These moving particles are focused and directed by the
electromagnetic elements, and their action on particles can be modeled and
optimized to transport particles with various properties down a beam line to a
desired location.

- parts of beam optics?
- Louisville theorom (conservation of phase space)
- connection to properties (resolution?)
- references to code/COSY?

\subsection{Considerations}
\subsection{Resolution}
