
\chapter{INTRODUCTION}
\label{ch:introduction}

The elements making up the universe were formed during a variety of
processes, beginning with Big Bang Nucleosynthesis (BBN) that formed the
lightest elements. Those elements common to life on Earth were primarily
formed through burning processes inside of stars, grouped together under
the title of stellar nucleosynthesis. Depending on the conditions within
the stellar environment, which are characterized by macroscopic
qualities about the star (temperature, pressure, mass, etc.) and the
elemental composition of the stellar interior where the burning process
takes place, the reactions accessible to the nuclei within the star
differ. The creation and destruction of different elements and isotopes
may be inhibited or enhanced by these differing conditions, and the
study of these processes at the nuclear level has spawned the field of
\emph{nuclear astrophysics} in order to understand the inner workings of
these stars. Nuclear astrophysics can be described as the specific and
directed study of those nuclear reactions that have an effect on the
properties or life cycle of celestial bodies. These reactions may take
place during the lifecycle of a star, during a cataclysmic event within
the universe such as a black hole merger or gamma ray burst, or at the
beginning of the universe itself. Additionally, the decay of various
isotopes can also play a major role within this domain, either as part
of a sequence of reactions or independently. The entire field of nuclear
astrophysics was conceived in the seminal papers \cite{B2FH} and
\cite{Cameron1957}, which have been used as a basis for much of the work
in the following decades.

For astrophysical reactions, the properties of the environment can play
a major role in how quickly the reaction proceeds or if it is even
energetically allowed. Cross sections for many of these reactions
rapidly decrease at lower energies, requiring increasingly sophisticated
detection methods in order for the energy dependence of the cross
section to be determined. Due to the decreasing cross sections, work has
primarily focused on strong resonances in this low-energy regime.


\section{Stellar Burning}

Stars in hydrostatic equilibrium can produce energy through a number of
different reaction channels based on the mass, temperature, and isotopic
enrichment of the stellar interior where the burning process takes
place. The net result of stellar burning is the fusion of lighter
isotopes into heavier isotopes and the release of energy. A single star
may undergo multiple distinct burning stages during its lifecycle, with
each subsequent burning stage occurring at progressively hotter
temperatures until a point at which the energy produced cannot maintain
hydrostatic equilibrium.

The isotopic abundances in the stellar interior at a given point in time
are based on the initial abundances within the interstellar medium when
the cloud condensed into the star, and the reaction channels that are
available in the stellar interior during the lifespan of the star.

\subsection{Low-Mass Hydrogen Burning}

The fusion of four \nuc{1}{H} nuclei into a single \nuc{4}{He} nucleus
is called \emph{hydrogen burning}, and can take place through a few
reaction sequences within different temperature ranges. The overall
outcome of the reaction chain given by
\[
    4\mnuc{1}{H} \rightarrow{} \mnuc{4}{He} + 2e^+ + 2\nu_e
\]
is the conversion of \nuc{1}{H} into \nuc{4}{He} and the release of
approximately 26.7~MeV in energy from that fusion. The underlying
processes that are undergone within stars under steady state conditions
are grouped under \emph{hydrostatic hydrogen burning}. Common
environments for such burning processes are core hydrogen burning in
stars similar to our sun in mass and metallicity, and in hydrogen
burning shells within asymptotic giant branch (AGB) stars. The
differences in the isotopic composition and temperature of the stellar
interior can allow for different processes to take place.

Stars similar to our sun fuse hydrogen through the proton-proton ($pp$)
chains, which are described by the reaction sequences
\begin{align*}
    \rm{PP-1:\,}& \mreact{\rm{p}}{\rm{p}}{e^+\nu_e}{
        \mreact{\mnuc{2}{H}}{\rm{p}}{\gamma}{
        \mreact{\mnuc{3}{He}}{\mnuc{3}{He}}{2\rm{p}}{\mnuc{4}{He}}}} \\
    \rm{PP-2:\,}& \mreact{\mnuc{3}{He}}{\alpha}{\gamma}{\mnuc{7}{Be}}
        \left(e^-,\bar{\nu}_e\right)\mreact{\mnuc{7}{Li}}{\rm{p}}{\alpha}{\mnuc{4}{He}} \\
    \rm{PP-3:\,}& \mreact{\mnuc{7}{Be}}{\rm{p}}{\gamma}{\mnuc{8}{B}}
        \left(\beta^+\right)\mnuc{8}{Be}\left(\alpha\right)\mnuc{4}{He},
\end{align*}
where the PP-2 and PP-3 chains are branches off from the PP-1 and PP-2
chains, respectively, at the point after the initial nuclei is created
within the chain. Typical temperatures for this burning process are on
the order of $T_6 \approx 8-55$, which the core temperature of our sun
($T_6 = 15.6$) falls squarely within~\cite{Iliadis}.

% Look up additional sources from within

The Carbon-Nitrogen-Oxygen (CNO) cycle is an additional pathway for
stable hydrogen burning accessible when the stellar interior has been
enriched with heavier nuclei. The relative abundances of the catalytic
elements C, N, O, and F will change based on the relative reaction
rates. The CNO cycles are described by the cyclic reaction sequences

\begin{align*}
    \rm{CNO-1:\,}& \mreact{\mnuc{12}{C}}{\rm{p}}{\gamma}{\mnuc{13}{N}}
        \left(\beta^+\right)\mreact{\mnuc{13}{C}}{\rm{p}}{\gamma}{
            \mreact{\mnuc{14}{N}}{\rm{p}}{\gamma}{\mnuc{15}{O}}}
        \left(\beta^+\right)\mreact{\mnuc{15}{N}}{\rm{p}}{\alpha}{\mnuc{12}{C}} \\
    \rm{CNO-2:\,}& \mreact{\mnuc{14}{N}}{\rm{p}}{\gamma}{\mnuc{15}{O}}
        \left(\beta^+\right)\mreact{\mnuc{15}{N}}{\rm{p}}{\gamma}{
        \mreact{\mnuc{16}{O}}{\rm{p}}{\gamma}{\mnuc{17}{F}}}\left(\beta^+\right)
        \mreact{\mnuc{17}{O}}{\rm{p}}{\alpha}{\mnuc{14}{N}} \\
    \rm{CNO-3:\,}& \mreact{\mnuc{15}{N}}{\rm{p}}{\gamma}{
        \mreact{\mnuc{16}{O}}{\rm{p}}{\gamma}{\mnuc{17}{F}}}\left(\beta^+\right)
        \mreact{\mnuc{17}{O}}{\rm{p}}{\gamma}{\mnuc{18}{F}}\left(\beta^+\right)
        \mreact{\mnuc{18}{O}}{\rm{p}}{\alpha}{\mnuc{15}{N}} \\
    \rm{CNO-4:\,}& \mreact{\mnuc{16}{O}}{\rm{p}}{\gamma}{\mnuc{17}{F}}\left(\beta^+\right)
        \mreact{\mnuc{17}{O}}{\rm{p}}{\gamma}{\mnuc{18}{F}}\left(\beta^+\right)
        \mreact{\mnuc{18}{O}}{\rm{p}}{\gamma}{
            \mreact{\mnuc{19}{F}}{\rm{p}}{\alpha}{\mnuc{16}{O}}}
\end{align*}

For the CNO cycles, the temperature of the interior of the star and the
individual reaction rates in question determine which cycle dominates as
well as which catalytic isotope will, in steady state conditions, have
the highest fractional abundance. These differing abundances based on
the properties of the star can have large consequences on the burning
phases at higher mass and temperature due to the isotopic enrichment
available.

As the burning progresses, an inert core primarily consisting of
\nuc{4}{He} is produced. Due to the lower temperatures in the core at
this stage in stellar evolution, no He burning takes place. This inert
core may ignite during the red giant stage of stellar evolution, at
which point the He within the outer shell of this inert core ignites and
helium burning becomes an open channel for stellar energy production.

\subsection{Helium Burning}

The primary nucleosynthesis products of Helium burning are \nuc{12}{C}
and \nuc{16}{O}, produced through the \emph{triple-$\alpha$ process}
given by \react{\mnuc{4}{He}}{\alpha\alpha}{\gamma}{\mnuc{12}{C}} and
\react{\mnuc{12}{C}}{\alpha}{\gamma}{\mnuc{16}{O}},
respectively~\cite{Aliotta2016}. The relative abundance ration C/O
between these two isotopes greatly affects the subsequent evolution of
the star at the end of the He burning phase. He burning progresses first
through the triple-$\alpha$ process until such a time that enough
\nuc{12}{C} has built up within the He core, at which point the
production of \nuc{16}{O} can begin~\cite{Aliotta2016}.

He burning may take place in varied stellar environments. Following the
end of H burning within main sequence stars, the stellar interior
compresses since not enough energy is being produced. This compression
increases the pressure and temperature near the stellar core to the
point that He burning becomes energetically
favorable~\cite{CarrollOstlie}. The star expands, transitioning from a
main sequence star to a red giant star, and maintains hydrostatic
equilibrium due to the increased outward pressure produced by He
burning.

Asymptotic Giant Branch (AGB) stars are a phase of stellar evolution
following the turnoff from the main sequence after initial H burning
completes~\cite{CarrollOstlie}. The stellar interior contains two
distinct burning regions: H and He. These regions alternate in which is
active as the reaction sequences progress within each
region~\cite{Aliotta2016}. During the early AGB phase, the H burning
shell is nearly inert until the point when mixing occurs between the H-
and He-rich regions near the center of the star~\cite{CarrollOstlie}.
The star then enters the thermal pulse AGB phase, defined by a
reactivated H burning region and periodic He shell flashes caused by the
rapid introduction of He from the H burning shell onto the top layer of
the He burning shell. The flash drives the H burning shell outward,
cooling it and reducing the burning rate, which in turn ends the shell
flash. As the He burning subsides, the H burning shell can reactivate,
restarting the cycle~\cite{CarrollOstlie}. Temperatures required for
this cyclic process are in the $T_6 = 20 - 60$ temperature range, where
temperatures at the higher end create a more-efficient environment for
the CNO reactions~\cite{Boeltzig2016}.

During these shell flashes, additional He burning reactions activate.
Two such reactions, \react{\mnuc{13}{C}}{\alpha}{\rm n}{\mnuc{16}{O}}
and \react{\mnuc{22}{Ne}}{\alpha}{\rm n}{\mnuc{25}{Mg}}, are the neutron
sources for the strong and weak $s$-process,
respectively~\cite{Aliotta2016}. The strong $s$-process is responsible
for approximately half of the cosmic abundances for isotopes with mass
$A \geq 90$, while the weak $s$-process contributes to the isotopic
abundances for isotopes with mass within the range $60 \leq A \leq
90$~\cite{Aliotta2016}.

\subsection{Additional Burning Processes}

While the previously-discussed reaction sequences play a large role in
the energy production of stars, additional burning sequences are
available for more massive and hotter stars. These burning cycles can be
thought of in similar terms to the CNO cycles, where baseline processes
convert $4\mnuc{1}{H}\rightarrow\mnuc{4}{He}$ and additional reactions
provide a ``breakout'' channel to higher burning processes. The reaction
\alpa{} is the final step in the MgAl burning cycle (shown in
Figure~\ref{fig:mgal}) that provides for the cycling of the catalytic
nuclei. These burning cycles activate at elevated temperatures in the
range of $T_6 = 60 - 100$ within massive AGB stars~\cite{Boeltzig2016}.

The \alpa{} reaction is relatively well-understood due to the
availability of target and beam material (see, for example,
\cite{Nelson1984}). The known characteristics of the reaction across a
range of center of mass energies make it a useful test reaction for new
experimental facilities. The remainder of the discussion will focus on
\alpa{} as an example.

\begin{figure}[t]
    \begin{center}
        \centerline{\includegraphics[width=0.95\textwidth]%
            {figures/cno_nena_mgal.pdf}}
        \caption[Schematic of burning
            cycles]{Schematic of burning cycles in the $12 \leq A \leq
            27$ mass range. These catalytic cycles play a large role in
            various abundance measurements and burning processes in red
            giant stars. The reaction in question is indicated. Figure
            adapted from \cite{Boeltzig2016}.}
        \label{fig:mgal}
    \end{center}
\end{figure}


\section{Capture Reactions}
\label{sec:01-capture-reactions}

Most astrophysical reactions in the low mass regime can be considered as
capture reactions, where a lighter particle with mass $m_a$ combines
with a heavier particle with mass $m_A$, forming a compound nucleus with
mass $m_C = m_a + m_A$. This compound nucleus is produced within an
excited state, and will decay through particle emission, resulting in
two particles $b$ and $B$. The overall reaction can be written as
\[
    \mreact{A}{a}{b}{B},
\]
where $A$ and $B$ denote our heavy particles and $a$ and $b$ denote our
lighter particles. When discussing reactions, we commonly consider the
lighter particle to be impinging on the heavier particle in the entrance
channel $a + A$. The compound nucleus $C$ is not explicitly denoted in
this equation. The reaction \alpa{} can be considered as this type of
compound reaction, where a compound nucleus \nuc{28}{Si} is created by
the direct capture of the proton by \nuc{27}{Al}, followed by the
emission of the $\alpha$ particle, leaving behind \nuc{24}{Mg}. If the
light particle $b$ is a $\gamma$ ray, the reaction is a \emph{radiative
capture reaction}, written as \react{A}{a}{\gamma}{B}. Both types of
reactions can be studied and understood in similar ways.

The amount of energy either released in the reaction or required for the
reaction to take place is called the reaction $Q$-value. The $Q$-value
is defined as the energy difference between the entrance and exit
channels, or
\[
    Q = (m_A + m_a - m_B - m_b)c^2,
\]
where $m_i$ is the mass of the denoted particle. Positive $Q$-values are
exothermic reactions, where the particles produced in the exit channel
carry away the excess energy, while negative $Q$-values are endothermic
reactions, where a minimum amount of energy in the entrance channel is
required in order for the reaction to take place.

For reactions that go through a compound nucleus, the energy states
accessible in the reaction are based on the beam energy and the reaction
$Q$-value. The excitation energy in the compound nucleus is given as
\[
    E_x = Q + E_{\rm CM} = Q + E_a\frac{m_A}{m_a + m_A},
\]
where $E_{\rm CM}$ is the center of mass energy. Unless explicitly
denoted as a different energy, in the following description
$E = E_{\rm CM}$.


\subsection{Cross Section and $S$-factor}

The probability that a reaction will occur is the reaction's \emph{cross
section}. The cross section can be defined as the ratio between the
number of reactions that actually took place and the total numbers of
incident target particles within some time frame, or
\[
    \sigma = \frac{N_{\rm{reactions}} / t}{N_{\rm{beam}} / t N_{\rm{target}} / A},
\]
where $A$ is the overlapping area between the target and the incident
beam of particles~\cite{Iliadis}. The above equation is implicitly
energy dependent, as the number of reactions may change with changing
beam energy. Additionally, the reaction products may be emitted through
different solid angles at different rates. In these cases, the
differential cross section $\rm{d}\sigma/\rm{d}\Omega$ is measured. The
differential cross section can be integrated over the complete solid
angle to obtain the total cross section
\[
    \sigma = \int \left(\frac{\rm{d}\sigma}{\rm{d}\Omega}\right)\,\rm{d}\Omega.
\]
The cross section is the primary quantity of interest when studying
nuclear reactions. The cross section for the \alpa{} reaction near the
energy range of interest for this experiment is shown in
Fig.~\ref{fig:alpa-cross-section}, where the structure will be described
later.

\begin{figure}[t]
    \begin{center}
        \centerline{\includegraphics[width=0.90\textwidth]%
            {figures/zero_degree_pa0.pdf}}
        \caption[Cross section for the \alpa{} reaction]{Cross section
            for the \alpa{} reaction around the energy range of interest
            for the experiment. Shown is the differential cross section
            at $\theta_{\rm{lab}} = 0\degree$, as St.\ George is
            designed to study reactions at 0\degree. From
            \cite{deBoer2017}, produced by AZURE2~\cite{AZURE2}}
        \label{fig:alpa-cross-section}
    \end{center}
\end{figure}

The cross section for charged particle reactions drops off precipitously
by orders of magnitude as the energy decreases due to repulsion from the
Coulomb barrier. To aide in visualization of the cross section, we can
remove the contribution to the cross section of the Coulomb repulsion,
where our resulting energy-dependent component is called the
\emph{astrophysical $S$-factor}. The $S$-factor is defined in reference
to the energy-dependent cross section $\sigma(E)$ as
\[
    \sigma(E) \equiv S(E)e^{-2\pi\eta} / E,
\]
where $\eta = Z_1Z_2e^2\sqrt{\mu_{01} / 2E\hbar^2}$ is the Sommerfeld
parameter. The term $\mu_{01}$ is the reduced mass of the system, given
as $\mu_{01} = m_0m_1/(m_0 + m_1)$. The full Gamow factor
$\rm{exp}(-2\pi\eta)$ is the probability that an s-wave nuclei
penetrates the Coulomb barrier and is the primary cause of the strong
decrease in the cross section at lower energies.

\subsection{Reaction Rate}

Under astrophysical conditions, instead of focusing on the probability
of a single reaction taking place, it is common to discuss the number of
reactions that take place under given conditions, primarily the interior
temperature of the burning process. We can define a quantity called the
\emph{reaction rate} that describes the number of reactions within a
given volume per unit time in terms of the energy-dependent cross
section $\sigma(E)$ as
\begin{equation}
\label{eq:reaction-rate}
    N_A\langle\sigma\nu\rangle_{01} = N_A\left(\frac{8}{\pi\mu_{01}}\right)^{1/2}
        \frac{1}{(kT)^{3/2}}\int_0^{\infty} E\sigma(E)e^{-E/kT}\,\rm{d}E,
\end{equation}
where the term $\rm{exp}(-E/kT)$ originates from the Maxwell-Boltzmann
distribution~\cite{Iliadis}. Based on the knowledge of the temperature
of the stellar interior, the reaction rate can be calculated for any
reaction, given the cross section.

For those stellar interiors, there are some energies that will dominate
the overall determination of the reaction rate. The contribution to the
reaction rate will be dominated by energies where the product of the two
exponential terms, $\rm{exp}(-E/kT)$ and $\rm{exp}(-2\pi\eta)$, is near
their maximum. This distribution can be approximated by a gaussian peak,
with the central energy called the \emph{Gamow energy}, given by
\[
    E_{\rm{Gamow}} = \left[\left(\frac{\pi}{\hbar}\right)^2
        \left(Z_0Z_1e^2\right)^2\left(\frac{\mu_{01}}{2}\right)
        \left(kT\right)^2\right]^{1/3},
\]
and the width of the peak describing the \emph{Gamow window} given by
\[
    \Delta_{\rm{Gamow}} = \frac{4}{\sqrt{3}}\sqrt{E_{\rm{Gamow}}kT},
\]
where $Z_i$ is the atomic number of the particle in
question~\cite{Iliadis}. This energy window has been the focus of low
energy nuclear astrophysics and improves the extrapolation of the
$S$-factor to lower energies. When there are no resonances within the
energy range, the extrapolation to these lower energies is much easier,
as the energy dependence of the cross section is more well-behaved.

\subsection{Resonances}

In the absence of nuclear structure, the above description encompasses
what is necessary to determine how the nuclear reaction progresses in
stellar interiors. For the \alpa{} reaction, the compound nucleus
\nuc{28}{Si} has a number of energy levels that can affect the yield
of the reaction, with some of those levels shown schematically in
Fig.~\ref{fig:alpa}.

\begin{figure}[t]
    \begin{center}
        \centerline{\includegraphics[width=0.55\textwidth]%
            {figures/reaction_energy_diagram.png}}
        \caption[Energy diagram for the \alpa{} reaction]{Energy diagram
            for the \alpa{} reaction, showing two resonances accessible
            for study at zero degrees.}
        \label{fig:alpa}
    \end{center}
\end{figure}

In cases where the center of mass energy of the system is close to one
of these energy levels, the reaction is said to be through a
\emph{resonance} or a resonant reaction. Resonances can be observed in
the cross section (see Fig.~\ref{fig:alpa-cross-section}) where there is
a large increase in the cross section over a narrow energy range.
Resonances dominate the overall reaction due to their higher cross
section and can contribute massively to the reaction rate even in cases
where the resonance falls outside of the Gamow window. In cases where
resonances dominate, the reaction rate can be simplified to only
consider the contributions from the resonances as
\[
    N_A\langle\sigma\nu\rangle = N_A\left(\frac{2\pi}{\mu_{01}kT}\right)^{3/2}
        \hbar^2e^{-E_r/kT}\omega\gamma,
\]
where $E_r$ is the resonance energy and $\omega\gamma$ is the resonance
strength~\cite{Iliadis}. In cases where multiple resonances contribute
to the reaction rate, the resonance-dependent term is instead a
summation over the resonances, or
\[
    \sum_i e^{-E_{r,i}/kT}\omega\gamma_i.
\]
Resonances are studied both to understand what reactions are important
and in what way in the interior of stars and to determine the properties
of the energy levels within the compound nucleus, and one can be used to
inform the properties of the other.

\subsection{Angular Properties}

The properties of the cross section, and thus the underlying energy
levels within the compound nucleus, depend on the properties of the
energy levels, specifically the spin-parity of the nuclear state. The
spin and parity of the nuclear states available for resonant reactions
are determined by the quantum mechanical selection rules from the spin,
parity, and angular momentum of the target and projectile nuclei, given
as
\[
    \left|\ell_{\rm{p}} - J_{\rm{p}} - J_{\mnuc{27}{Al}}\right|
    \leq J_{\mnuc{28}{Si}} \leq
    \left|\ell_{\rm{p}} + J_{\rm{p}} + J_{\mnuc{27}{Al}}\right|
\]
and
\[
    \pi_{\mnuc{28}{Si}} = \pi_{\rm{p}}\pi_{\mnuc{27}{Al}}(-1)^{\ell_{\rm{p}}},
\]
where the properties of the compound nuclear state in \nuc{28}{Si} are
$J_{\mnuc{28}{Si}}$ and $\pi_{\mnuc{28}{Si}}$, and $\ell_{\rm{p}}$ is
the orbital angular momentum of the incident proton. Given that the
ground state of \nuc{27}{Al} is $5/2+$ and the spin-parity of protons is
$1/2+$, we have an entrance channel spin of $j_s = 2$ or 3, where
$j_s = j_{\rm{p}} + j_{\mnuc{27}{Al}}$.

For the remainder of the discussion, we will focus on the two resonances
that are studied as part of this dissertation, with the properties shown
in Table~\ref{tab:resonance-properties}.

\begin{table}
    \begin{center}
        \caption{RESONANCE PARAMETERS}
        \begin{tabular}{ccc}
            \toprule
            \midrule
                \textbf{Property, Units} & \textbf{Low} & \textbf{High} \\
            \midrule
                $E_{x}$, MeV \cite{NNDC}        & 12.726 & 12.902 \\
                $E_{p}$, MeV                    & 1.182 & 1.363 \\
                $J^{\pi}$ \cite{Nelson1984}     & \multicolumn{2}{c}{2+} \\
                $W(\theta)$ \cite{Andersen1961} & \multicolumn{2}{c}{1} \\
                $Q$, MeV                        & \multicolumn{2}{c}{1.601} \\
            \bottomrule
        \end{tabular}
        \label{tab:resonance-properties}
    \end{center}
\end{table}

Our allowed values for $\ell_{\rm{p}}$ are 0, 2, and 4 due to parity,
where $\ell_{\rm{p}} = 0$ only contributes in the $j_s = 2$ case. The
entrance channel thus is a mixture of different components, based on the
allowed angular momenta.

In the exit channel $\mnuc{24}{Mg} + \alpha$, we can perform similar
calculations. Since both the ground state of \nuc{24}{Mg} and the
spin-parity of $\alpha$ are $0+$, we are restricted to natural parity
states within \nuc{28}{Si} ($J^{\pi} = 0+$,~$1-$,~$2+$,~$\ldots$), which
includes our resonances. Since our resonances are $J^{\pi} = 2+$, the
only allowed orbital angular momentum for the $\alpha$ particle is
$\ell_{\alpha} = 2$, which means that there is no mixing in the exit
channel.

Since we are observing our reaction products within a small solid angle,
we need to consider the angular correlation of the produced $\alpha$
particles to the incident proton beam. We can relate the differential
cross section to the angular correlation $W(\theta)$ with
\[
    \left(\frac{\rm{d}\sigma}{\rm{d}\Omega}\right)_{\theta} =
        \frac{1}{4\pi}\sigma W(\theta),
\]
where the cross section is for the energy in question~\cite{Iliadis}.
The two resonances in question have been shown to be isotropic, so
$W(\theta) = 1$~\cite{Andersen1961}. For those resonances that are not
isotropic, the measurement of particle yields at multiple angles can be
used to determine the angular correlation of the produced particles and
can inform the assignment of state spins and parities (see for example
\cite{deBoer2015}).


\section{Recoil Separation}
\label{sec:ch01-recoil-separation}

The experimental process in which the reaction products produced by a
direct beam can be filtered out from that direct beam in order to be
detected is called \emph{recoil separation} or alternatively
\emph{recoil mass separation}. A recoil separator is the system,
consisting of a sequence of electromagnetic elements, designed to
perform this task.

\subsection{Motivation}
\label{ssec:recoil-separation-motivation}

Recoil mass separation was conceived as an alternate way to measure the
cross sections of radiative capture reactions. These reactions had
previously been studied by detecting the produced $\gamma$ rays, which
are potentially subject to high background count rates near the energy
of interest and low detection efficiency for the produced $\gamma$ rays.
The heavy reaction product can instead be detected by a detector
situated behind the target, assuming that the target is thin enough to
allow the produced recoils to leave the target. In this thin target
case, the incident beam will likely pass through the target as well,
making it a source of background at the detector plane. In the cases of
interest for nuclear astrophysics, this background count rate could be
$\times 10^{15}$ that of the particles of interest and will damage the
detector.

The produced recoils may be filtered out from the incident beam by
electromagnetic elements situated between the target and the detector.
For example, if we are looking at radiative alpha capture reactions
\react{A}{\alpha}{\gamma}{B}, the interaction between the heavy incident
beam with mass $A$ and linear momentum $p$ and the $\alpha$ particles
within the target produces a heavy compound nucleus with mass $A + 4$
and momentum $p$. Ignoring the effect of the emitted $\gamma$ ray on the
momentum and assuming that there is no spread in the momentum, the use
of electrostatic elements can separate the recoils from the beam based
on their different magnetic and electric rigidities. The magnetic
rigidity is defined as
\begin{equation}
    \label{eq:brho}
    B\rho = \frac{p}{q} = \frac{\sqrt{2mT}}{q}\textrm{ [Tm],}
\end{equation}
where $p$, $q$, $m$, and $T$ are the momentum, charge state, mass, and
kinetic energy of the desired particle, respectively. The magnetic
rigidity defines the trajectory of the particle's movement within a
homogeneous magnetic field of strength $B$ along a circular path with
radius $\rho$. Similarly, the electric rigidity is defined as
\begin{equation}
    \label{eq:erho}
    E\rho = \frac{pv}{q} = \frac{2T}{q}\textrm{ [MV]}
\end{equation}
with the same variable definitions as before, and defines the circular
trajectory a particle takes within an electric field of strength $E$.

With a single momentum and velocity (or kinetic energy) selected for,
the recoil particles of interest can be uniquely identified by the
optical system. The design of recoil separators make use of this
relatively simple idea as the basis of their design. Despite this, there
have been relatively few recoil separators that have been brought into
service due to the complexities of their design and operation that are
not adequately taken into account in this description.

\subsubsection{Inverse Kinematics}
Reactions may be studied in two ``configurations'', as defined by which
particles are the target and the projectile: forward and reverse
kinematics. We will discuss each of these options in turn.

In forward kinematics, a beam composed of relatively lighter nuclei is
directed onto a target made up of relatively heavy nuclei, with the
produced heavy recoil staying primarily within the target and the light
ejectile being the particle to be detected by the detector system. We
will write this reaction as \react{A}{a}{b}{B}, where $A$ and $B$ are
the heavy particles. For radiative capture reactions, the light ejectile
$b$ is a $\gamma$ ray which must be detected by a Ge detector which has
a relatively low efficiency and is subject to background. In cases where
the emitted $\gamma$ has energy similar to exceedingly strong background
lines, the detection of the produced $\gamma$ can be almost impossible.
As many reactions of astrophysical interest are radiative capture
($({\rm p},\gamma)$ or $(\alpha,\gamma)$), these complications can
prevent the study of some important reactions. Contaminating nuclei
within the target can have the same effect, where a reaction channel
with the contaminant nuclei has the same $\gamma$ energy as the desired
reaction, preventing the direct detection of the desired reaction.

Performing a reaction in reverse kinematics can help avoid some of these
problems. In reverse kinematics, a heavy projectile is impinged on a
target made up of lighter nuclei, and the heavy recoil is detected by
the detection system, or \react{a}{A}{B}{b}. For radiative capture
reactions, the produced $\gamma$ may also be detected in coincidence
with the primary detection system to further reduce background, if
desired. Since the heavy recoil is detected, a high-efficiency particle
detector, such as a solid state detector, can be used. Since reactions
of astrophysical interest in the energy regions of astrophysical
interest are commonly lower in cross section, the ability to detect the
infrequently-produced heavy recoils with high efficiency can make the
study of that particular reaction experimentally feasible. For reactions
of astrophysical interest, the target becomes a \nuc{1}{H} or
\nuc{4}{He} (commonly) gas cell or jet. While these targets have their
own complications, they can also be made incredibly isotopically pure,
reducing additional background from contaminants within the target.

A primary source of background for reactions studied in reverse
kinematics is beam-induced background. The emitted recoils are emitted
within a small solid angle cone oriented along the incident beam
direction, co-linearly with the unreacted beam that passes through the
thin, light nuclei target. In order to detect those produced recoils,
the must be separated or filtered out from the unreacted beams, which
can be accomplished through an electromagnetic separator such as a
recoil mass separator.

\subsubsection{Radiative Capture}
When studying radiative capture reactions \react{A}{a}{\gamma}{B}, the
emitted $\gamma$ arises from the decay of the excited state within the
compound nucleus $B$ as it is emitted. Radiative capture reactions
produce a compound nucleus in an excited state that has the same linear
momentum as the beam in the lab frame, given as
\[
    p^*_{\rm{recoil}} = \sqrt{2m_aE_a},
\]
where $*$ denotes that the recoil is in an excited state. When this
state de-excites, it produces a $\gamma$ ray, either due to a single
transition to the ground state or a cascade of $\gamma$s through
intermediate states. As $B$ is produced in an excited state, the
$\gamma$ may be emitted in any direction, which affects the final
momentum of the recoil.

If the $\gamma$ is emitted along the $z$-axis (the incident beam
direction), the momentum of the final recoil is given by
\[
    p_{{\rm recoil}} = \sqrt{2m_aE_a} \pm p_{\gamma},
\]
where $p_{\gamma} = E_{\gamma}/c$. In this case, no angular change is
imparted to the recoil, so it can be assumed to still be traveling along
the beam axis. At the opposite end, if the $\gamma$ is emitted
perpendicular to the beam axis, there is a maximal angular change, where
the lab angle of the recoil is given by
\[
    \theta_{{\rm recoil}} = {\rm atan}\left(\frac{p_{\gamma}}{\sqrt{2m_aE_a}}\right).
\]
There is a small change to the total momentum of the recoil from the
imparted transverse momentum from the emitted $\gamma$. In cases where
the $\gamma$ is emitted somewhere between these two extremes, there will
be a smaller angular change imparted than the transverse case and a
smaller momentum change than in the longitudinal case.

In this case, the emitted recoils are confined within a forward cone
with some opening angle and with some momentum spread. The various
properties of the separator will be based on these kinematic properties
for the reactions of interest. For those reactions that have smaller
acceptance cones or energy spreads than what the separator was design
for, studying those reactions becomes easier. The final design for the
recoil separator will take into account both the central value of the
recoil momentum distribution as well as the limits for the angular and
momentum distribution. We can consider the momentum and energy
distributions as essentially interchangeable, since the mass of the
desired recoil is known.

\subsection{Beam Optics}
\label{sec:ch1-beam-optics}

The understanding of how recoil separators work is grounded in the
theory of beam optics which describe the effect of electric and magnetic
fields on moving charged particles. These moving particles are focused
and directed by the electromagnetic elements, and their action on
particles can be modeled and optimized to transport particles with
various properties down a beam line to a desired location.

\subsubsection{Mass Dispersion}
A recoil separator is designed to have a mass dispersion at at least one
point within the separator where the incident beam with a different mass
is rejected. This location is commonly before the detector plane to
avoid the potentially high count rates from the high intensity incident
beam.

Mass dispersion can be described by the coefficients within the designed
transport matrix at the point where the mass dispersion should be
located within the separator. We can determine the distance in the
dispersive $x$ plane between the incident beam and the desired heavy
recoils to first order as
\[
    x = (x|x)x_0 + (x|\theta)\theta_0 + (x|\delta_E)\delta_E + (x|\delta_M)\delta_M,
\]
where $\delta_E = \Delta E/E_0$ and $\delta_M = \Delta M/M_0$ are the
energy and mass dispersions, respectively~\cite{Davids2003}. The matrix
elements can be understood as their effect on the final position of the
particle in question in relation to its position at the target location,
where the coefficient in question either represents a focusing or a
divergence along the horizontal axis. Each coefficient is the amount of
focus or divergence given the particle trajectory at the target location
and is based on all of the intervening magnetic and electrostatic
elements along the beamline.

The coefficient $(x|x)$ is the magnification of the beam spot, so
independent of any other effects, the spot at the detector plane is
larger or smaller based on the magnitude of the coefficient. Similarly,
the coefficient $(x|\theta)$ describes the angular dependence of the
horizontal beam spot. Since we want a focus in the $x$ plane, we
require that this coefficient be zero or as close to zero as possible.
The coefficient $(x|\delta_E)$ is the energy dispersion, so the change
in beam spot size due to the difference in energy from the central tuned
energy for the separator. St.\ George is designed to have a $\pm7.5$\,\%
energy acceptance, so this value should be zero or close to zero for a
range of values. For alternate tunes of the separator, this requirement
may be relaxed. Finally, the coefficient $(x|\delta_M)$ is the mass
dispersion, which describes the change in horizontal position based on
the difference in mass from the central tuned particle. As St.\ George
is designed to provide a mass separation following a reaction, this
coefficient should be non-zero and preferably large in order to
separate out the incident beam particles from the produced reaction
particles.

To higher orders, the design of the transport matrix and through that
the decision on what electromagnetic elements and their properties for
the physical recoil separator need to be designed by some computational
codes. Within the physical separator, at least two of the available
dispersive elements magnetic dipole, electric dipole, or Wien filter
must be used to achieve a dispersion at a focal plane of
mass/charge~\cite{Davids2003}. In order to leave that dispersion at the
focal plane, either magnetic and electric dipoles or a magnetic dipole
and Wien filter must be used. While a single element of each type is the
minimum requirement, physical separators commonly have multiple elements
due to the physical nature of the beam envelope within the beam line and
the restrictions of the physical laboratory space.

\subsubsection{Beam Suppression}
Recoil separators must achieve a certain level of beam-induced
background suppression, henceforth \emph{beam suppression}, in order to
observe the relatively uncommon produced recoils from the reaction. This
beam suppression can be achieved by the separator and detector system
combined, with a majority of the beam suppression obtained by the
separator. We will define beam suppression as
\[
    S = \frac{N_{b,\rm{incident}}}{N_{b,\rm{target}}},
\]
or the number of incident beam particles that will lead to a single beam
particle detection at the detector system. Beam suppression is linked to
the reaction in question, the rigidities of the beam and recoil, the
energies of interest, and the detector system used. Values of $S$ for
existing separator systems will be cited, with larger values more
successful at rejecting beam and more able to measure cross sections
with lower yields.

\subsection{Recoil Separator Facilities}
\label{sec:prevwork}

The use of recoil separators to study radiative capture reactions has
been explored recently at a number of facilities. The design of St.
George is based on the knowledge gained from the design, construction,
and operation of these previous recoil separator systems. The entire
system, inclusive of the beam source, target, and detector, must be
discussed as a whole when evaluating the capabilities of a given
separator.

Recoil separators have been used for a variety of physics programs, such
as heavy element searches, recoil-gamma spectroscopy, spin
distributions, accelerator mass spectrometry, and nuclear
astrophysics~\cite{Davids2003}. Since the St.\ George recoil mass
separator was designed to be used primarily for reaction studies focused
on nuclear astrophysics, our discussion will focus on those separators
also aligned with this goal.

\subsubsection{CalTech Separator}
The design and use of recoil separators for nuclear astrophysics
research was pioneered by Smith \textit{et al.}~\cite{Smith1991}. This
separator was a proof-of-concept design to determine the feasibility of
performing reaction studies with this technique. The design of recoil
mass separators specifically designed to study reactions of
astrophysical interest is similar to the design and operating
characteristics of similar recoil separators designed and built around
the same time (see for instance DRS at Daresbury
Laboratory~\cite{James1988}), and many of the decisions made for optimal
characteristics of the separator system have been replicated by more
modern separator systems.

The primary considerations for the design of the CalTech separator are
the study of radiative capture reactions in inverse kinematics using
radioactive isotope beams with near-complete rejection of the incident
beam before the detector system. The separator achieves the mass
separation by using a Wien filter to provide initial velocity selection,
an electrostatic deflector created by extending the Wien filter's
electrostatic plates beyond the magnetic field limits to provide energy
selection, and a magnetic dipole to provide momentum
selection~\cite{Smith1991}. Additional experimental choices used for
this separator system, such as the use of an offset Si detector at the
target location to monitor beam current indirectly or the use of a
$\gamma$ detector in coincidence with the final detector system to
provide additional beam suppression, have been replicated in future
systems. During initial experiments, beam suppression on the order of
$10^{10}$ was observed~\cite{Smith1991}.

\subsubsection{ARES}
The Astrophysics REcoil Separator (ARES) was built at Louvain-la-Neuve
to study $(\rm{p},\gamma)$ and $(\alpha,\gamma)$ reactions using
radioactive incident beams provided by the CYCLONE44
cyclotron~\cite{Angulo2001}. Self-supporting solid targets, containing
the required H or He, were used for the reaction studies. The system is
designed with a single magnetic dipole for charge selection and a Wien
filter for velocity selection, along with multiple magnetic quadrupoles
(one triplet and two doublets) to maintain the transportation of the
produced recoil particles to the detector system. The condensed and
limited size of the separator is based on the constraints of the
experimental hall~\cite{Couder2003}. The detector system consists of a
single $\Delta E - E$ gas telescope which separates out the reaction
products from the remaining incident beam particles. The initial test of
the separator used a stable incident beam to compare to results obtained
by other methods within the lab, and the focus of the initial work was
on low-lying resonances of astrophysical interest.

\subsubsection{DRAGON}
The DRAGON recoil separator at TRIUMF-ISAC was built for the same
reasons as ARES, but differs in the actual construction and usage of the
separator. The separator uses two large magnetic dipoles and two large
electrostatic dipoles for its momentum and energy
selection~\cite{Engel2005}. The primary target is an extended gas
target, which is used with the incident radioactive beams for nuclear
astrophysics studies near the valley of stability.

DRAGON has been successful at measuring reactions with radioactive and
stable beams, such as
\react{\mnuc{20}{Ne}}{\rm{p}}{\gamma}{\mnuc{21}{Na}},
\react{\mnuc{21}{Ne}}{\rm{p}}{\gamma}{\mnuc{22}{Na}}, and
\react{\mnuc{24}{Mg}}{\rm{p}}{\gamma}{\mnuc{25}{Al}}~\cite{Engel2005}.
Due to using primarily radioactive beams, reaction studies at DRAGON
focus on low energy resonances due to the extremely low cross sections
and beam intensities involved. Beam suppression on the order of $10^{8}
- 10^{13}$ were observed, depending on the reaction in
question~\cite{Engel2005}.

\subsubsection{ERNA}
The European Recoil separator for Nuclear Astrophysics (ERNA) at Bochum
was designed to study the
\react{\mnuc{12}{C}}{\alpha}{\gamma}{\mnuc{16}{O}} reaction in inverse
kinematics at energies near the Gamow window~\cite{Rogalla2003}. This
reaction has a large spread in both angle and energy, making the
transport of the produced recoils to the detector plane more difficult.

ERNA uses two Wien filters and a single magnetic dipole to provide the
velocity and momentum separation. The primary target is an extended gas
target with offset Si detectors to measure the elastic scattering yield
to monitor the beam current~\cite{DiLeva2008}. ERNA has been used to
measure \react{\mnuc{12}{C}}{\alpha}{\gamma}{\mnuc{16}{O}},
\react{\mnuc{3}{He}}{\alpha}{\gamma}{\mnuc{7}{Be}}, and other
astrophysically important reactions at low center-of-mass energies
important for nuclear astrophysics. For the
\react{\mnuc{3}{He}}{\alpha}{\gamma}{\mnuc{7}{Be}} reaction, beam
suppression in the range $10^{10} - 10^{12}$ was
observed~\cite{DiLeva2008}.

\subsubsection{St.\ George}
The separator consists of six dipole magnets, eleven quadrupole magnets,
and a Wien filter. The separator was designed to accept recoils with a
maximum energy and angular spread of $\Delta E/E = \pm7.5\%$ and
$\Delta\theta = \pm40$~mrad, respectively, and to provide a mass
separation of $m/\Delta m = 100$ and beam suppression of a factor $\geq
10^{15}$.

Combined with the HIPPO (High-Pressure Point-like target) supersonic gas
jet target, St.\ George will be primarily used to study low energy
$(\alpha,\gamma)$ reactions using stable beams~\cite{Kontos2012}. The
use of high-intensity stable beams allows for measurements of the cross
section outside of regions where resonances dominate.

\subsection{Requirements}

Each recoil separator needs to be properly understood before the
experimental results produced using it can be trusted. The primary
properties of the separator, namely the energy and angular acceptances
and the beam suppression, must be verified using a variety of techniques
and test beams in order to determine the expected performance of the
separator. These tests should be performed in the energy range of
interest with beams consisting of nuclei with similar properties, such
as the mass and charge state, as those that will be produced during
experimental campaigns. Once the properties of the separator have been
understood, quantified, and hopefully consistent with the design
parameters, the experimental results obtained from the separator can be
adequately related to the astrophysical process that includes the
reaction in question. Since some of these reaction can only be studied
to acceptable precision using recoil mass separators, achieving that
high precision is a long and involved task.

To that effect, the commissioning and experimental results produced with
St.\ George during the initial phases of its use for studying reactions
of astrophysical interest will be presented within. The commissioning
work discussed covers the energy acceptance up to the final detector
plane, and the angular acceptance and the beam suppression up to the
focal plane after the Wien filter. The first reaction study using the
partially commissioned St.\ George will be discussed. The initial
experiment is a study of the \alpa{} reaction at two prominent
resonances. Finally, the plan for improving the use of St.\ George for
non-standard reactions, namely those that are not $(\alpha,\gamma)$
reactions, will be laid out.
