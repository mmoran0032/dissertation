\chapter{DISCUSSION AND CONCLUSION}
\label{ch:discussion-and-conclusion}

The experiment was designed to experimentally confirm an aspect of the
acceptance for St.\ George, specifically the angular acceptance at small
energy deviations, using a well-known reaction. Additionally, the
experiment aimed to allow for an additional set of reactions to be
studied using the facility. The technical capabilities of the separator
system were shown to be adequate even with sub-optimal characteristics
in the experimental setup, opening up the possibilities of studying low
energy $(\rm{p},\alpha)$ reactions in the future. The angular acceptance at the
peak of the resonances was shown to be consistent with the desired design
characteristics of the separator.


\section{Advantages and Disadvantages}
% from call with Manoel: 2018-05-16
(p,a) can suffer from elastic scattering, but St. George can avoid that by
looking at zero degrees. Ratio of alphas to elastically scattered protons
can have a prohibitivve count rate

Disadvantage: no angular distribution, requires recoil separator
Advantage: can avoid killing detectors, check zero degree


\section{Uncertainties}
\label{sec:uncertainties}

%% NOTES

systematic: alphas hitting interior walls, energy cutoff at detector,
transmission (faraday cup offset?)

Missed counts can be measured, say that target cup roughly restricts to 40 mrad
in all directions, so those additional counts could be from within 40 mrad (I
feel weird about this, so don't directly add it in, but mention the amounts
either here or in potential sources of error)

%% END NOTES

The final uncertainties on the acceptances at each run energy are skewed
distributions. Since basic error propagation relies on the errors being
gaussian distributed, the fact that our uncertainties are not partially
justifies the Bayesian approach described previously. Part of the reason
for the skewed distributions is that the acceptance is bounded by zero
and $\pi/2$, and since our distributions sit closer to the zero end instead
of near the middle of the range somewhat requires that the distribution
be skewed.

\subsection{Statistical}

The uncertainties on most of the inputs are gaussian distributed, as
that represents the statistical nature of the process that creates that
input value. For example, the beam current in gaussian distributed
because...

DISCUSS

The final uncertainty bands for each of the acceptance measurements can
be analyzed by what values affect the range for the uncertainty. We can
limit this discussion to inputs that are controllable by the
experimenter. The final uncertainty will be made up of the uncertainty
from inputs and the uncertainty from those statistical and irreducible
processes. The four inputs that the experimenter can control are the
energy, time, current, and thickness uncertainties. The energy
uncertainty is related to the stability of the accelerator and the
calibration of the analyzing magnet, both of which can be measured and
regulated to the point where the uncertainty can be minimized. The time
uncertainty is based on the total runtime and the interruptions caused
by requiring the stoppage of the beam in order to measure the current
and can be reduced through synchronization of the DAQ with the start of
bombardment, and by minimizing interruptions during the data collection
process. The current uncertainty can be minimized by measuring the
current continuously during the experiment, as there will then be fewer
unknown changes in the beam current and a single value for the beam
current does not need to be applied to the entirety of the experimental
run. Finally, the thickness uncertainty can be minimized by performing
target thickness measurements at multiple energies and with potentially
multiple particles, and by running those measurements for longer such
that the energy loss by the particles can be more accurately determined.

Each of these inputs affects a different part of the final acceptance,
based on how it relates to the experimental and theoretical yield, or
both. We can determine the impact of reducing the uncertainty on each of
these inputs by setting the uncertainty to zero within the analysis
pipeline, which would return a different uncertainty band for the run in
question. Since these uncertainties are not necessarily independent of
each other, we should also look at all combinations of these four inputs
being controlled for to get a full picture of the impartances.
Additionally, the irreducible uncertainty can be determined by keeping
all of the inputs constant. The contribution to the final uncertainty is
expressed as a percent of the total uncertainty band for both the 67\%
and 95\% confidence interval, so the amount of the band that is
accounted for by the inputs that are not held constant.

TABLE_67
TABLE_95

From these tables, we observe a few interesting trends that we can
leverage during follow-up experiments to improve the final uncertainty
of the acceptance and from that the uncertainty on the experimental
yield. The trends are...

DISCUSS

\subsection{Systematic}


\section{Uniformity of Acceptances}
\label{sec:uniformity-of-acceptances}

When calculating the acceptance for St.\ George, it was assumed that the
acceptance cone was described by a single opening angle. In practice,
the horizontal and vertical opening angles may be distinct from each
other. During preliminary experiments for the acceptance of St.\ George,
it required much less fine tuning of magnetic fields to achieve the
maximum vertical acceptance than it was to achieve the maximum
horizontal acceptance. This observation may be due to the lack of
dispersive elements in the vertical plane.

The strips of the Si detector were aligned such that an individual strip
was oriented in the vertical direction, or a particle that is deflected
horizontally would be detected on a different strip (see FIGURE).
This orientation allowed for improved tuning in the horizontal plane
with the lack of sensitivity in the vertical plane. The auxiliary runs
performed where the detector was place in the ``low'' position (where
the top of the detector is located where the bottom of the detector
would be in the regular running position) inform the amount of particles
that are not captured in the vertical plane due to minor mistuning of
the separator, and the auxiliary runs used to center the produced alpha
particle distribution on the detector horizontally inform the amount of
particles that are not captured in the horizontal direction. Ideally, a
detector segmented in both the horizontal and vertical plane would give
a full description of the alpha-particle beam spot density at the
detector plane and could be used to better relate the distribution of
counts at the detector plane to the acceptance cone at the target
location.

In the final configuration of the target system, a series of conical
collimators will be located following the target location to defined the
40 mrad acceptance cone. As the desired configuration of St.\ George is
to measure $(\alpha,\gamma)$ reactions where the heavy recoil particles are
emitted from the target within a cone with an opening angle less than 40~mrad,
NOTES. For experiments similar to this where the ejected
particles are emitted within a cone larger than 40~mrad, these
collimators would ensure that the particles reaching the final detector
must have been emitted within that known acceptance cone. This
restriction would improve the tuning of the separator for similar
experiments, as the emitted particle beam spot at the detector plane can
be more easily tuned to fit completely on the detector.


\section{Potential Sources of Error}
\label{sec:potential-sources-of-error}

The preliminary tunes were determined by keeping a beam with a given
energy and angular deviation from the mean reached the detector plane
within the physical space of the detector as measured on a quartz.


\section{The $(\rm{p},\alpha_1)$ channel}
\label{sec:the-palpha_1-channel}

At the resonances probed, the $(\rm{p},\alpha_1)$ reaction channel is also
open. Measuring the cross section for this reaction at the two desired
resonances is a more difficult experiment due to the lower rigidity of
the produced alpha particles due to the lower energy. The kinematics for
this reaction are given in TABLE.

The lower rigidity is still within the design parameters of St.\ George,
but due to the altered tune required to direct the produced alpha
particles to the detector plane has different rejection properties than
the standard tune. As such, the incident proton beam is close enough in
rigidity that the beam may strike the detector. The beam reduction
levels would not be high enough to avoid damaging the Si detector,
preventing the measurement of the cross section without either
additional rejection capabilities or an improvement in the tune.

MORE DETAILS FROM LOGBOOK


\section{Requirements for Replication and Improvement}
\label{sec:requirements-for-replication-and-improvement}


\section{Next Steps}
\label{sec:next-steps}


\section{Closing Thoughts}
\label{sec:closing-thoughts}
