
\chapter{DISCUSSION}
\label{ch:discussion}

The experiment to measure the \alpa{} cross section at zero degrees was
designed to experimentally confirm the angular acceptance at small
energy deviations of St.\ George. Additionally, the experiment aimed to
allow for an additional set of reactions to be studied using St.\ George
to extend its usage beyond $(\alpha,\gamma)$ reactions of astrophysical
interest. The technical capabilities of the separator system were shown
to be adequate even with sub-optimal characteristics in the experimental
setup, opening up the possibilities of studying low energy $({\rm
p},\alpha)$ reactions in the future. The angular acceptance at the peak
of the resonances was shown to be consistent with the desired design
characteristics of the separator.


\section{Acceptance}

The designed angular acceptance of St.\ George is
$\Delta\theta = 40$~mrad, which is assumed to be cylindrically symmetric
around the beam axis. The acceptance was determined for each run
independently, based on the properties of the beam, target, and cross
section for the \alpa{} reaction. The acceptances are shown in
Fig.~\ref{fig:final-acceptance} and tabulated in
Table~\ref{tab:acceptance-uncertainty}.

\begin{figure}
    \begin{center}
        \label{fig:final-acceptance}
        \centerline{
            \includegraphics[width=0.8\textwidth]{figures/acceptance_uncertainty.png}}
        \caption[Final acceptances]{Final acceptances for St.\ George
            determined by measuring the \alpa{} yield and comparing to
            the expected value from the beam, target, and cross section
            properties. The \textit{maximum a posteriori} acceptance
            values are shown as the black tick marks. The displayed
            uncertainty bounds are the 67\,\% band (dark green) and the
            95\,\% band (light green). The dotted horizontal line is the
            40 mrad designed acceptance for St.\ George.}
    \end{center}
\end{figure}

\begin{landscape}
\begin{table}
    \begin{center}
        \caption{ACCEPTANCE WITH UNCERTAINTY}
        \label{tab:acceptance-uncertainty}
        \begin{tabular}{cS[table-format=2.4]S[table-format=3.2]
        S[table-format=3.2]S[table-format=3.2]S[table-format=3.2]
        S[table-format=3.2]S[table-format=3.2]S[table-format=3.2]}
            \toprule
            \midrule
            \textbf{Run Number} & \textbf{$E_{\rm p}$ [MeV]} &
                \textbf{Acceptance [mrad]} & \textbf{67\,\%$_{\rm L}$} &
                \textbf{67\,\%$_{\rm H}$} & \textbf{67\,\%$_{\rm W}$} &
                \textbf{95\,\%$_{\rm L}$} & \textbf{95\,\%$_{\rm H}$} &
                \textbf{95\,\%$_{\rm W}$} \\
            \midrule
264           & 1.374 & 24.8 & 21.3 & 25.0 &  3.7 & 20.7 & 32.0 & 11.3 \\
270$^\dagger$ & 1.369 & 38.9 & 37.9 & 39.8 &  1.9 & 37.0 & 40.9 &  3.9 \\
277           & 1.364 & 68.1 & 55.7 & 80.9 & 25.2 & 41.5 & 92.0 & 50.6 \\
282           & 1.359 & 56.6 & 52.5 & 59.8 &  7.3 & 49.7 & 64.2 & 14.5 \\
288           & 1.353 & 61.6 & 59.0 & 65.1 &  6.1 & 55.4 & 67.4 & 12.0 \\
248           & 1.198 & 27.3 & 24.8 & 28.7 &  3.8 & 23.4 & 31.5 &  8.2 \\
255           & 1.193 & 24.0 & 21.9 & 24.7 &  2.8 & 21.0 & 27.4 &  6.4 \\
260$^\dagger$ & 1.188 & 40.3 & 39.2 & 41.3 &  2.1 & 38.3 & 42.4 &  4.1 \\
234           & 1.183 & 33.4 & 16.7 & 42.7 & 26.0 & 13.8 & 55.6 & 41.8 \\
241           & 1.178 & 43.8 & 40.9 & 46.7 &  5.8 & 38.2 & 49.9 & 11.7 \\
            \bottomrule
        \end{tabular}

        \vspace{0.5em}
        $\dagger$: Denotes runs at resonance energy
    \end{center}
\end{table}
\end{landscape}

The acceptance values have different uncertainties based on the
specifics of the run, such as the beam energy used and the underlying
cross section. The uncertainty around the on-resonance runs are the
smallest due to the higher count rate, shorter run times, and the
underlying cross section. The cross section for the on-resonance runs
peaks near the center of the energy range covered by the target,
resulting in a slight shift in beam energy having a smaller effect on
the final yield and thus a lower final uncertainty due to the
uncertainty in the beam energy (see
Fig.~\ref{fig:example-cross-section}).

From our analysis, we can also decompose the uncertainty for each run
into the contributions from individual uncertainties through our Monte
Carlo method. By focusing in the four uncertainties that are primarily
under the control of the experimenter\---{}the beam energy uncertainty
$\delta E$, the run time uncertainty $\delta t$, the beam current
uncertainty $\delta i$, and the target thickness uncertainty
$\delta\Delta$\---{}the focus of subsequent experiments can be on
minimizing the largest uncertainty where necessary. The contributions of
the uncertainties are shown in
Tables~\ref{tab:acceptance-uncertainty-67} and
\ref{tab:acceptance-uncertainty-95}, where the values are the percentage
of the 67\,\% and the 95\,\% (respectively) confidence intervals that
remains when holding the given variable or variables constant.

\begin{figure}
    \begin{center}
        \label{fig:example-cross-section}
        \centerline{
            \includegraphics[width=0.8\textwidth]{figures/xs_example.png}}
        \caption[Example cross section]{Example cross section from the
            $R$-matrix fit~\ref{deBoer2017} for a run near the resonance
            energy. Note that the scale is linear. As the cross section
            peak is near the center of the energy range that the beam
            has within the target, the final yield is not drastically
            affected by the uncertainty in the beam energy.}
    \end{center}
\end{figure}

Since the final contributions to the uncertainty are not known, and the
correlations between different variables not explicitly taken into
account, these values should be used as a guide for future improvements
instead of relying on the absolute values directly. From the above
tables, we can make generalizations about which uncertainties may have
the largest impact on further studies. The on-resonance runs are
extremely dependent on the beam current uncertainty, as the beam current
directly affects what the final yield at the detector will be, and if a
smaller uncertainty band is required, controlling the beam current and
reducing its uncertainty should be the most important aspect of the
experiment. Conversely, the point just below the resonance in energy has
an extremely large final uncertainty, mainly due to the beam energy
uncertainty. Due to the nature of the cross section, a small shift in
beam energy can result in drastically different final yields due to the
rapidly changing cross section below the resonance. Subsequent
below-resonance runs should focus on reducing the uncertainty from the
beam energy, which requires spending more time fine-tuning the
accelerator to provide a more energy-stable beam to the experimental
area.

Additionally, the complete uncertainty for our acceptance values cannot
be reduced to zero. The irreducible uncertainty present (in the last row
of Tables~\ref{tab:acceptance-uncertainty-67} and
\ref{tab:acceptance-uncertainty-95}) for each run is from the
uncertainties due to SRIM, the detector statistics, and other sources as
described in Section~\ref{sec:uncertainties}.


\section{Advantages and Disadvantages}

The choice of measuring $({\rm p},\alpha)$ reactions using a recoil
separator such as St.\ George has a number of advantages and
disadvantages over other methods. As the general differences for using
recoil separators for reaction studies was discussed in
Ch.~\ref{ch:introduction} and will not be repeated here.

The primary advantage of using a recoil separator is that the reaction
can be studied at zero degrees. As more common experimental setups have
no way to stop the incident high-intensity beam and not the produced
particles along zero degrees, any detector placed at zero degrees would
either be destroyed due to the high particle rate or the signal
completely obscured by the noise from the high proton peak. By using a
recoil separator to reject that incident beam, the particles produced at
zero degrees can be directly studied in a relatively safe way (in
reference to the detector). The tradeoff is that the recoil separator
can \textit{only} measure those particles produced at zero degrees, and
thus cannot measure the full angular distribution.

Even when using a standard setup to measure the angular distribution,
the elastically scattered protons can still be seen at the detectors.
There is still a possibility that the count rate from these particles is
prohibitive for measuring the produced $\alpha$ particles. In this case,
some energy ranges might be inaccessible for study.

\begin{table}
\small
    \begin{center}
        \caption{ACCEPTANCE BOUNDS WITH HELD VARIABLES, 67\,\%}
        \label{tab:acceptance-uncertainty-67}
        \begin{tabular}{crrrrrrrrrr}
            \toprule
            \midrule
            \textbf{Held}
                & \textbf{241} & \textbf{234} & \textbf{260}$^\dagger$ & \textbf{255} & \textbf{248}
                & \textbf{288} & \textbf{282} & \textbf{277} & \textbf{270}$^\dagger$ & \textbf{264} \\
            \midrule
$E_{\rm p}$ [MeV] & 1.178 & 1.183 & 1.188 & 1.193 & 1.198 & 1.353 & 1.359 & 1.364 & 1.369 & 1.374 \\
\midrule
$\delta E$     &  49.0 &   5.8 &  96.0 &  92.1 &  95.7 &  64.2 &  37.8 &  13.8 & 106.2 &  68.8 \\
$\delta t$     & 101.4 &  93.5 &  98.3 & 103.3 & 101.8 &  97.4 & 101.4 & 100.2 & 105.0 &  99.2 \\
$\delta i$     &  90.7 &  96.1 &  24.7 &  29.3 &  61.1 &  78.1 &  90.8 & 101.9 &  27.0 &  87.4 \\
$\delta\Delta$ & 102.1 &  96.1 &  99.6 &  95.9 &  82.2 &  99.3 &  99.9 &  99.5 & 105.9 &  85.6 \\
$\delta E$, $\delta t$     &  51.0 &   5.9 &  97.8 &  89.5 &  92.5 &  62.9 &  38.3 &  13.3 & 101.3 &  66.7 \\
$\delta E$, $\delta i$     &  12.8 &   1.3 &  23.5 &  20.2 &  56.1 &  19.5 &  12.3 &   3.2 &  24.3 &  57.8 \\
$\delta E$, $\delta\Delta$ &  49.0 &   5.8 &  97.8 &  88.9 &  76.7 &  59.0 &  37.5 &  13.6 & 101.3 &  31.1 \\
$\delta t$, $\delta i$     &  92.3 &  99.7 &  11.4 &  23.3 &  63.0 &  75.9 &  92.0 & 100.0 &  12.5 &  87.2 \\
$\delta t$, $\delta\Delta$ & 101.8 &  94.7 &  98.1 &  98.4 &  80.3 &  91.9 &  98.2 &  98.6 & 102.7 &  83.2 \\
$\delta i$, $\delta\Delta$ &  88.9 &  98.8 &  25.2 &  22.4 &  24.1 &  76.1 &  92.6 &  98.9 &  26.1 &  77.1 \\
$\delta E$, $\delta t$, $\delta i$     &  11.2 &   0.8 &   7.8 &  17.4 &  56.0 &  18.9 &  11.4 &   1.3 &   5.4 &  58.2 \\
$\delta E$, $\delta t$, $\delta\Delta$ &  49.8 &   5.8 &  95.7 &  88.4 &  76.5 &  58.8 &  37.7 &  13.3 &  98.8 &  29.0 \\
$\delta E$, $\delta i$, $\delta\Delta$ &   7.8 &   1.3 &  21.3 &  10.1 &   5.0 &   8.5 &   7.1 &   2.8 &  22.8 &   6.6 \\
$\delta t$, $\delta i$, $\delta\Delta$ &  89.4 &  99.6 &  10.9 &  17.9 &  21.5 &  78.6 &  93.4 & 102.3 &  10.9 &  76.2 \\
All &   5.6 &   0.7 &   3.9 &   3.6 &   3.8 &   6.4 &   5.0 &   0.9 &   3.9 &   2.0 \\
            \bottomrule
        \end{tabular}

        \vspace{0.5em}
        {\fontsize{10}{12}\selectfont $\dagger$: Denotes runs at
        resonance energy}
    \end{center}
\end{table}

\begin{table}
\small
    \begin{center}
        \caption{ACCEPTANCE BOUNDS WITH HELD VARIABLES, 95\,\%}
        \label{tab:acceptance-uncertainty-95}
        \begin{tabular}{crrrrrrrrrr}
            \toprule
            \midrule
            \textbf{Held}
                & \textbf{241} & \textbf{234} & \textbf{260}$^\dagger$ & \textbf{255} & \textbf{248}
                & \textbf{288} & \textbf{282} & \textbf{277} & \textbf{270}$^\dagger$ & \textbf{264} \\
            \midrule
$E_{\rm p}$ [MeV] & 1.178 & 1.183 & 1.188 & 1.193 & 1.198 & 1.353 & 1.359 & 1.364 & 1.369 & 1.374 \\
\midrule
$\delta E$     &  49.4 &   7.5 & 101.1 &  75.8 &  93.0 &  64.1 &  39.6 &  13.2 & 102.4 &  50.9 \\
$\delta t$     & 101.8 &  95.3 & 101.3 &  99.8 &  99.6 & 102.7 & 101.0 &  97.9 &  99.0 & 107.6 \\
$\delta i$     &  86.6 &  97.3 &  26.8 &  51.1 &  66.3 &  77.7 &  93.9 &  93.6 &  26.5 & 102.0 \\
$\delta\Delta$ &  95.4 &  98.2 & 101.4 &  80.4 &  82.9 &  98.7 & 104.2 &  95.8 & 103.5 &  75.8 \\
$\delta E$, $\delta t$     &  47.4 &   7.2 &  96.7 &  73.4 &  93.4 &  63.8 &  39.6 &  13.2 & 101.4 &  50.2 \\
$\delta E$, $\delta i$     &  12.0 &   1.7 &  23.0 &  18.1 &  55.5 &  20.1 &  12.5 &   3.0 &  24.0 &  42.7 \\
$\delta E$, $\delta\Delta$ &  46.8 &   7.6 & 100.3 &  72.0 &  73.8 &  60.4 &  38.0 &  12.9 & 103.5 &  20.6 \\
$\delta t$, $\delta i$     &  88.0 &  97.2 &  13.4 &  50.1 &  64.6 &  78.2 &  96.8 &  96.9 &  15.2 &  94.7 \\
$\delta t$, $\delta\Delta$ &  98.5 &  98.3 &  98.2 &  86.7 &  83.5 &  99.4 & 102.0 &  98.8 &  99.8 &  75.9 \\
$\delta i$, $\delta\Delta$ &  87.7 &  95.9 &  26.6 &  29.7 &  29.1 &  76.9 &  93.4 &  93.5 &  25.6 &  72.9 \\
$\delta E$, $\delta t$, $\delta i$     &  11.1 &   1.0 &   8.1 &  18.6 &  54.0 &  17.8 &  11.5 &   1.5 &   8.3 &  43.5 \\
$\delta E$, $\delta t$, $\delta\Delta$ &  47.4 &   7.2 &  96.8 &  70.1 &  76.0 &  59.9 &  39.0 &  12.9 &  99.4 &  20.4 \\
$\delta E$, $\delta i$, $\delta\Delta$ &   7.9 &   1.6 &  21.4 &   8.4 &   5.0 &   8.4 &   7.4 &   2.7 &  22.4 &   4.6 \\
$\delta t$, $\delta i$, $\delta\Delta$ &  86.2 &  98.6 &  13.1 &  29.6 &  27.0 &  78.2 &  94.2 &  96.5 &  14.5 &  68.0 \\
All &   5.4 &   1.0 &   4.0 &   2.8 &   3.8 &   6.6 &   5.2 &   0.9 &   3.7 &   1.3 \\
            \bottomrule
        \end{tabular}

        \vspace{0.5em}
        {\fontsize{10}{12}\selectfont$\dagger$: Denotes runs at
        resonance energy}
    \end{center}
\end{table}


\section{Uncertainties}
\label{sec:uncertainties}

The final uncertainties on the acceptances at each run energy are skewed
distributions. Since basic error propagation relies on the errors being
gaussian distributed, the fact that our uncertainties are not partially
justifies the Bayesian approach described previously. Part of the reason
for the skewed distributions is that the acceptance is bounded by zero
and $\pi/2$, and since our distributions sit closer to the zero end
instead of near the middle of the range somewhat requires that the
distribution be skewed. Our process avoids propagating errors from
arbitrary or non-Gaussian distributions and allows for a richer
discussion of the contributions of each uncertainty.

\subsection{Statistical}

The uncertainties on most of the inputs can be considered to be gaussian
distributed, as that represents the statistical nature of the process
that creates that input value. Individual collections of these values
still exhibit a normal distribution. For values that may be extremely
skewed on are known to have other properties, a gaussian distribution
may not be the correct underlying uncertainty for the value.

Our final acceptance measurement has a skewed distribution, which can be
evidenced from the fact that the angular acceptance cannot be a negative
value. In this case, if we were to directly model the final uncertainty
of our acceptance, a log-normal distribution would be more apt. For each
uncertainty, a brief description of the underlying assumptions will be
presented.

\subsubsection{Target Thickness}
No specific underlying distribution was assumed for the target
thickness, although a Gaussian distribution could be used for
simplicity. The target thickness measurements performed before the
experiment were used to determine the particle density within the
target. The counts at the detector for the runs were assumed to be
Poisson distributed since they are discrete counts. From the peak shifts
due to energy loss when the $\alpha$ particles passed through the
target, the number of target nuclei could be determined. The original
spectra were used as the base properties to generate new count
distributions, each of which when combined with the energy loss would
lead to a unique number of target nuclei. The SRIM tabulated values that
were used to relate the energy loss through the target to the target
thickness were assumed to have a gaussian distribution, where the
standard deviation was taken as 3\,\% of the nominal value. For
subsequent uses of the target thickness, a number was drawn from the
generated distribution. If a model for the target thickness were
desired, the shape was approximately Gaussian with parameters $\mu =
1.42$ and $\sigma = 0.03$ ($\times 10^{18}$ nuclei/cm\squared{}).

\subsubsection{Beam Energy}
The beam energy was modeled with a Gaussian distribution. The underlying
statistical fluctuation of the multiple additive and multiplicative
values that determine the beam energy\---{}the power supplies for the
ion source, accelerator, and analyzing magnet\---{}result in a gaussian
distribution for the beam energy. While the beam energy can never be
negative and thus may be better modeled by a log-normal distribution,
the values of the beam energy are sufficiently far from zero such that
the assumption of a normal distribution can be used. The mean of the
distribution is based on the set point for the run, which the standard
deviation is constant for all runs. A value of 500~eV was taken as a
conservative value from previous energy calibration runs. Since no
explicit energy calibration was performed using the exact setup for this
experiment, this conservative value represents that additional
uncertainty.

\subsubsection{Beam Current}
The beam current was modeled with a Gaussian distribution. For similar
reasons as the beam energy uncertainty, a gaussian distribution for the
beam current was chosen. Observations on the beam current roughly showed
a normal distribution around a central value, and the values and
fluctuation were such that a log-normal distribution did not need to be
assumed. During experimental runs, samples of the beam current were
taken. Without a direct measurement of the beam current during the run,
which could be obtained using an offset Faraday cup or Si detector, no
more information about the beam current could be obtained. A Cauchy
distribution may be considered as an alternative, especially when the
accelerator system is producing extremely variable currents, due to the
heavier tails of the distribution. For our purposes, conservative
estimates of the beam current were obtained from a gaussian distribution
based on the current observations performed during the experiment and
shown in Table~\ref{tab:beam-current-uncertainty}.

\begin{table}
    \begin{center}
        \caption{BEAM CURRENT UNCERTAINTY}
        \label{tab:beam-current-uncertainty}
        \begin{tabular}{cS[table-format=4.2]S[table-format=4.2]S[table-format=4.2]}
            \toprule
            \midrule
            \textbf{Run} & \textbf{$E_{\rm p}$ [MeV]} & \textbf{$i_{\rm
                p}$ [$\mu$A]} & \textbf{$\delta i_{\rm p}$ [$\mu$A]} \\
            \midrule
                264           & 1.374 & 2.75 & 0.14 \\
                270$^\dagger$ & 1.369 & 2.60 & 0.13 \\
                277           & 1.364 & 2.70 & 0.14 \\
                282           & 1.359 & 2.50 & 0.13 \\
                288           & 1.353 & 2.34 & 0.15 \\
                248           & 1.198 & 2.18 & 0.24 \\
                255           & 1.193 & 1.85 & 0.19 \\
                260$^\dagger$ & 1.188 & 2.50 & 0.13 \\
                234           & 1.183 & 2.00 & 0.10 \\
                241           & 1.178 & 2.47 & 0.16 \\
            \bottomrule
        \end{tabular}

        \vspace{0.5em}
        $\dagger$: Denotes runs at resonance energy
    \end{center}
\end{table}

\subsubsection{Time}
The time was modeled with a Gaussian distribution. Our run times were
recorded both as a coarse clock time for the start and end of the run
and with the DAQ. For those runs lasting longer than 15 minutes,
additional stoppages in the measurement were made in the middle of the
run without stopping the DAQ. Values recorded by the DAQ also differed
from the set values for the total run time. As each of these factors has
a role in the total run time of the experiment and are approximately
additive in nature, a gaussian distribution makes sense here. The longer
runs had both the stoppage time estimated and subtracted from the total
run time, and the uncertainty was also taken to be larger. The
uncertainties for the run time are given in
Table~\ref{tab:run-time-uncertainty}.

\begin{table}
    \begin{center}
        \caption{RUN TIME UNCERTAINTY}
        \label{tab:run-time-uncertainty}
        \begin{tabular}{cccc}
            \toprule
            \midrule
            \textbf{Run} & \textbf{$E_{\rm p}$ [MeV]} & \textbf{$t$ [s]}
                & \textbf{$\delta t$ [s]} \\
            \midrule
                264           & 1.374 &  910 & 10 \\
                270$^\dagger$ & 1.369 &  906 & 10 \\
                277           & 1.364 &  989 & 10 \\
                282           & 1.359 & 3182 & 22 \\
                288           & 1.353 & 9436 & 53 \\
                248           & 1.198 & 5455 & 26 \\
                255           & 1.193 &  907 & 10 \\
                260$^\dagger$ & 1.188 &  907 & 10 \\
                234           & 1.183 & 1078 & 10 \\
                241           & 1.178 & 6503 & 51 \\
            \bottomrule
        \end{tabular}

        \vspace{0.5em}
        $\dagger$: Denotes runs at resonance energy
    \end{center}
\end{table}

\subsubsection{Detector Counts}
The total counts at the detector was modeled as a Poisson distribution.
As each bin within the Si detector spectra counts the particles that
fall within that energy bucket, a Poisson distribution was adopted for
the detector counts. This choice is common for counting experiments.

\subsubsection{SRIM Tabulated Values}
The uncertainty of the values obtained from SRIM were taken to be
Gaussian distributed. The target stopping power and those values related
to finding the target thickness were taken from SRIM. Due to the fact
that the \alpa{} reaction is relatively well studied, a conservative
estimate of 3\,\% was adopted for all values arising from SRIM results.

\subsubsection{Final Bands}
The final uncertainty bands for each of the acceptance measurements can
be analyzed by what values affect the range for the uncertainty. We can
limit this discussion to inputs that are controllable by the
experimenter. The final uncertainty will be made up of the uncertainty
from inputs and the uncertainty from those statistical and irreducible
processes. The four inputs that the experimenter can control are the
energy, time, current, and thickness uncertainties. The energy
uncertainty is related to the stability of the accelerator and the
calibration of the analyzing magnet, both of which can be measured and
regulated to the point where the uncertainty can be minimized. The time
uncertainty is based on the total runtime and the interruptions caused
by requiring the stoppage of the beam in order to measure the current
and can be reduced through synchronization of the DAQ with the start of
bombardment, and by minimizing interruptions during the data collection
process. The current uncertainty can be minimized by measuring the
current continuously during the experiment, as there will then be fewer
unknown changes in the beam current and a single value for the beam
current does not need to be applied to the entirety of the experimental
run. Finally, the thickness uncertainty can be minimized by performing
target thickness measurements at multiple energies and with potentially
multiple particles, and by running those measurements for longer such
that the energy loss by the particles can be more accurately determined.

Each of these inputs affects a different part of the final acceptance,
based on how it relates to the experimental and theoretical yield, or
both. We can determine the impact of reducing the uncertainty on each of
these inputs by setting the uncertainty to zero within the analysis
pipeline, which would return a different uncertainty band for the run in
question. Since these uncertainties are not necessarily independent of
each other, we should also look at all combinations of these four inputs
being controlled for to get a full picture of the importances.
Additionally, the irreducible uncertainty can be determined by keeping
all of the inputs constant. The contribution to the final uncertainty is
expressed as a percent of the total uncertainty band for both the 67\,\%
and 95\,\% confidence interval, so the amount of the band that is
accounted for by the inputs that are not held constant.

\subsection{Systematic}

Potential sources of systematic uncertainty must also be considered when
discussing the acceptance results. As low yield future experiments would
not have the relatively high count rates at the detector seen during
this experiment, the potential for lost counts due to the systematics
could drastically affect the final yield and thus the cross section
measurements. Potential sources of systematic uncertainty can be divided
into experimental sources and analytical sources.

\subsubsection{Analytical Sources}
The analytical sources are much less complex and numerous due to the
relatively straightforward analysis required. The assumption of the
symmetry of the acceptance cone is discussed in
Sec.~\ref{sec:uniformity-of-acceptances} and will not be repeated here.

The energy threshold used to discriminate against the possible protons
that reached the detector also discriminates against some of the
produced $\alpha$ particles. Adjusting this threshold opens up the
possibility of some proton particles being interpreted as $\alpha$
particles or reduce the number of detected $\alpha$ particles. The
choice of the value was to ideally eliminate any false positives
(protons interpreted as $\alpha$ particles). As the number of counts
below this threshold was small, the effect on the final yield is
minimal ($\ll 1$\,\%).

\subsubsection{Experimental Sources}
The experimental sources of systematic uncertainty are those sources
that affect the assumptions behind how many $\alpha$ particles reach the
detector. The first assumption is the 100\,\% transmission of the beam
and the simulated recoil particles to the detector plane. The particle
transmission was measured before the experiment, for a variety of
particle rigidities, between the target Faraday cup and the Wien filter
detector cup and shown to be consistent with 100\,\% transmission.
Additionally, $\alpha$ particles at similar energies to those found in
the experiment were shown to reach the actual detector for the full
40~mrad acceptance cone, as shown by the Wien filter quartz viewer. In
both of these tests, the possibility for a transmission value that is
not 100\,\%, either due to unforeseen energy losses or a shift or offset
in the Faraday cup measurements, opens up the possibility of the
detector system not measuring the full yield produced in the reaction.
This produced yield would be the yield within the measurable acceptance
cone for St.\ George. If the tune that was shown to have 100\,\%
transmission for the particles we'd like to detect, but instead the
transmission is lower, fewer counts would be seen at the detector and
the measured acceptance would be lower.

Within the separator between the target chamber and the detector plane,
there is a possibility that the produced $\alpha$ particles do not reach
the detector by interacting with something within the beamline. The
particles could strike the interior walls, scattering and losing energy,
or one of the removed pieces of diagnostic equipment, or scatter off of
the residual vacuum within the beamline. In these cases, the particles
produced within the acceptance cone do not reach the detector plane.
Those that may reach the detector plane could either be outside of the
physical extent of the detector or never reach the detector plane due
to these interactions. In all of these cases, the yield measured at the
detector will be lower than what was produced within the acceptance
cone, leading to a lower acceptance value.

The settings of St.\ George were determined by previous runs and
optimized to transport $\alpha$ particles from within a 40~mrad cone and
with a small energy spread to the active surface of the detector. Since
not every possible $\alpha$ energy could be checked, a single setting
for the separator was determined then scaled to the settings that would
match the expected $\alpha$ energy. This choice has the possibility of
under- and over-estimating the final acceptance. The systematic shift
would be caused by the produced particle being under- or over-focused at
different points along the separator, leading to a change in the final
position of the particles at the detector plane. While the scaling of
the elements in St.\ George are based on the rigidity of the particles
and the fields required, there is still a possibility of a systematic
shift in the magnetic field. This shift may come from magnetic
hysteresis that was not fully corrected for or an offset or differential
scaling in the power supplies for the magnets. This possibility may
especially affect the quadrupole magnets, since they are set only by the
current setting for the power supply and not a magnetic field
measurement.


\section{Uniformity of Acceptances}
\label{sec:uniformity-of-acceptances}

When calculating the acceptance for St.\ George, it was assumed that the
acceptance cone was described by a single opening angle. In practice,
the horizontal and vertical opening angles may be distinct from each
other. During preliminary experiments for the acceptance of St.\ George,
it required much less fine tuning of magnetic fields to achieve the
maximum vertical acceptance than it was to achieve the maximum
horizontal acceptance. This observation may be due to the lack of
dispersive elements in the vertical plane.

The strips of the Si detector were aligned such that an individual strip
was oriented in the vertical direction, or a particle that is deflected
horizontally would be detected on a different strip (see
Fig.~\ref{fig:det-position}). This orientation allowed for
improved tuning in the horizontal plane but with the tradeoff of the
lack of sensitivity in the vertical plane. The auxiliary runs performed
where the detector was place in the ``low'' position (where the top of
the detector is located where the bottom of the detector would be in the
regular running position) inform the amount of particles that are not
captured in the vertical plane due to minor mistuning of the separator,
and the auxiliary runs used to center the produced alpha particle
distribution on the detector horizontally inform the amount of particles
that are not captured in the horizontal direction. Ideally, a detector
segmented in both the horizontal and vertical plane would give a full
description of the alpha-particle beam spot density at the detector
plane and could be used to better relate the distribution of counts at
the detector plane to the acceptance cone at the target location.

In the final configuration of the target system, a series of conical
collimators will be located following the target location to defined the
40 mrad acceptance cone. As the desired configuration of St.\ George is
to measure $(\alpha,\gamma)$ reactions where the heavy recoil particles
are emitted from the target within a cone with an opening angle less
than 40~mrad, the potential effect of a non-symmetric acceptance is
reduced. For experiments similar to this where the ejected particles are
emitted within a cone larger than 40~mrad, these collimators would
ensure that the particles reaching the final detector must have been
emitted within that known acceptance cone. This restriction would
improve the tuning of the separator for similar experiments, as the
emitted particle beam spot at the detector plane can be more easily
tuned to fit completely on the detector.


\section{The $({\rm p},\alpha_1)$ Channel}
\label{sec:the-palpha_1-channel}

At the resonances probed, the $({\rm p},\alpha_1)$ reaction channel is
also open. Measuring the cross section for this reaction at the two
desired resonances is a more difficult experiment due to the lower
rigidity of the produced alpha particles due to the lower energy. The
kinematics for this reaction are given in Table~\ref{tab:alpha-one}.

\begin{table}
    \begin{center}
        \caption{KINEMATICS FOR $({\rm p},\alpha_1)$}
        \label{tab:alpha-one}
        \begin{tabular}{cc}
            \toprule
            \midrule
            \textbf{Quantity [units]} & \textbf{Value} \\
            \midrule
                Field [G]                   & 1690.37 \\
                $E_{\rm p}$ [MeV]           & 1.359 \\
                $i_{\rm target}$ [$\mu$A]   & 2.6 \\
                $E_{\alpha}$ [MeV]          & 1.397 \\
                $B\rho$ [Tm]                & 0.1702\\
                $E\rho$ [MV]                & 1.397 \\
            \bottomrule
        \end{tabular}
    \end{center}
\end{table}

The lower rigidity is still within the design parameters of St.\ George,
but due to the altered tune required to direct the produced alpha
particles to the detector plane has different rejection properties than
the standard tune. As such, the incident proton beam is close enough in
rigidity that the beam may strike the detector. The beam reduction
levels would not be high enough to avoid damaging the Si detector,
preventing the measurement of the cross section without either
additional rejection capabilities or an improvement in the tune.

There is still the possibility that, for different energies, the beam
rejection properties of St.\ George would be adequate for measuring
particles from excited states of the compound nucleus. An additional
possibility is utilizing the remaining third of St.\ George for its
final rejection capabilities. This solution has the added benefit of not
requiring an altered tune to transport the produced particles to the
detector. With the addition of the post-target collimators, the detected
recoils will only come from within 40~mrad, allowing for a more accurate
measurement of the yield for the reaction. Since the measured resonances
in this experiment are isotropic, it becomes much simpler to extrapolate
from the detector yield to the full, angle-integrated yield.
