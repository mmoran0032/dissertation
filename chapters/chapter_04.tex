\chapter{EXPERIMENTAL PROCEDURE}
\label{ch:experimental-procedure}

% we need to rework most/all of this chapter, remove the data analysis
% portion from here, etc.

% \begin{center}\textit{Alphas have more mass because they spend so much
% time at the gym getting swole. \---{} Laura Moran}\end{center}

An experimental campaign to study the \alpa{} reaction with the St.\
George recoil separator was undertaken at the NSL. Runs were completed
in December 2016 and February 2017, with runs focusing on determining
the correct magnetic fields within St.\ George completed in Fall 2016
and February 2017. Two low energy resonances were measured with beam
currents in the $2-3$~$\mu$A range in February 2017. Studying this
reaction provides a test of the angular and energy acceptances of St.\
George in preparation for studying $(\alpha,\gamma)$ reactions across a
wide range of targets and energies.

The first portion of these runs fall under general St.\ George
commissioning work as discussed in Chapter~\ref{ch:commissioning} and
will not be repeated here. The second portion of the runs involved
characterizing the target and the detector, finalizing the optimal
settings for the separator, and performing the experiment. The reaction
of interest produces $\alpha$ particles in the energy range of $2-3$~MeV
for the desired proton energy range.


\section{Altered Tune}

The magnet settings for St.\ George were determined to transport
$\alpha$ particles, where the entirety of the particle envelope is
described by a characteristic energy and angular [term]. The produced
$\alpha$ particles had a low (values?) energy spread and a high angular
spread, where only those particles emitted within the desired 40~mrad
acceptance cone for St.\ George were tuned to reach the detector focal
plane $F_2$ after the Wien filter and impinge the installed Si strip
detector.

The restrictions on the beam spot for measuring $(\rm{p}\alpha)$
reactions at this focal plane require an approximately symmetric spot
size in both directions and one that is smaller than the physical face
of the detector, whereas the standard tune for studying
$(\alpha,\gamma)$ reactions required that beam spot to be asymmetric
with the beam spot being narrow in the dispersive $x$-plane and tall in
the $y$-plane. The initial COSY code for St.\ George (see
Section~\ref{sec:cosy}) was altered to model the shortened separator and
provide information on the beam characteristics at the new detector
focal plane. The magnetic field settings for the seven quadrupoles
$Q_{1-7}$ were adjusted to transport the recoil particles to the
detector plane with a final beam spot no larger than the face of the Si
detector of $58\times 58$~mm. Final pole tip fields are given in
Table~\ref{tab:poletip}.

\begin{table}
    \begin{center}
        \caption{POLE TIP FIELDS FOR $(\alpha,\gamma)$ AND
            $(\rm{p},\alpha)$ STUDIES}
        \label{tab:poletip}
        \begin{tabular}{cS[table-format=2.9]S[table-format=2.6]}
            \toprule
            \midrule
             & \multicolumn{2}{c}{\textbf{Pole Tip Field [T]}} \\
            \textbf{Quadrupole} & {$(\alpha,\gamma)$} &
            {$(\rm{p},\alpha)$} \\
            \midrule
            1  & -0.16303276 & -0.157\\
            2  &  0.18882363 &  0.187\\
            3  &  0.09384148 &  0.09411\\
            4  & -0.12620402 & -0.04\\
            5  &  0.10032405 &  0.092 \\
            6  &  0.04693654 &  0.0585 \\
            7  &  0.0        & -0.015 \\
            8  & -0.09779179 & \\
            9  &  0.17439627 & \\
            10 &  0.21092228 & \\
            11 & -0.13962355 & \\
            \bottomrule
        \end{tabular}
    \end{center}
\end{table}

For the $(\rm{p},\alpha)$ experiment, the transported $\alpha$ particles
have the properties listed in Table~\ref{tab:alpha_prop}. The incident
proton beam is rejected within the COSY ion optics solution after the
first dipole doublet $B_1B_2$, and the beam properties are not listed
here.

\begin{table}
    \begin{center}
        \caption{ALPHA PARTICLE PROPERTIES}
        \label{tab:alpha_prop}
        \begin{tabular}{cc}
            \toprule
            \midrule
            \textbf{Property [Unit]} & \textbf{Value} \\
            \midrule
            Energy [MeV]        & 2.504 \\
            $\Delta$Energy [\%] & 3 \\
            Angular spread [mrad] & 40 \\
            Target diameter [mm] & 3 \\
            $Q$ [$e$]           & 2 \\
            $B\rho$ [Tm]        & 0.228 \\
            $E\rho$ [MV]        & 4.0 \\
            \bottomrule
        \end{tabular}
    \end{center}
\end{table}

\subsection{Separator Properties}

The energy resolving power is the minimum energy difference required to
resolve a peak from the central image peak assuming that the change in
energy is the only difference between the two peaks. By definition this
quantity is only a first-order value, so only those parameters with a
linear relationship with the position need be considered. The energy
resolving power of the separator in relation to the terms present in the
COSY transport map is defined as
\begin{equation}
    \delta_k(\textrm{RP}) \equiv
        \frac{2\left[(x|x)x_0 + (x|a)a_0\right]}{(x|\delta_k)},
\end{equation}
where $x_0$ and $a_0$ are the initial half-widths for position (in
meters) and angle (in radians), respectively, and the remaining terms
are the values from the transport map. The resolving power is only taken
in the horizontal plane due to the vertical symmetry of the separator.
The terms taken from the transport map are
\begin{align*}
    (x|x) &= 2.261610 \\
    (x|a) &= {-0.1368242} \\
    (x|\delta_k) &= {-0.2774295},
\end{align*}
where signs are conserved for completeness. The maximal deviation caused
by each terms is taken to be a positive value. The half-widths $x_0$ and
$a_0$ are physically limited by the target chamber and taken to be $x_0
= 1.5$~mm and $a_0 = 42$~mrad, giving a resolving power of
$\delta_k(\textrm{RP}) = 0.286$. Since the produced $\alpha$ particles
have an inherent spread in energy due to the incoming beam and the
particles themselves interacting with the target, the energy resolution
should be viewed as the window within which the energies are
indistinguishable. As this window covers the expected energy spread of
the produced $\alpha$ particles, there are no energy corrections
required across the detector strips.

Beam currents at the target location were recorded before and after each
run. For runs lasting longer than 15~m, the current was recorded every
15~m. The beam current was seen to fluctuate around the recorded value
by up to 100~nA. For runs with multiple current readings, the average
was taken as the nominal current. Time on target was recorded by the
acquisition system.

\subsection{Beam Reduction}

Incident proton beam reduction on the order of $10^{10} - 10^{14}$ are
required to avoid damaging the Si detector and observing off-resonance
yields of the produced $\alpha$ particles. These limits are within the
designed capabilities of St.\ George but must be verified experimentally

% Glenlivet 12 Year

\section{Campaign Procedure}

The procedure for performing the \alpa{} measurements is divided between
the experimental procedure and the procedure to measure a single energy
point for clarity. Verification of the initial field settings of St.\
George was discussed in Section~[REFERENCE] as is assumed for the
remainder of this discussion.


\section{Run Procedure}

At each energy point, preliminary runs were performed to verify the
field settings of St.\ George and determine the systematic uncertainties
of the tune. Following these checks, a final measurement at that energy
was performed until the statistical uncertainty was below 1\,\% for
on-resonance points and below 5\,\% for off-resonance points.
