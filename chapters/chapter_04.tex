\chapter{MEASURING THE \alpa{} CROSS SECTION}

% \begin{center}\textit{Alphas have more mass because they spend so much
% time at the gym getting swole. \---{} Laura Moran}\end{center}

An experimental campaign to study the \alpa{} reaction with the St.\ George
recoil separator was undertaken at the NSL. Runs were completed in December
2016 and February 2017, with runs focusing on determining the correct magnetic
fields within St.\ George completed in Fall 2016 and February 2017. Two low
energy resonances were measured with beam currents in the $2-3$~$\mu$A range in
February 2017. Studying this reaction provides a test of the angular and energy
acceptances of St.\ George in preparation for studying $(\alpha,\gamma)$
reactions across a wide range of targets and energies.

The first portion of these runs fall under general St.\ George
commissioning work as discussed in Chapter~\ref{ch:commissioning} and will not
be repeated here. The second portion of the runs involved characterizing the
target and the detector, finalizing the optimal settings for the separator, and
performing the experiment. The reaction of interest produces $\alpha$ particles
in the energy range of $2-3$~MeV for the desired proton energy range.


\section{Altered Tune}

The magnet settings for St.\ George were determined to transport $\alpha$
particles, where the entirety of the particle envelope is described by a
characteristic energy and angular [term]. The produced $\alpha$ particles had
a low (values?) energy spread and a high angular spread, where only those
particles emitted within the desired 40~mrad acceptance cone for St.\ George
were tuned to reach the detector focal plane $F_2$ after the Wien filter and
impinge the installed Si strip detector.

The restrictions on the beam spot for measuring $(\rm{p}\alpha)$ reactions
at this focal plane require an approximately symmetric spot size in both
directions and one that
is smaller than the physical face of the detector, whereas the standard tune
for studying $(\alpha,\gamma)$ reactions required that beam spot to be
asymmetric with the beam spot being narrow in the dispersive $x$-plane and
tall in the $y$-plane. The initial COSY code for St.\ George
(see Section~\ref{sec:cosy}) was altered
to model the shortened separator and provide information on the beam
characteristics at the new detector focal plane. The magnetic field settings
for the seven quadrupoles $Q_{1-7}$ were adjusted to transport the recoil
particles to the detector plane with a final beam spot no larger than the face
of the Si detector of $58\times 58$~mm. Final pole tip fields are given in
Table~\ref{tab:poletip}.

\begin{table}
    \begin{center}
        \caption{POLE TIP FIELDS FOR $(\alpha,\gamma)$ AND
            $(\rm{p},\alpha)$ STUDIES}
        \label{tab:poletip}
        \begin{tabular}{cS[table-format=2.9]S[table-format=2.6]}
            \toprule
            \midrule
             & \multicolumn{2}{c}{\textbf{Pole Tip Field [T]}} \\
            \textbf{Quadrupole} & {$(\alpha,\gamma)$} & {$(\rm{p},\alpha)$} \\
            \midrule
            1  & -0.16303276 & -0.157\\
            2  &  0.18882363 &  0.187\\
            3  &  0.09384148 &  0.09411\\
            4  & -0.12620402 & -0.04\\
            5  &  0.10032405 &  0.092 \\
            6  &  0.04693654 &  0.0585 \\
            7  &  0.0        & -0.015 \\
            8  & -0.09779179 & \\
            9  &  0.17439627 & \\
            10 &  0.21092228 & \\
            11 & -0.13962355 & \\
            \bottomrule
        \end{tabular}
    \end{center}
\end{table}

For the $(\rm{p},\alpha)$ experiment, the transported $\alpha$ particles
have the properties listed in Table~\ref{tab:alpha_prop}. The incident proton
beam is rejected within the COSY ion optics solution after the first dipole
doublet $B_1B_2$, and the beam properties are not listed here.

\begin{table}
    \begin{center}
        \caption{ALPHA PARTICLE PROPERTIES}
        \label{tab:alpha_prop}
        \begin{tabular}{cc}
            \toprule
            \midrule
            \textbf{Property [Unit]} & \textbf{Value} \\
            \midrule
            Energy [MeV]        & 2.504 \\
            $\Delta$Energy [\%] & 3 \\
            Angular spread [mrad] & 40 \\
            Target diameter [mm] & 3 \\
            $Q$ [$e$]           & 2 \\
            $B\rho$ [Tm]        & 0.228 \\
            $E\rho$ [MV]        & 4.0 \\
            \bottomrule
        \end{tabular}
    \end{center}
\end{table}



\section{Experimental Considerations}

\subsection{Beam Reduction}
Incident proton beam reduction on the order of $10^{10} - 10^{14}$ are required
to avoid damaging the Si detector and observing off-resonance yields of the
produced $\alpha$ particles. These limits are within the designed capabilities
of St.\ George but must be verified experimentally

% Glenlivet 12 Year



\section{Procedure}

The procedure for performing the \alpa{} measurements is divided between the
experimental procedure and the procedure to measure a single energy point for
clarity. Verification of the initial field settings of St.\ George was
discussed in Section~[REFERENCE] as is assumed for the remainder of this
discussion.

\subsection{Campaign Procedure}



\subsection{Run Procedure}

At each energy point, preliminary runs were performed to verify the field
settings of St.\ George and determine the systematic uncertainties of the tune.
Following these checks, a final measurement at that energy was performed until
the statistical uncertainty was below 1\,\% for on-resonance points and below
5\,\% for off-resonance points.




\section{Data Reduction and Analysis}

Each region of interest covered a single narrow resonance from the \alpa{}
reaction. Extraction of resonance parameters from these regions required a
data analysis pipeline to process the individual spectra from each strip of
the Si detector and the detector spectra as a whole. Detection systematic
uncertainties were determined individually for each energy measure prior to
the final measurement at that energy. Target effects on the incident beam
and produced $\alpha$ particles were explored [HOW?].

\subsection{Experimental Systematics}
The properties of the target, detector, and separator for the given experiment
were explored through various means before, during, and after the data
collection phase. Where applicable, the properties were compared to anticipated
or predicted values. Differences between operating and testing conditions may
be a cause of certain discrepancies within the detector spectra, as discussed
below.

\subsubsection{Target Properties}
The self-supporting \nuc{27}{Al} target was measured to have a thickness of
$62.50 \pm 0.05$~$\mu$g/cm$^2$. Target thickness was measured using an offline
detector station with a \nuc{241}{Am}/\nuc{148}{Gd} mixed $\alpha$ source. Runs
with and without the target foil between the source and the detector lasted
600~s. The annular Si detector was not able to resolve the lowest intensity
\nuc{241}{Am} peak during the runs, and only the highest intensity peak could
be reliably resolved in the spectrum obtained with the target in place. The two
spectra are shown in Fig.~\ref{fig:calibration}, and the $\alpha$-particle
peaks are given in Table~\ref{tab:calibration}.

\begin{figure}[t]
    \begin{center}
        \centerline{\includegraphics[width=1.0\textwidth]%
            {figures/target_thickness.png}}
        \caption[Target thickness measurement]{Target thickness measurement,
            showing the shift in the $\alpha$ peaks to lower energies due to
            the presence of the target. The initial spectrum is in green and
            the degraded spectrum is in orange. Only the energy range of
            interest is shown.}
        \label{fig:calibration}
    \end{center}
\end{figure}

\begin{table}
    \begin{center}
        \caption{ALPHA PARTICLE ENERGIES FOR \nuc{241}{Am}/\nuc{148}{Gd} MIXED
            SOURCE}
        \label{tab:calibration}
        \begin{tabular}{cS[table-format=5.3]%
                S[table-format=3.1, table-space-text-post=\,\%]}
            \toprule
            \midrule
            {\textbf{Isotope}} & {$\mathbf{E_{\alpha}}$\textbf{ [keV]}} &
                {\textbf{Intensity}} \\
            \midrule
            \nuc{148}{Gd} & 3182.69 & 100\,\% \\
            \nuc{241}{Am} & 5388    &   1.6\,\% \\
                          & 5442.8  &  13.1\,\% \\
                          & 5485.56 &  84.8\,\% \\
            \bottomrule
        \end{tabular}
    \end{center}
\end{table}

The spectra were calibrated from the source-only run using the known energies
of the emitted peaks. The shift in energy of the two largest peaks were
recorded. The expected range for each peak was determined by interpolating the
tabulated results from SRIM~\cite{SRIM}. The target thickness was the difference in
range for each peak between the degraded and undegraded $\alpha$ energies.

During the experiment, the total time of beam on target was minimized to limit
the amount of carbon deposited on the target. Total charge accumulation on the
order of 100~mC, with $\approx 75$~mC from the final measurement runs. The
longest measurement run deposited $\approx 25$~mC on the target. At
these levels, no appreciable change to the target thickness during the
experiment could be seen. The target thickness was not remeasured following
the end of the experimental campaign.


\subsubsection{Detector Properties}
The 16-strip Si detector was caibrated using a \nuc{241}{Am}/\nuc{148}{Gd}
mixed $\alpha$ source. Calibration runs were taken before and after the data
collection phase, since the bias voltage was changed for the final runs. The
final calibration run was taken with the same electronics setup and detector
installation as during the run, except that the beam shield was removed.

Each strip and ADC were calibrated separately. A linear fit was used, as there
only the two highest intensity $\alpha$-particle peaks were resolved. Bins
were shown to be approximately 2~keV for every strip. The resolution is cited
for only the $E_{\alpha} = 3182.69$~keV peak from \nuc{148}{Gd} as this is
closer in energy to our expected particles. The calibration constants and
detector resolutions are shown in Table~\ref{tab:calibration_results}.

The efficiencies of the individual strips were not determined and assumed to be
100\,\% for all. Measurements of similar detectors for the St.\ George detector
system commissioning work showed $>99$\,\% efficiencies across all strips.

The detector response shows low energy tails for both alpha peaks (see
Fig.~\ref{fig:response}). The data show that the $\alpha$ events may fall
outside of the two peaks. All events within our detector can then be considered
real events, reducing the reliance on the energy calibration. The data also
shows that the lower intensity \nuc{241}{Am} $\alpha$ peaks cannot be resolved.

\begin{table}
    \begin{center}
        \caption{DETECTOR ENERGY CALIBRATION AND RESOLUTION}
        \label{tab:calibration_results}
        \begin{tabular}{cS[table-format=4.2]S[table-format=1.4]%
                S[table-format=1.2, table-space-text-post=\,\%]}
            \toprule
            \midrule
            {\textbf{Strip}} & {$\mathbf{a_0}$\textbf{ [keV]}} &
                {$\mathbf{a_1}$\textbf{ [keV/ch]}} & {\textbf{Resolution}} \\
            \midrule
             1 &  -85.61 & 1.6534 & 1.92\,\% \\
             2 & -165.12 & 1.7327 & 2.67\,\% \\
             3 & -123.66 & 1.5791 & 2.38\,\% \\
             4 & -172.41 & 1.6824 & 2.71\,\% \\
             5 & -194.53 & 1.6923 & 2.71\,\% \\
             6 & -235.36 & 1.7636 & 2.88\,\% \\
             7 & -202.59 & 1.7946 & 2.99\,\% \\
             8 & -203.93 & 1.8526 & 2.85\,\% \\
             9 & -269.19 & 1.8725 & 3.00\,\% \\
            10 & -252.46 & 1.8166 & 3.03\,\% \\
            11 & -243.02 & 1.8509 & 2.91\,\% \\
            12 & -272.25 & 1.8291 & 2.87\,\% \\
            13 & -235.06 & 1.8161 & 2.97\,\% \\
            14 & -172.81 & 1.7601 & 2.77\,\% \\
            15 & -235.26 & 1.8246 & 3.10\,\% \\
            16 & -110.00 & 1.6412 & 2.17\,\% \\
            \bottomrule
        \end{tabular}
    \end{center}
\end{table}

\begin{figure}
    \begin{center}
        \centerline{\includegraphics[width=1.0\textwidth]%
            {figures/calibration_overlay.png}}
        \caption[Detector response example]{Example of the detector response
            from the energy calibration run. All 16 calibrated strips are
            overlayed to better show the overall trend. Long, low-energy tails
            from both calibration peaks can be seen. The counts in the lowest
            energy range are noise.}
        \label{fig:response}
    \end{center}
\end{figure}


\subsubsection{Separator Properties}
The energy resolving power is the minimum energy difference required to resolve
a peak from the central image peak assuming that the change in energy is the
only difference between the two peaks. By definition this quantity is only a
first-order value, so only those parameters with a linear relationship with the
position need be considered. The energy resolving power of the separator
in relation to the terms present in the COSY transport map is defined as
\begin{equation}
    \delta_k(\textrm{RP}) \equiv
        \frac{2\left[(x|x)x_0 + (x|a)a_0\right]}{(x|\delta_k)},
\end{equation}
where $x_0$ and $a_0$ are the initial half-widths for position (in meters) and
angle (in radians), respectively, and the remaining terms are the values from
the transport map. The resolving power is only taken in the horizontal plane
due to the vertical symmetry of the separator. The terms taken from the
transport map are
\begin{align*}
    (x|x) &= 2.261610 \\
    (x|a) &= {-0.1368242} \\
    (x|\delta_k) &= {-0.2774295},
\end{align*}
where signs are conserved for completeness. The maximal deviation caused by
each terms is taken to be a positive value. The half-widths $x_0$ and $a_0$ are
physically limited by the target chamber and taken to be $x_0 = 1.5$~mm and
$a_0 = 42$~mrad, giving a resolving power of $\delta_k(\textrm{RP}) = 0.286$.
Since the produced $\alpha$ particles have an inherent spread in energy due to
the incoming beam and the particles themselves interacting with the target, the
energy resolution should be viewed as the window within which the energies are
indistinguishable. As this window covers the expected energy spread of the
produced $\alpha$ particles, there are no energy corrections required across
the detector strips.


\subsection{Yield Extraction}
For each energy, the number of $\alpha$-particle ejectiles produced was the
total sum of counts within each detector strip above the maximum proton energy.
Due to the detector response, any event above the noise threshold at
$\approx 200$~keV is a potential $\alpha$ event. There is still the possibility
that some of the incident proton beam strikes the detector despite the
rejection capabilities of St.\ George, so those counts below the maximum proton
energy are rejected. The maximum proton energy is taken as the incident beam
energy plus 3\,\% to be conservative. Since the proton beam is degraded in
energy based on its interaction with the thin foil, this upper limit for the
proton energy and lower limit for the $\alpha$ energy range will prevent any
proton-induced counts from being counted. An example is shown in
Fig.~\ref{fig:proton_peak}. While most runs did not show a discernible peak at
the expected energy, this cut was still made. In those cases, ``lost'' counts
were smaller than the statistical uncertainty of the counts above the energy
threshold.

\begin{figure}[t]
    \begin{center}
        \centerline{\includegraphics[width=0.85\textwidth]%
            {figures/proton_peak.png}}
        \caption[Possible proton peak]{Possible proton peak within the spectrum
            of a single ADC. Data is from Run 241
            ($E_{\textrm{p}} = 1.1781$~MeV). The vertical dashed line shows the
            energy cut, with the potential beam peak below this energy. The
            remaining 15 strips for this run showed a similar peak in both
            central energy and counts.}
        \label{fig:proton_peak}
    \end{center}
\end{figure}

Beam currents at the target location were recorded before and after each run.
For runs lasting longer than 15~m, the current was recorded every 15~m. The
beam current was seen to fluctuate around the recorded value by up to 100~nA.
For runs with multiple current readings, the average was taken as the nominal
current. Time on target was recorded by the acquisition system.

The total yield for each energy is given by $Y(E) = N_r / N_b$, where $N_r$ is
the number of reaction products produced and $N_b$ is the number of incident
beam particles. If we include the detector and transport efficiency of our
setup, and relate $N_b$ to our incident beam current, our yield becomes
\begin{equation}
    \label{eq:yield}
    Y(E) = \frac{N_r}{\epsilon_d\epsilon_tI_bt}.
\end{equation}
In this case, we assume that the detector and transport efficiencies are
100\,\%, based on the usage of a Si detector and the setting of St.\ George
to create a 100\,\% transmission state.

The uncertainties in the number of incident beam particles come from the
uncertainty in the collection time at the detector and the uncertainty or
walk within the incident beam current arriving at the target location.
Due to the start-up and shut-down timing for the DAQ, a time uncertainty of
5~s was adopted for each run, which should be a conservative value.
Without the offset Si detector to measure scattering at the target, we
do not have a direct measure of the beam current \textit{in situ} and must
rely on the beam currents measured before and after a given experimental run.
From those measurements, and from previous experimental runs using the 5U,
an uncertainty of 0.1~$\mu$A (less than 5\,\% in most cases) was adopted.
Final uncertainty for the number of incident particles is thus below a 5\,\%
statistical threshold.

% systematic for beam current stability...?

The uncertainty in the number of reaction products produced can be divided
between the statistical uncertainty in the counting of particles at the
detector and the systematic uncertainty of the missed particles at the
detector plane due to the tuning of the magnetic and electrostatic elements
of St.\ George. The statistical uncertainty of the counts is $\approx 5$\,\%,
as experimental runs were continued until this point. The systematic uncertainty
can be approximated by using the additional runs performed before the actual
experimental run, used to finalize the tune of the Wien filter and to estimate
the loss below the detector position, to estimate the ``lost'' counts
arising from the final experimental configuration of the separator.

% determine systematic uncertainty


The mounted target is a $62.50\pm0.05$~$\mu$g/cm$^2$ self-supporting Al foil
provided by Dr.\ Simon's group at the NSL. The target thickness was determined
with a mixed \nuc{241}{Am}/\nuc{148}{Gd} $\alpha$ source, providing $\alpha$
particles across the expected energy range, as shown in Table~[REFERENCE].
Only the two highest intensity peaks in \nuc{241}{Am} could be reliably
discriminated from the background, thus providing only three energy points for
calibration purposes of the detector. Target thicknesses were measured using an
offline setup, consisting of a Si detector within a vacuum chamber connected to
a data acquisition system. Data was recorded using MAESTRO for Windows
[REFERENCE] and converted using \texttt{pyne}.




\section{Detector Spectrum
Verification}\label{detector-spectrum-verification}

Due to the suboptimal energy resolution of the Si strip detector, caused
by the decreased bias voltage and increased leakage current, a
well-defined spectrum for the produced $\alpha$ particles was not
obtained. The spectra for each run and each strip show a wide peak at
the expected location of the $\alpha$ energy peak which we need to
verify that it represents the same underlying expected energy peak
produced in the reaction. This final verification is a convolution of
the incident beam energy, the properties of the target, the cross
section domain that we are probing, and the energy resolution of the
detector. Each of these components will be discussed in turn.

To reduce this check to a smaller subset of possibilities, only the two
resonance energy runs will be checked, but the same process described
below can be repeated for any energy in question, given the existence of
the required files to support the analysis.

This analysis relies on the usage of SRIM/TRIM [CITE] to generate
the files describing the target effects, and those files should be
generated according to the steps described below before the analysis is
begun. The file generation can be done programmatically following the
SRIM user guide, or ``by hand'' using the included GUI, both of which
are standard procedures and will not be discussed here.


\subsection{Assumptions}\label{assumptions}

In order to fully describe reaction process from incident beam to
detector energy spectrum, we must make some assumptions and check that
they are approximately true for the desired reaction.

First, the stopping power of the target must be slowly varying across
the width of the target. This assumption reduces the number of required
SRIM files, as described below. The definition of ``slowly varying'' in
this case is that the stopping power can be locally modeled as a linear
function of the beam energy, and that the coefficient relating the
energy to the stopping power be relatively small.

Second, knowledge of the underlying cross section is required. The
procudure described below can be inverted slightly to move from the
detector energy spectrum to the reaction cross section, knowing all
other parts, but this more-complex procedure was not required for the
reaction in question. Using known resonance properties, the
\react{\mnuc{27}{Al}}{rm{p}}{alpha}{} reaction cross section
can be modeled with the AZURE [CITE] $R$-Matrix code to produce
the file. The energies returned from this code should be transformed
into lab coordinates.

Third, the target should be thin enough that the beam can pass through
the target relatively unimpeded. Should this assumption not be met, the
procedure outlined must be updated to account for the additional
straggling, especially where the exit angle of the particles is
sufficiently spread out.

Finally, the thickness of the target in energy for the incident beam
energies must be known. Given that the thickness in
$mu\rm{g}/\rm{cm}^2$ can be measured using standard techniques,
the thickness in energy can be determined by
\[
    EQUATION HERE,
\]
where $d\rm{E}/dx$ is the stopping power in units of [UNITS]. The
target thickness only needs to be determined for the incident beam, as
this defines the cross section energy domain.

The additional knowledge required to perform the following analysis,
such as the incident beam energy uncertainty and the detector energy
resolution, will not be discussed separately.


\subsection{Required SRIM Files}\label{required-srim-files}

The pre-generation of SRIM files requires special note to justify the
energy and depths used. In cases where the incident beam energy spread
is greater, the target is thicker, the stopping power varies quickly, or
any other deviation from the assumptions above, the procedure below must
be amended and additional SRIM files will most likely need to be
generated to account for those changes.

The incident beam files describe the target effects on the beam energy
at multiple depths within the target. To simplify, all incident
particles are assumed to strike the target perpendicular to the target,
and the angles for each particle following their interaction with the
target are saved but not used within the analysis. While the primary
goal of this portion of the analysis is verifying the energy spectrum,
the slight corrections to the energy loss attributed to the angle of the
particles, these corrections would be minimal and do not affect the
final result. The initial energies chosen for the incident beam protons
are the central resonance energies, and the energies chosen for the
$\alpha$ particles cover the expected range on possible energies based
on previous kinematics studies.

The depths probed for the target are in percents of the total thickness
in Angstroms. For this analysis, depths from 1 to 99\,\% in steps of 2\,\%
were used, which provides adequate coverage of the energy loss through
the target while minimizing the amount of runs required for SRIM. Both
the incident beam protons and the produced $\alpha$ particles require
thicknesses across the entire range of the target thickness. Note that
depths of 0 and 100\,\% were not used; depths of zero percent have no
meaning for the incident beam case, and depths of 100\,\% are unphysical
since the incident particle will have left the target without reacting.

All files simulated 10k particles, and the final energies of those
particles were saved.


\subsection{Simulating the reaction}\label{simulating-the-reaction}

The process below describes how the full reaction, from the incident
beam to the final detector energy spectrum, can be simulated. The SRIM
files required are assumed to be generated beforehand. The process has
been wrapped within the \verb+pyne+ package under the \verb+tree+
submodule, which takes as input the path to the generated files and the
masses and energies of the particles in question.

\subsubsection{Simulating the incident
beam}\label{simulating-the-incident-beam}

The incident beam energy for the resonance scans were determined by the
5U's analyzing magnet settings. The 5U provides a small energy
uncertainty particle beam to the target area, meaning that we have
approximately a monoenergetic beam striking our target. The generated
SRIM files for the incident beam are this central beam energy at
multiple depths within the target. To simulate the known spread of the
beam energy around this central value, the deviation from this central
value was randomly assigned, assuming a Gaussian distribution of
energies. The individual particle energies are given as
\[
    E_i \sim E_{\rm{Resonance}} + \rm{Normal}(0, \sigma),
\]
where $\sigma = 300$~eV is the measured beam energy uncertainty of the 5U.

The cross section that the incident particles will probe is defined by
the initial energy of the beam and the thickness of the target in units
of energy. For the target in question, the energy thickness is
approximately 10~keV for both resonances. The cross
section within this energy domain was determined for the central
resonance energy, which is an approximation that initially ignores the
energy spread of the beam. As this energy spread is less than 0.1\,\%,
the effect of this choice is minimal and helps to simplify the process
since a single probability distribution for the cross section needs to
be determined. The cross section within the energy range is divided into
energy bins of 1\,\%, with the values interpolated where the results of
the $R$-matrix calculation do not match with the integer depth values.
These depths are then randomly chosen based on the normalized cross
section in that region. An example is shown in [FIGURE].

[FIGURE]

Once a simulated depth is assigned to each particle, the SRIM results at
that depth are used to simulate the final energy of that particle before
it reacts with the target. As the energy distributions returned by SRIM
are not analytic, we can draw from the 10k simulated values, assuming
equal probability for each, which for large sample sizes is
approximately equivalent. Each incident particle is now associated with
an initial energy deviation from the resonance energy, the depth at
which it reacts, and a final energy at that depth. The initial simulated
energy is added to the final energy to give the textit{reaction
energy} of the incident proton.


\subsubsection{Simulating the produced
particles}\label{simulating-the-produced-particles}

From these reaction energies, the $\alpha$ energy is determined through
basic kinematics. Again, any potential angular spreads are not
considered for this stage. As the detected $\alpha$ particles left the
target within a solid angle cone of opening angle
40~mrad, we know that we are limited to a small solid
angle region that we are actually seeing at the detector. The produced
$\alpha$ particles now have an initial energy and a depth within the
target, meaning that each particle also has a known target thickness
that they must pass through.

The generated $\alpha$ particle SRIM files are used to find the exit
energy of the particles in the same way that the final proton energy was
found above. In this case, instead of the energies being defined in
reference to a well-defined input energy, the deviation from the closest
generated $\alpha$ energy in the SRIM files is used, and that file is
used to simulate the exit energy for that particle's depth. The energy
deviation is added back to the simulated energy drawn from the samples
obtained through SRIM to give a final \textit{exit energy} for that
$\alpha$ particle.


\subsubsection{Detector spectrum
comparison}\label{detector-spectrum-comparison}

The generated exit energy spectrum of the $\alpha$ particles describes
the energy of the recoils as they enter St.\ George. The angle of these
particles can be descirbed by the kinematics of the reaction instead of
through the SRIM calculations, so the angle is again ignored for these
particles. Since this analysis is focused on the energy spectrum seen at
the detector, the angular effects do not need to enter into this
calculation.

The energy resolution of the detector was previously measured to be
[VALUE] (see [SECTION]), which we must convolve with the exit
energy of the $\alpha$ particles. For each particle, a random energy
was generated from a Gaussian distribution centered around that
particle's exit energy, according to
\[
    E_{\rm{detector}} = \rm{Normal}(E_{\rm{exit}},\sigma_{\rm{detector}}),
\]
where $\sigma_{\rm{detector}}$ is the energy resolution
of the detector. This final spectrum is a representation of what the
experimenter would see from the detector system and must be related to
the actual experimental data.

Note that this spectrum assumed that every particle that entered our
target reacted, so we can't directly use it to measure the yield.
Relating this simulation to the actual spectrum can be done by scaling
each particle to the cross section at the location it ``reacted'', and
scaling the number of particles at each final energy by a factor
proportional to the determined incident beam current during this
experiment. An alternative approach is to scale both the simulated
spectrum and the measured spectrum to unit area, which would conserve
the number of particles reaching the detector. The previous two options
are useful for checking if the calculated reaction rate in the two cases
are consistent with each other. A final option would be to see if the
two spectra could have been generated from the same underlying
distribution, which avoids having to scale either distribution, through a
KS test.

The simulated distribution is approximately Gaussian, but the actual
distribution is a skewed Gaussian due to the detector response (seen in
energy resolution studies too). Not sure how to model this...


\subsection{SRIM Results}

Insert this here, maybe...?
