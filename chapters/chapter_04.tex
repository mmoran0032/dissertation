\chapter{EXPERIMENTAL PROCEDURE}
\label{ch:experimental-procedure}

% \begin{center}\textit{Alphas have more mass because they spend so much
% time at the gym getting swole. \---{} Laura Moran}\end{center}

An experimental campaign to study the \alpa{} reaction with the St.\
George recoil separator was undertaken at the NSL. Runs were completed
in December 2016 and February 2017, with runs focusing on determining
the correct magnetic fields within St.\ George completed in Fall 2016
and February 2017. Two low energy resonances were measured with beam
currents in the $2-3$~$\mu$A range in February 2017. Studying this
reaction provides a test of the angular and energy acceptances of St.\
George in preparation for studying $(\alpha,\gamma)$ reactions across a
wide range of targets and energies.

The first portion of these runs fall under general St.\ George
commissioning work as discussed in Chapter~\ref{ch:commissioning} and
will not be repeated here. The second portion of the runs involved
characterizing the target and the detector, finalizing the optimal
settings for the separator, and performing the experiment. The reaction
of interest produces $\alpha$ particles in the energy range of $2-3$~MeV
for the desired proton energy range.


\section{Altered Tune}

The magnet settings for St.\ George were determined to transport
$\alpha$ particles, where the entirety of the particle envelope is
described by a characteristic energy and angular [term]. The produced
$\alpha$ particles had a low (values?) energy spread and a high angular
spread, where only those particles emitted within the desired 40~mrad
acceptance cone for St.\ George were tuned to reach the detector focal
plane $F_2$ after the Wien filter and impinge the installed Si strip
detector.

The restrictions on the beam spot for measuring $(\rm{p}\alpha)$
reactions at this focal plane require an approximately symmetric spot
size in both directions and one that is smaller than the physical face
of the detector, whereas the standard tune for studying
$(\alpha,\gamma)$ reactions required that beam spot to be asymmetric
with the beam spot being narrow in the dispersive $x$-plane and tall in
the $y$-plane. The initial COSY code for St.\ George (see
Section~\ref{sec:cosy}) was altered to model the shortened separator and
provide information on the beam characteristics at the new detector
focal plane. The magnetic field settings for the seven quadrupoles
$Q_{1-7}$ were adjusted to transport the recoil particles to the
detector plane with a final beam spot no larger than the face of the Si
detector of $58\times 58$~mm. Final pole tip fields are given in
Table~\ref{tab:poletip}.

\begin{table}
    \begin{center}
        \caption{POLE TIP FIELDS FOR $(\alpha,\gamma)$ AND
            $(\rm{p},\alpha)$ STUDIES}
        \label{tab:poletip}
        \begin{tabular}{cS[table-format=2.9]S[table-format=2.6]}
            \toprule
            \midrule
             & \multicolumn{2}{c}{\textbf{Pole Tip Field [T]}} \\
            \textbf{Quadrupole} & {$(\alpha,\gamma)$} &
            {$(\rm{p},\alpha)$} \\
            \midrule
            1  & -0.16303276 & -0.157\\
            2  &  0.18882363 &  0.187\\
            3  &  0.09384148 &  0.09411\\
            4  & -0.12620402 & -0.04\\
            5  &  0.10032405 &  0.092 \\
            6  &  0.04693654 &  0.0585 \\
            7  &  0.0        & -0.015 \\
            8  & -0.09779179 & \\
            9  &  0.17439627 & \\
            10 &  0.21092228 & \\
            11 & -0.13962355 & \\
            \bottomrule
        \end{tabular}
    \end{center}
\end{table}

For the $(\rm{p},\alpha)$ experiment, the transported $\alpha$ particles
have the properties listed in Table~\ref{tab:alpha_prop}. The incident
proton beam is rejected within the COSY ion optics solution after the
first dipole doublet $B_1B_2$, and the beam properties are not listed
here.

\begin{table}
    \begin{center}
        \caption{ALPHA PARTICLE PROPERTIES}
        \label{tab:alpha_prop}
        \begin{tabular}{cc}
            \toprule
            \midrule
            \textbf{Property [Unit]} & \textbf{Value} \\
            \midrule
            Energy [MeV]        & 2.504 \\
            $\Delta$Energy [\%] & 3 \\
            Angular spread [mrad] & 40 \\
            Target diameter [mm] & 3 \\
            $Q$ [$e$]           & 2 \\
            $B\rho$ [Tm]        & 0.228 \\
            $E\rho$ [MV]        & 4.0 \\
            \bottomrule
        \end{tabular}
    \end{center}
\end{table}

\subsection{Separator Properties}

The energy resolving power is the minimum energy difference required to
resolve a peak from the central image peak assuming that the change in
energy is the only difference between the two peaks. By definition this
quantity is only a first-order value, so only those parameters with a
linear relationship with the position need be considered. The energy
resolving power of the separator in relation to the terms present in the
COSY transport map is defined as
\begin{equation}
    \delta_k(\textrm{RP}) \equiv
        \frac{2\left[(x|x)x_0 + (x|a)a_0\right]}{(x|\delta_k)},
\end{equation}
where $x_0$ and $a_0$ are the initial half-widths for position (in
meters) and angle (in radians), respectively, and the remaining terms
are the values from the transport map. The resolving power is only taken
in the horizontal plane due to the vertical symmetry of the separator.
The terms taken from the transport map are
\begin{align*}
    (x|x) &= 2.261610 \\
    (x|a) &= {-0.1368242} \\
    (x|\delta_k) &= {-0.2774295},
\end{align*}
where signs are conserved for completeness. The maximal deviation caused
by each terms is taken to be a positive value. The half-widths $x_0$ and
$a_0$ are physically limited by the target chamber and taken to be $x_0
= 1.5$~mm and $a_0 = 42$~mrad, giving a resolving power of
$\delta_k(\textrm{RP}) = 0.286$. Since the produced $\alpha$ particles
have an inherent spread in energy due to the incoming beam and the
particles themselves interacting with the target, the energy resolution
should be viewed as the window within which the energies are
indistinguishable. As this window covers the expected energy spread of
the produced $\alpha$ particles, there are no energy corrections
required across the detector strips.

Beam currents at the target location were recorded before and after each
run. For runs lasting longer than 15 minutes, the current was recorded
every 15 minutes. The beam current was seen to fluctuate around the
recorded value by up to 100~nA. For runs with multiple current readings,
the average was taken as the nominal current. Time on target was
recorded by the acquisition system.

\subsection{Beam Reduction}

% Glenlivet 12 Year

Incident proton beam reduction on the order of $10^{10} - 10^{14}$ is
necessary in order to avoid damaging the Si detector and to allow for
the desired alpha peak to be detected. This reduction factor becomes
more important for those off-resonance runs where the count rate of the
produced $\alpha$ particles is much lower.

Due to the location of the Si detector at the post-Wien filter focal
plane $F_2$, incident beam reduction can only be achieved through the
tuning of the separator. In experiments that use the full length of
St.\ George, beam rejection can be obtained through the use of the mass
slits at $F_2$ to stop the beam after the Wien filter. The location of
these slits is the same location as the Si detector, eliminating their
use in this experiment. Due to the larger $\Delta m$ and $\Delta E$
between the incident proton beam and the produced $\alpha$ particles,
the incident beam is adequately reduced at this point by the dipole
magnets and Wien filter.

The Si detector provides the last stage of rejection for the incident
beam. Due to the energy difference between the protons and $\alpha$
particles, the particle peaks will be well separated in the energy
spectra. Low energy tails of the $\alpha$ peak observed during the
energy calibration runs do not have a large effect on the ability to
reject the remainder of the incident beam.


\section{Run Procedure}

For each experimental run, we aim to measure the experimental yield at
the detector in order to compare it to the theoretical yield for a given
angular and energy acceptance. The experimental yield is dependent on
the beam current, the run time, and the target properties.

\subsection{beam preparation/initial tuning}

-- includes magnet recycling if necessary
-- check logbook for additional details

\subsection{detector moving}

-- slowly up to see if we have a massive count rate
-- if we do, something is wrong

\subsection{target in place}

-- use known position from beam preparation

\subsection{detector moving up slowly}

-- could change count rate from beam
-- want to make sure on resonance not a lot of counts

\subsection{shift WF E-field if needed}

-- want count distribution across the strips centered
-- E field set by known equation, but physical electrodes could
   cause some difference from that (plus B fringe field)

\subsection{low detector position}

-- checks vertical extent of the beam (assume vertical symmetry)

\subsection{final run}

-- run for 15 minutes or 20k counts (statistical uncertainty low),
   rinse and repeat
