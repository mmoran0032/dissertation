\chapter{RESULTS}



\section{Detector Spectrum
Verification}\label{detector-spectrum-verification}

Due to the suboptimal energy resolution of the Si strip detector, caused
by the decreased bias voltage and increased leakage current, a
well-defined spectrum for the produced $\alpha$ particles was not
obtained. The spectra for each run and each strip show a wide peak at
the expected location of the $\alpha$ energy peak which we need to
verify that it represents the same underlying expected energy peak
produced in the reaction. This final verification is a convolution of
the incident beam energy, the properties of the target, the cross
section domain that we are probing, and the energy resolution of the
detector. Each of these components will be discussed in turn.

To reduce this check to a smaller subset of possibilities, only the two
resonance energy runs will be checked, but the same process described
below can be repeated for any energy in question, given the existence of
the required files to support the analysis.

This analysis relies on the usage of SRIM/TRIM [CITE] to generate
the files describing the target effects, and those files should be
generated according to the steps described below before the analysis is
begun. The file generation can be done programmatically following the
SRIM user guide, or ``by hand'' using the included GUI, both of which
are standard procedures and will not be discussed here.


\subsection{Assumptions}\label{assumptions}

In order to fully describe reaction process from incident beam to
detector energy spectrum, we must make some assumptions and check that
they are approximately true for the desired reaction.

First, the stopping power of the target must be slowly varying across
the width of the target. This assumption reduces the number of required
SRIM files, as described below. The definition of ``slowly varying'' in
this case is that the stopping power can be locally modeled as a linear
function of the beam energy, and that the coefficient relating the
energy to the stopping power be relatively small.

Second, knowledge of the underlying cross section is required. The
procudure described below can be inverted slightly to move from the
detector energy spectrum to the reaction cross section, knowing all
other parts, but this more-complex procedure was not required for the
reaction in question. Using known resonance properties, the
\react{\mnuc{27}{Al}}{rm{p}}{alpha}{} reaction cross section
can be modeled with the AZURE [CITE] $R$-Matrix code to produce
the file. The energies returned from this code should be transformed
into lab coordinates.

Third, the target should be thin enough that the beam can pass through
the target relatively unimpeded. Should this assumption not be met, the
procedure outlined must be updated to account for the additional
straggling, especially where the exit angle of the particles is
sufficiently spread out.

Finally, the thickness of the target in energy for the incident beam
energies must be known. Given that the thickness in
$mu\rm{g}/\rm{cm}^2$ can be measured using standard techniques,
the thickness in energy can be determined by
\[
    EQUATION HERE,
\]
where $d\rm{E}/dx$ is the stopping power in units of [UNITS]. The
target thickness only needs to be determined for the incident beam, as
this defines the cross section energy domain.

The additional knowledge required to perform the following analysis,
such as the incident beam energy uncertainty and the detector energy
resolution, will not be discussed separately.


\subsection{Required SRIM Files}\label{required-srim-files}

The pre-generation of SRIM files requires special note to justify the
energy and depths used. In cases where the incident beam energy spread
is greater, the target is thicker, the stopping power varies quickly, or
any other deviation from the assumptions above, the procedure below must
be amended and additional SRIM files will most likely need to be
generated to account for those changes.

The incident beam files describe the target effects on the beam energy
at multiple depths within the target. To simplify, all incident
particles are assumed to strike the target perpendicular to the target,
and the angles for each particle following their interaction with the
target are saved but not used within the analysis. While the primary
goal of this portion of the analysis is verifying the energy spectrum,
the slight corrections to the energy loss attributed to the angle of the
particles, these corrections would be minimal and do not affect the
final result. The initial energies chosen for the incident beam protons
are the central resonance energies, and the energies chosen for the
$\alpha$ particles cover the expected range on possible energies based
on previous kinematics studies.

The depths probed for the target are in percents of the total thickness
in Angstroms. For this analysis, depths from 1 to 99\,\% in steps of 2\,\%
were used, which provides adequate coverage of the energy loss through
the target while minimizing the amount of runs required for SRIM. Both
the incident beam protons and the produced $\alpha$ particles require
thicknesses across the entire range of the target thickness. Note that
depths of 0 and 100\,\% were not used; depths of zero percent have no
meaning for the incident beam case, and depths of 100\,\% are unphysical
since the incident particle will have left the target without reacting.

All files simulated 10k particles, and the final energies of those
particles were saved.


\subsection{Simulating the reaction}\label{simulating-the-reaction}

The process below describes how the full reaction, from the incident
beam to the final detector energy spectrum, can be simulated. The SRIM
files required are assumed to be generated beforehand. The process has
been wrapped within the \verb+pyne+ package under the \verb+tree+
submodule, which takes as input the path to the generated files and the
masses and energies of the particles in question.

\subsubsection{Simulating the incident
beam}\label{simulating-the-incident-beam}

The incident beam energy for the resonance scans were determined by the
5U's analyzing magnet settings. The 5U provides a small energy
uncertainty particle beam to the target area, meaning that we have
approximately a monoenergetic beam striking our target. The generated
SRIM files for the incident beam are this central beam energy at
multiple depths within the target. To simulate the known spread of the
beam energy around this central value, the deviation from this central
value was randomly assigned, assuming a Gaussian distribution of
energies. The individual particle energies are given as
\[
    E\_i sim E_{\rm{Resonance}} + \rm{Normal}(0, \sigma),
\]
where $\sigma = 300$~eV is the measured beam energy uncertainty of the 5U.

The cross section that the incident particles will probe is defined by
the initial energy of the beam and the thickness of the target in units
of energy. For the target in question, the energy thickness is
approximately 10~keV for both resonances. The cross
section within this energy domain was determined for the central
resonance energy, which is an approximation that initially ignores the
energy spread of the beam. As this energy spread is less than 0.1\,\%,
the effect of this choice is minimal and helps to simplify the process
since a single probability distribution for the cross section needs to
be determined. The cross section within the energy range is divided into
energy bins of 1\,\%, with the values interpolated where the results of
the $R$-matrix calculation do not match with the integer depth values.
These depths are then randomly chosen based on the normalized cross
section in that region. An example is shown in [FIGURE].

[FIGURE]

Once a simulated depth is assigned to each particle, the SRIM results at
that depth are used to simulate the final energy of that particle before
it reacts with the target. As the energy distributions returned by SRIM
are not analytic, we can draw from the 10k simulated values, assuming
equal probability for each, which for large sample sizes is
approximately equivalent. Each incident particle is now associated with
an initial energy deviation from the resonance energy, the depth at
which it reacts, and a final energy at that depth. The initial simulated
energy is added to the final energy to give the textit{reaction
energy} of the incident proton.


\subsubsection{Simulating the produced
particles}\label{simulating-the-produced-particles}

From these reaction energies, the $\alpha$ energy is determined through
basic kinematics. Again, any potential angular spreads are not
considered for this stage. As the detected $\alpha$ particles left the
target within a solid angle cone of opening angle
40~mrad, we know that we are limited to a small solid
angle region that we are actually seeing at the detector. The produced
$\alpha$ particles now have an initial energy and a depth within the
target, meaning that each particle also has a known target thickness
that they must pass through.

The generated $\alpha$ particle SRIM files are used to find the exit
energy of the particles in the same way that the final proton energy was
found above. In this case, instead of the energies being defined in
reference to a well-defined input energy, the deviation from the closest
generated $\alpha$ energy in the SRIM files is used, and that file is
used to simulate the exit energy for that particle's depth. The energy
deviation is added back to the simulated energy drawn from the samples
obtained through SRIM to give a final \textit{exit energy} for that
$\alpha$ particle.


\subsubsection{Detector spectrum
comparison}\label{detector-spectrum-comparison}

The generated exit energy spectrum of the $\alpha$ particles describes
the energy of the recoils as they enter St.\ George. The angle of these
particles can be descirbed by the kinematics of the reaction instead of
through the SRIM calculations, so the angle is again ignored for these
particles. Since this analysis is focused on the energy spectrum seen at
the detector, the angular effects do not need to enter into this
calculation.

The energy resolution of the detector was previously measured to be
[VALUE] (see [SECTION]), which we must convolve with the exit
energy of the $\alpha$ particles. For each particle, a random energy
was generated from a Gaussian distribution centered around that
particle's exit energy, according to
\[
    E_{\rm{detector}} = \rm{Normal}(E_{\rm{exit}},\sigma_{\rm{detector}}),
\]
where $\sigma_{\rm{detector}}$ is the energy resolution
of the detector. This final spectrum is a representation of what the
experimenter would see from the detector system and must be related to
the actual experimental data.

Note that this spectrum assumed that every particle that entered our
target reacted, so we can't directly use it to measure the yield.
Relating this simulation to the actual spectrum can be done by scaling
each particle to the cross section at the location it ``reacted'', and
scaling the number of particles at each final energy by a factor
proportional to the determined incident beam current during this
experiment. An alternative approach is to scale both the simulated
spectrum and the measured spectrum to unit area, which would conserve
the number of particles reaching the detector. The previous two options
are useful for checking if the calculated reaction rate in the two cases
are consistent with each other. A final option would be to see if the
two spectra could have been generated from the same underlying
distribution, which avoids having to scale either distribution, through a
KS test.

The simulated distribution is approximately Gaussian, but the actual
distribution is a skewed Gaussian due to the detector response (seen in
energy resolution studies too). Not sure how to model this...


\subsection{SRIM Results}

Insert this here, maybe...?

